 \documentclass[a4paper, 12pt, twoside, openright]{report}
\usepackage[T1]{fontenc}
\usepackage[utf8]{inputenc}
\usepackage[english]{babel}
\usepackage{graphicx}
%\usepackage{caption}
\usepackage{float}
\usepackage[amssymb]{SIunits}
\usepackage{amsmath, amssymb}
%\usepackage[makeroom]{cancel}
%\usepackage{subscript}
%\usepackage{empheq}
\graphicspath{{./}{Immagini/}}

\newcommand{\sub}{\textsubscript}
\newcommand{\super}{\textsuperscript}
\setlength{\topmargin}{-1.5cm}
\setlength{\textheight}{24.5cm}
\setlength{\textwidth}{17.5cm}
\setlength{\oddsidemargin}{-1.0cm}
\setlength{\evensidemargin}{-1.0cm}
\begin{document}

\tableofcontents

\chapter{Scaling}

\section{Introduction}

\textbf{Scaling and scalable devices}: in this report it will be analyzed how technology should be modified to allow scaling of devices and deeper integration.
In figure xxx a comparison between a very old process (3 um channel length) with a relatively newer one (250 nm channel length) is shown and it is possible to notice that device dimensions have become smaller and this fact introduces a lot of issues.
First it's important to say that devices have been scaled much more than interconnections; in other words, interconnects cannot be scaled with the same policy used for electronic devices. This could be a problem since in a technology where devices and interconnects have comparable dimensions, interconnects parasitics do not have influence on the overall performance; on the contrary, in a technology where the devices and the interconnections are not scaled with the same rate, as in this case (devices smaller than interconnects), \textbf{parasitics have a significant influence on the perfomance}. This is one of the consequences of scaling that can only be partially solved by technology, so it is essential to act from the design point of view in order to limit the effect of parasitics. The other consequence of the scaling is related to \textbf{Short Channel Effects (SCE)}. Short channel effect is a phenomenon that is attributable to short channel devices, in other words when the channel length is of the same order of magnitude as the depletion-layer widths (xdD, xdS) of the source and drain junction. One of the most important consequences of this effect is a change of the threshold voltage and so a less strong electrostatic control of the channel. The SCE can be strongly limited by appropriate technological choices (this is the main target of integrated technologies); \\

The \textbf{ITRS (International Technology Roadmap for Semiconductors)} classifies technologies in the following way (2011 edition):

	\begin{itemize}
	\item \textbf{HP}: High Performace processes have as main aim to reduce the gate delay even if the power dissipation can be high; it is clear that this kind of processes cannot be used to produce portable devices, they can be used only for mainframes or data centers; in the last years HP processes have reached the natural end;
	\item \textbf{LOP}: Low Operating Power processes are used for relatively high performance mobile applications where the focus is on reduced operating power (acceptable performance and acceptable dynamic power consumption);
	\item \textbf{LSTP}: Low STandby Power processes have as main aim to reduce the static power for low performance and low battery capacity applications (for example mobile phones); so, LSTP processes try to reduce the power consumption when the device is in standby mode, paying in terms of performance (reduced clock frequency).
	\end{itemize}

Nowadays, HP is becoming marginal with respect to the other process families, because modern technology tends to pay attention in particular to the reduction of leakage (by avoiding to reduce too much the threshold voltage), since the supply voltage has been scaled more and more over the years. For these reasons, LOP technology is playing a key role: it has become the new HP family in \textbf{ITRS} (2013 edition), which will be used throughout this chapter for technological parameters. As a consequence, the old LSTP has become the new LP family. In conclusion, from now on this chapter will refer to two different technological families:

\begin{itemize}
\item \textbf{HP}: High Performace processes have as main goal to improve perfomance even though power consumption increases.
\item \textbf{LP}: Low Power processes have aims at reducing the static power for low performance and low battery capacity applications.
\end{itemize}

The \textbf{ITRS} makes previsions on which direction research will proceed regarding to different areas of technology up to about 12 years into the future. Until now, these previsions have been really accurate. In the following four tables, two per each technological family, there are all predicted parameters. For some of them the manufacturable solutions are known and they are being optimized, some others already have manufacturable solution, whereas for a few parameters there are not any manufacturable solutions available so far. A color legend will be used to distinguish among the different cases:

%%INSERIRE LEGENDA A COLORI CON QUADRATINI

As already stated above, there are two tables per technological family. That is due to different technological processes that can be exploited to implement both HP and LP families, which consist of Bulk technology and Multi-Gate (MG) technology. The former is going to disappear by the end of 2017 and the latter has become dominant in the last few years, whereas SOI technology has already become out-of-date. As a matter of fact, all the major foundaries in the world have already switched to MG technology, focusing their production on finFET-on-Bulk devices.

\begin{table}[h]
\centering
\resizebox{\textwidth}{!}{\begin{tabular}{||l||c|c|c|c|c|c|c|c|c|c|c|c||}
\hline
Year&2013&2014&2015&2016&2017&2018&2019&2020&2021&2022&2023&2024\\
\hline
$L_{g}$ [nm]&20&18&16.7&15.2&13.9&12.7&11.6&10.6&9.7&8.8&8&7.3\\
\hline
$L_{ch}$ [nm]&16&14.4&13.4&12.2&11.1&10.2&9.3&8.5&7.8&7&6.4&5.8\\
\hline
$V_{dd}$ [V]&0.86&0.85&0.83&0.81&0.8&0.78&0.77&0.75&0.74&0.72&0.71&0.69\\
\hline
EOT [nm]&0.8&0.77&0.73&0.7&0.67&0.64&0.61&0.59&0.56&0.54&0.51&0.49\\
\hline
$\epsilon_{HK}$&12.5&13&13.5&14&14.5&15&15.5&16&16.5&17&17.5&18\\
\hline
$T_{HK}$ [nm]&2.56&2.57&2.53&2.51&2.49&2.46&2.42&2.42&2.37&2.35&2.29&2.26\\
\hline
$Doping_{ch}$ [$10^{18}$/$cm^{3}$]&6&7&7.7&8.4&9&&&&&&&\\
\hline
CET [nm]&1.1&1.07&1.03&1&0.97&0.94&0.91&0.89&0.86&0.84&0.81&0.79\\
\hline
$C_{ch}$ [fF/$\mu$m]&0.502&0.465&0.448&0.42&0.396&0.373&0.352&0.329&0.311&0.289&0.273&0.255\\
\hline
$\mu$ [$cm^{2}$/V*s]&400&400&400&400&400&&&&&&&\\
\hline
$I_{off}$ [pA/$\mu$m]&100&100&100&100&100&100&100&100&100&100&100&100\\
\hline
$I_{d,sat}$ [$\mu$A/$\mu$m]&1348&1355&1340&1295&1267&&&&&&&\\
\hline
$V_{t,lin}$ [V]&0.306&0.327&0.334&0.357&0.378&&&&&&&\\
\hline
$V_{t,sat}$ [V]&0.19&0.2&0.206&0.218&0.23&&&&&&&\\
\hline
S/D $R_{s}$ [$\omega$*$\mu$m]&188&179&171&162&156&&&&&&&\\
\hline
$C_{fringing}/C_{intrinsic}$ &1.2&1.3&1.4&1.5&1.6&1.7&1.8&1.9&2&2&2&2\\
\hline
$C_{fringing}$ [fF/$\mu$m]&0.6&0.6&0.63&0.63&0.63&0.63&0.63&0.62&0.62&0.58&0.55&0.51\\
\hline
$C_{g,total}$ [fF/$\mu$m]&1.1&1.07&1.07&1.05&1.03&1.01&0.99&0.95&0.93&0.87&0.82&0.77\\
\hline
$CV^{2}$ [fJ/$\mu$m&0.82&0.77&0.74&0.69&0.66&0.61&0.58&0.54&0.51&0.45&0.41&0.36\\
\hline
$\tau$ [ps]&0.705&0.67&0.666&0.656&0.65&&&&&&&\\
\hline
1/$\tau$ [1/ps]&1.42&1.49&1.5&1.52&1.54&&&&&&&\\
\hline
$I_{d,satn}/I_{d,satp}$&1.25&1.24&1.22&1.21&1.2&1.19&1.18&1.16&1.15&1.14&1.13&1.12\\
\hline
\end{tabular}}
\caption{\textbf{Bulk process for HP (ITRS 2013)}}
\end{table}

\begin{table}[h]
\centering
\resizebox{\textwidth}{!}{\begin{tabular}{||l||c|c|c|c|c|c|c|c|c|c|c|c||}
\hline
Year&2013&2014&2015&2016&2017&2018&2019&2020&2021&2022&2023&2024\\
\hline
$L_{g}$ [nm]&20&18&16.7&15.2&13.9&12.7&11.6&10.6&9.7&8.8&8&7.3\\
\hline
$L_{ch}$ [nm]&16&14.4&13.4&12.2&11.1&10.2&9.3&8.5&7.8&7&6.4&5.8\\
\hline
$V_{dd}$ [V]&0.86&0.85&0.83&0.81&0.8&0.78&0.77&0.75&0.74&0.72&0.71&0.69\\
\hline
EOT [nm]&0.8&0.77&0.73&0.7&0.67&0.64&0.61&0.59&0.56&0.54&0.51&0.49\\
\hline
$\epsilon_{HK}$&12.5&13&13.5&14&14.5&15&15.5&16&16.5&17&17.5&18\\
\hline
$T_{HK}$ [nm]&2.56&2.57&2.53&2.51&2.49&2.46&2.42&2.42&2.37&2.35&2.29&2.26\\
\hline
$Doping_{ch}$ [$10^{18}$/$cm^{3}$]&0.1&0.1&0.1&0.1&0.1&0.1&0.1&0.1&0.1&0.1&0.1&0.1\\
\hline
$T_{body}$ [nm]&6.4&5.8&5.3&4.9&4.4&4.1&3.7&3.4&3.1&2.8&2.6&2.3\\
\hline
CET [nm]&1.1&1.07&1.03&1&0.97&0.94&0.91&0.89&0.86&0.84&0.81&0.79\\
\hline
$C_{ch}$ [fF/$\mu$m]&0.502&0.465&0.448&0.42&0.396&0.373&0.352&0.329&0.311&0.289&0.273&0.255\\
\hline
$\mu$ [$cm^{2}$/V*s]&250&250&250&250&250&200&200&200&200&200&200&150\\
\hline
$I_{off}$ [pA/$\mu$m]&100&100&100&100&100&100&100&100&100&100&100&100\\
\hline
$I_{d,sat}$ [$\mu$A/$\mu$m]&1670&1680&1700&1660&1660&1610&1600&1480&1450&1350&1330&1170\\
\hline
$V_{t,lin}$ [V]&0.219&0.225&0.231&0.239&0.264&0.266&0.265&0.276&0.295&0.303&0.306&0.319\\
\hline
$V_{t,sat}$ [V]&0.174&0.179&0.179&0.186&0.191&0.196&0.199&0.205&0.214&0.219&0.222&0.233\\
\hline
S/D $R_{s}$ [$\omega$*$\mu$m]&128&146&130&126&124&117&120&116&112&111&113&123\\
\hline
$C_{fringing}/C_{intrinsic}$ &1.2&1.3&1.4&1.5&1.6&1.7&1.8&1.9&2&2&2&2\\
\hline
$C_{fringing}$ [fF/$\mu$m]&0.6&0.6&0.63&0.63&0.63&0.63&0.63&0.62&0.62&0.58&0.55&0.51\\
\hline
$C_{g,total}$ [fF/$\mu$m]&1.1&1.07&1.07&1.05&1.03&1.01&0.99&0.95&0.93&0.87&0.82&0.77\\
\hline
$CV^{2}$ [fJ/$\mu$m]&0.82&0.77&0.74&0.69&0.66&0.61&0.58&0.54&0.51&0.45&0.41&0.36\\
\hline
$\tau$ [ps]&0.569&0.541&0.525&0.512&0.496&0.488&0.474&0.483&0.477&0.463&0.437&0.451\\
\hline
1/$\tau$ [1/ps]&1.76&1.85&1.91&1.95&2.02&2.05&2.11&2.07&2.1&2.16&2.29&2.22\\
\hline
$I_{d,satn}/I_{d,satp}$&1.25&1.24&1.22&1.21&1.2&1.19&1.18&1.16&1.15&1.14&1.13&1.12\\
\hline
\end{tabular}}
\caption{\textbf{MG process for HP (ITRS 2013)}}
\end{table}

\begin{table}[h]
\centering
\resizebox{\textwidth}{!}{\begin{tabular}{||l||c|c|c|c|c|c|c|c|c|c|c|c||}
\hline
Year&2013&2014&2015&2016&2017&2018&2019&2020&2021&2022&2023&2024\\
\hline
$L_{g}$ [nm]&23&21&19&18&16&14.6&13.3&12.2&11.1&10.1&9.3&8.5\\
\hline
$L_{ch}$ [nm]&18.4&16.8&15.2&14.4&12.8&11.7&10.6&9.8&8.9&8.1&7.4&6.8\\
\hline
$V_{dd}$ [V]&0.86&0.85&0.83&0.81&0.8&0.78&0.77&0.75&0.74&0.72&0.71&0.69\\
\hline
EOT [nm]&0.8&0.77&0.73&0.7&0.67&0.64&0.61&0.59&0.56&0.54&0.51&0.49\\
\hline
$\epsilon_{HK}$&12.5&13&13.5&14&14.5&15&15.5&16&16.5&17&17.5&18\\
\hline
$T_{HK}$ [nm]&2.56&2.57&2.53&2.51&2.49&2.46&2.42&2.42&2.37&2.35&2.29&2.26\\
\hline
$Doping_{ch}$ [$10^{18}$/$cm^{3}$]&5&6&7&7.7&8.4&&&&&&&\\
\hline
CET [nm]&1.1&1.07&1.03&1&0.97&0.94&0.91&0.89&0.86&0.84&0.81&0.79\\
\hline
$C_{ch}$ [fF/$\mu$m]&0.577&0.542&0.509&0.497&0.455&0.429&0.404&0.379&0.356&0.332&0.317&0.297\\
\hline
$\mu$ [$cm^{2}$/V*s]&400&400&400&400&400&&&&&&&\\
\hline
$I_{off}$ [pA/$\mu$m]&10&10&20&20&50&&&&&&&\\
\hline
$I_{d,sat}$ [$\mu$A/$\mu$m]&490&459&456&485&422&&&&&&&\\
\hline
$V_{t,lin}$ [V]&0.619&0.639&0.636&0.676&0.647&&&&&&&\\
\hline
$V_{t,sat}$ [V]&0.528&0.543&0.533&0.54&0.53&&&&&&&\\
\hline
S/D $R_{s}$ [$\omega$*$\mu$m]&188&179&171&162&156&&&&&&&\\
\hline
$C_{fringing}/C_{intrinsic}$ &1.1&1.2&1.3&1.4&1.5&1.6&1.7&1.8&1.9&2&2&2\\
\hline
$C_{fringing}$ [fF/$\mu$m]&0.64&0.65&0.66&0.7&0.68&0.69&0.69&0.68&0.68&0.66&0.63&0.59\\
\hline
$C_{g,total}$ [fF/$\mu$m]&1.21&1.19&1.17&1.19&1.14&1.12&1.09&1.06&1.03&1&0.95&0.89\\
\hline
$CV^{2}$ [fJ/$\mu$m]&0.9&0.86&0.81&0.78&0.73&0.68&0.65&0.6&0.57&0.52&0.48&0.42\\
\hline
$\tau$ [ps]&2.128&2.08&2.132&1.992&2.159&&&&&&&\\
\hline
1/$\tau$ [1/ps]&0.47&0.45&0.47&0.5&0.46&&&&&&&\\
\hline
$I_{d,satn}/I_{d,satp}$&1.27&1.26&1.25&1.24&1.22&1.21&1.2&1.19&1.18&1.16&1.15&1.14\\
\hline
\end{tabular}}
\caption{\textbf{Bulk process for LP (ITRS 2013)}}
\end{table}

\begin{table}[h]
\centering
\resizebox{\textwidth}{!}{\begin{tabular}{||l||c|c|c|c|c|c|c|c|c|c|c|c||}
\hline
Year&2013&2014&2015&2016&2017&2018&2019&2020&2021&2022&2023&2024\\
\hline
$L_{g}$ [nm]&23&21&19&18&16&14.6&13.3&12.2&11.1&10.1&9.3&8.5\\
\hline
$L_{ch}$ [nm]&18.4&16.8&15.2&14.4&12.8&11.7&10.6&9.8&8.9&8.1&7.4&6.8\\
\hline
$V_{dd}$ [V]&0.86&0.85&0.83&0.81&0.8&0.78&0.77&0.75&0.74&0.72&0.71&0.69\\
\hline
EOT [nm]&0.8&0.77&0.73&0.7&0.67&0.64&0.61&0.59&0.56&0.54&0.51&0.49\\
\hline
$\epsilon_{HK}$&12.5&13&13.5&14&14.5&15&15.5&16&16.5&17&17.5&18\\
\hline
$T_{HK}$ [nm]&2.56&2.57&2.53&2.51&2.49&2.46&2.42&2.42&2.37&2.35&2.29&2.26\\
\hline
$Doping_{ch}$ [$10^{18}$/$cm^{3}$]&0.1&0.1&0.1&0.1&0.1&0.1&0.1&0.1&0.1&0.1&0.1&0.1\\
\hline
$T_{body}$ [nm]&7.4&6.7&6.1&5.8&5.1&4.7&4.3&3.9&3.6&3.2&3&2.7\\
\hline
CET [nm]&1.1&1.07&1.03&1&0.97&0.94&0.91&0.89&0.86&0.84&0.81&0.79\\
\hline
$C_{ch}$ [fF/$\mu$m]&0.577&0.542&0.509&0.497&0.455&0.429&0.404&0.379&0.356&0.332&0.317&0.297\\
\hline
$\mu$ [$cm^{2}$/V*s]&375&375&375&375&375&375&375&300&300&300&300&300\\
\hline
$I_{off}$ [pA/$\mu$m]&10&10&10&10&10&10&10&20&20&20&20&20\\
\hline
$I_{d,sat}$ [$\mu$A/$\mu$m]&643&610&618&589&574&556&550&533&537&461&458&395\\
\hline
$V_{t,lin}$ [V]&0.483&0.492&0.492&0.496&0.507&0.507&0.51&0.501&0.507&0.521&0.511&0.52\\
\hline
$V_{t,sat}$ [V]&0.446&0.453&0.453&0.454&0.461&0.459&0.461&0.447&0.446&0.454&0.453&0.46\\
\hline
S/D $R_{s}$ [$\omega$*$\mu$m]&128&146&130&126&124&117&120&116&112&111&113&123\\
\hline
$C_{fringing}/C_{intrinsic}$ &1.1&1.2&1.3&1.4&1.5&1.6&1.7&1.8&1.9&2&2&2\\
\hline
$C_{fringing}$ [fF/$\mu$m]&0.64&0.65&0.66&0.7&0.68&0.69&0.69&0.68&0.68&0.66&0.63&0.59\\
\hline
$C_{g,total}$ [fF/$\mu$m]&1.21&1.19&1.17&1.19&1.14&1.12&1.09&1.06&1.03&1&0.95&0.89\\
\hline
$CV^{2}$ [fJ/$\mu$m]&0.9&0.86&0.81&0.78&0.73&0.68&0.65&0.6&0.57&0.52&0.48&0.42\\
\hline
$\tau$ [ps]&1.622&1.661&1.573&1.64&1.587&1.564&1.525&1.491&1.424&1.556&1.474&1.557\\
\hline
1/$\tau$ [1/ps]&0.62&0.6&0.64&0.61&0.63&0.64&0.66&0.67&0.7&0.64&0.68&0.64\\
\hline
$I_{d,satn}/I_{d,satp}$&1.27&1.26&1.25&1.24&1.22&1.21&1.2&1.19&1.18&1.16&1.15&1.14\\
\hline
\end{tabular}}
\caption{\textbf{MG process for LP (ITRS 2013)}}
\end{table}

\noindent Let's compare the values that we have in the three tables.

\paragraph{Gate Length~:} The gate length is defined as the distance between the source and drain terminals. Taking, for example, the 2012 node, we can observe that the gate length is 14 nm for the HP logic and it increases in the LOP logic (18 nm), reaching a higher value for the LSTP logic (20 nm). What we can understand is that to get high performance it is necessary to reduce the channel length. First of all, reducing the channel length allows to have \emph{smaller parasitic capacitances} and, above all, we get a \emph{higher current}. So, generally a higher speed is the result of two main components: smaller capacitances and greater current.



\paragraph{Power Supply Voltage $V_{DD}$~:} 
What we can expect is that  $V_{DD}$ should be more scaled in LOP than in HP since, in the former case, we want to have low operating power consumption. 
Speaking of which, we recall that the dynamic power is $P_{D} = C_L \ \alpha \  f_{CK} \ V_{DD}^2$, so, reducing $V_{DD}$ we strongly reduce the dynamic power. Regarding LSTP, we need to have a higher $V_{DD}$ in order to increase $V_{TH}$ and reduce leakages (drain-source junction leakages and channel leakages, $I_{OFF}$). 

Now we focus on the channel leakages, represented by $I_{OFF}$. If we want to transform an HP process in an LSTP one, we need to \textbf{reduce} $I_{OFF}$. When, in an nMOS, $V_G = 0$, $I_{OFF}$ should be zero. How can we reduce this leakage current? The solution is to \textbf{increase}$ V_{TH}$: in this way, when $V_{G} = 0$, we are sure that the transistor is really off. But, increasing $V_{TH}$ means reducing $I_{D}$. In order to compensate this reduction we need to increase $V_{DD}$. Infact:

\begin{equation}
I_{DS,\ sat} = \frac{\mu_n C_{OX} W}{2 L} (V_{DD} - V_{TH})^2
\label{ids_sat}
\end{equation}

Looking at the equation (\ref{ids_sat}) it is clear that increasing $V_{DD}$ and $V_{TH}$ by the same amount allows to maintain constant the value of $I_{D}$.



\paragraph{Parameters $I_{on}$, $I_{off}$, $\tau$~:} These three important parameters are usually used to compare technologies. 

$I_{on}$ ($\mu A/\mu m$) is the maximum drive current per unit of width of a transistor. $I_{on}$ can be measured connecting gate and drain so that the transistor is in saturation. Then, considering a $1 \mu m$ width transistor, we measure the maximum saturation current flowing through the drain node.

	\begin{figure}[h]
	\centering
	\includegraphics[width=0.85\textwidth]{figIon}
	\caption{Ion current in function of time.}
	\label{}
	\end{figure}

Analizing the line related to $I_{on}$, for each technology, we can see that scaling gives better values of this current. Moreover, what we can observe comparing the three tables is that $I_{on}$ is higher in HP, while in LOP and in LSTP it is scaled. Today, an acceptable value for $I_{on}$ is approximately 1 $mA/\mu m$.

$I_{off}$ ($nA/\mu m$) is the current flowing in the transistor while it is open; in an ideal transistor $I_{off}$ should be zero. However, in HP technology $I_{off}$ has been characterized by a significant increase year per year.\\
$I_{off}$ can be measured, for a 1$\mu m$ width transistor, connecting gate and source to GND and drain to $V_{DD}$.

	\begin{figure}[h]
	\centering
	\includegraphics[width=0.85\textwidth]{figIoff}
	\caption{Ioff current in function of time.}
	\label{}
	\end{figure}

We can notice that, differently from $I_{on}$, the graph that represents the trend of the subthreshold current in time is in semi-logarithmic scale which means that the range of variation of $I_{off}$ is larger than the range of $I_{on}$. Analyzing the above graph, it can be seen that in LSTP the subthreshold current is reduced, approximately, of three order of magnitude with respect to HP.

An important parameter is the ratio between $I_{on}$ and $I_{off}$. In an ideal technology this ratio should be infinite. For the HP technology $\frac{I_{on}}{I_{off}} \approx 10^4$. In the LOP logic it is roughly $10^5$. Finally for the LSTP tech. it is approximately $10^7$. It is clear that the higher is the ratio, the higher is $I_{on}$ and this is equivalent to higher performances.

An efficient method to \textbf{measure the performance of a process} is to evaluate $\tau$, the \textbf{intrinsic time}. $\tau$ can be seen as the time required to remove the charge stored in the gate capacitance (of a transistor with W = 1$\mu m$) by mean of the $I_{on}$ current.

	\begin{figure}[h]
	\centering
	\includegraphics[width=0.85\textwidth]{figtau}
	\caption{Intrinsic delay in function of time.}
	\label{tau}
	\end{figure}

The dotted capacitor is the equivalent input capacitance of the transistor and it acts as load capacitance for the other transistor (the one of which we want to measure $\tau$). $\tau$ can be evaluated (this is an \emph{approximation}) as the constant current transient (see the equation of $\tau$ in the figure above). This expression is just an approximation (useful to easily estimate $\tau$) since in a real transient the current varies during the discharge of the capacitor, it is not constant. What we can say is that $\tau$ can be reduced:

	\begin{enumerate}
		\item reducing parasitics since C is the gate capacitance (that is a parasitic cap.);
		\item increasing $I_{on}$.
	\end{enumerate}

Since the input capacitance depends only on the geometries, generally, if we want to improve the intrinsic time we have to act on $I_{on}$. This is another reason why, in advanced processes, the main aim is to increase $I_{on}$ as much as possible. $\tau$ and the clock frequency are \emph{not} the same thing but, increasing the former can lead, probably, to an improvement of the latter.\\
Observing the graph in figure \ref{tau}, it can been seen that the ratio between $\tau$ in these three technology is kept node by node: $\tau_{HP} \approx \frac{1}{3} \tau_{LSTP}$.


\paragraph{Threshold voltage~  $V_{T}$~:} To define a good technology the threshold voltage can be considered as target: on $V_{T}$ depends all the other choices that can be taken. Starting from the HP technlogy, we can observe that the value of $V_{T}$ is quite low (it is around or under 200 mV).\\
What has been kept constant during the years is the ratio $V_{DD}/V_{T}$. This ratio is the fundamental parameter to \textbf{distinguish a technological process} from another.

	\begin{figure}[H]
	\centering
	\includegraphics[width=0.75\textwidth]{figVddVt}
	\caption{$V_{DD}/V_{T}$ in function of time.}
	\label{tau}
	\end{figure}

In HP we have that $\frac{V_{DD}}{V_{T}} \approx 6/7$. Generally, when the \textbf{ratio is high} it means that the considered process will be characterized by a \textbf{high current}. For many years (during the golden age of the CMOS technology: '70, '80, beginning of '90) $V_{DD}$ was 5 V and $V_{T}$ 1 V, so their ratio was equal to 5.\\
In LOP the ratio is lower and it is, approximately, 3 since $V_{T}$ is higher.\\
In LSTP the quantity $V_{DD}/V_{T}$ is slightly smaller, approximately less than 3 ($\approx$ 2.5).

\paragraph{Third row: EOT - $T_{ox}$ (\r{A}).} The thickness of the gate oxide is reduced node by node. For example, in the HP process, in 2012, $T_{ox}$ is equal to 7 \r{A}; considering that the $SiO_{2}$ lattice constant is 2.7, having a thickness of 7 \r{A} means having roughly 3 atomic layers. This thickness is not enough to prevent \emph{tunneling} and this causes an \textbf{increase of the gate current} which, instead, should be close to zero since the gate should be isolated. This could be another problem for deeply scaled devices.\\
Especially in the last nodes, in all the three technologies, $I_{gate}$ (expressed as current density, A/cm$^{2}$) has high values. In particular, what we can see is that HP has the highest $I_{gate}$ values among the three technologies. In LSTP we have minimum values and, consequently, a thicker gate oxide.\\
\textbf{Why in a scaled device the gate oxide thickness must be reduced?} The answer is that reducing $t_{ox}$, although it causes an increase of parasitic capacitances, it increases, also, $C_{ox}$ (gate capacitance); infact:

\begin{equation}
C_{ox} = \frac{\epsilon_{ox}}{t_{ox}}
\label{}
\end{equation}

Moreover, as shown in the equation below, $I_{ON}$ linearly depends on $C_{ox}$ so, increasing the latter means increasing the current.

\begin{equation}
I_{ON} = \frac{\mu_n C_{ox} W}{2 L} (V_{DD} - V_{TH})^2
\label{}
\end{equation}

So, with a \textbf{thinner oxide} we are able to control, in a better way, the quantity of charge in the channel and this means \textbf{higher trasconductance}, that is \textbf{higher current}.\\
However, this is not the main motivation to reduce the oxide thickness in a scaled process. The main motivation is that reducing $t_{ox}$ allows to reduce $V_{TH}$:

\begin{equation}
V_{TH} = V_{FB} + 2 \Phi_{P} + \frac{1}{C_{ox}} \sqrt{2 q \epsilon_s N_A 2 \Phi_{P}}
\label{V_th}
\end{equation}

So, the followed sequence is:

\begin{equation}
\text{SCALING} \ \Rightarrow \ t_{ox} \ \downarrow \ \Rightarrow \ V_{TH} \ \downarrow \ \Rightarrow \ C_{ox} \ \uparrow
\end{equation}

We said that a drawback of the reduction of $t_{ox}$ is the strong increase of the gate current: in order to avoid this unwanted consequence, the gate structure and materials should be completely changed.

\textbf{Indepentently of the technological family (HP, LOP or LSTP) scaling means higher} $I_{OFF}$. New processes try to maintan limited, under control, $I_{OFF}$. For the moment we consider $I_{OFF} = I_{SUBV_{T}}$; so we are not considering other leakage sources as junction leakages, gate leakages, \dots.

\subsection{Subthreshold current}

It is very important to understand which are the physical mechanisms that produce $I_{SubVT}$.

	\begin{figure}[H]
	\centering
	\includegraphics[scale=0.44]{07_MOS_band_diag}
	\caption{Band diagram of a n-MOS}
	\label{}
	\end{figure}

$V_{S}$ is the voltage drop across silicon, when we describe a MOS system, that is equivalent to $\Psi(0)$.\\
$V_{ox}$ is the oxide voltage drop.

Considering this voltage references, we draw the most important figure for a MOS system which is $|Q_t|$ ($Q_{t}$ is the total charge of the silicon in the MOS system):

	\begin{figure}[H]
	\centering
	\includegraphics[scale=0.43]{07_total_charge}
	\caption{The quantity of charge in function of $V_{S}$.}
	\label{}
	\end{figure}

This graph is related to a \textbf{p-type device}.\\
The red curve corresponds to $Q_{t}$, not $|Q_{t}|$. $Q_{t}$ is the sum of ions (depletion charge) and free carriers (electrons and holes).\\
Between the accumulation and the depletion regions there is a point where the total charge in the silicon is zero. At this point we have that $V_S = 0$ so, this should be the true origin of the graph. When $V_S = 0$ we are applying at the gate (so at the entire MOS system) a voltage equal to $V_{FB}$.\\
If we want to increase the quantity of positive charge in a p-type device we need to increase the number of holes.\\
The maximum voltage drop applicable to silicon is $2\Phi_P$ (the typical value of $\Phi_P$ is a p-type silicon is roughly 0.35-0.40). Starting from the point in which $V_S = 2\Phi_P$ the system is in strong inversion. In strong inversion region the channel is created and the quantity of charge in the channel increases linearly with the gate voltage. In the \textbf{strong inversion region} we evaluate $I_{on}$: infact when $V_S = 2\Phi_P$, we are applying at the gate a voltage equal to $V_{TH}$, at least. In the \textbf{weak inversion region}, where $ \Phi_P < V_S < 2\Phi_P$ and $V_G < V_{TH}$, we evaluate $I_{off}$. When the system is in weak inversion, $|Q_{t}|$ is not enough high to provide $I_{on}$ and not enough low to consider $I_{off}$ = 0 (the transistor is not completely off). It can be considered as a transition region.\\
Only when $V_S < \Phi_P$ we can consider the transistor is completely off.

Let's consider this figure:

	\begin{figure}[H]
	\centering
	\includegraphics[scale=0.6]{07_graph1}
	\caption{}
	\label{}
	\end{figure}

When $V_G > V_{TH}$ we have $I_{on}$ current that is mainly drift current; when $V_G < V_{TH}$ we have $I_{off}$ current that is mainly diffusion current. This can be grater than zero even if the total quantity of carriers is very small because what is relevant is the variation of the concentration, not its absolute value. When we consider $I_{on}$ we generally neglect diffusion current, but, actually, it is present and it is due to the inversion charge (red) when the system is in weak inversion:

	\begin{figure}[H]
	\centering
	\includegraphics[scale=0.5]{07_graph2}
	\caption{Depletion and inversion layer in the channel.}
	\label{}
	\end{figure}

The depletion charge increase moving from source to drain and the inversion charge decrease moving from S to D.\\
The concentration $Q_{n}$ varies along the channel and its derivative $\frac{\partial Q_n (y)}{\partial y}$ becomes dominant.
(We recall that the expression of $Q_{n}$ in strong inversion is $Q_{n} = C_{ox} (V_{GS} - V_{TH})$)\\
In order to  $I_{off}$, we need to:

\begin{enumerate}
\item find out an expression for $Q_{n}$ in weak inversion;
\item evaluate $I_{SubVT}$ as diffusion current starting from $\frac{\partial Q_n (y)}{\partial y}$.
\end{enumerate}

A MOS system can be analyzed in terms of total charge density in the silicon below the gate oxide:

\begin{itemize}
\item writing the Poisson equation:
	
	\begin{eqnarray*}
	\frac{d^2  {\Psi(x)}}{dx^2} &=& -\frac{q \left[ p(x)-n(x)-N_A^- \right]}{\epsilon_S}\\[1ex]
	p(x)&=&N_A e^{\displaystyle\frac{- {\Psi(x)}}{kT}}\\[2ex]
	n(x)&=&\frac{n_i^2}{N_A}e^{\displaystyle\frac{ {\Psi(x)}}{kT}}
	\end{eqnarray*}

\item integrating it and then applying the Gauss’s theorem at the silicon silicon dioxide interface.
\end{itemize}

The total charge density can be expressed as:

    \begin{eqnarray*}
      Q_S&=&-\frac{\sqrt{2} \epsilon_S V_T}{L_D}\cdot \\
         & &\displaystyle\left\{e^{\frac{-\Psi(0)}{V_T}}+ {\frac{\Psi(0)}{V_T}}-1+
             \frac{n_i^2}{N_A^2}\left[\displaystyle {e^{ \frac{\Psi(0)}{V_T}}} -
             \frac{\Psi(0)}{V_T}-1\right]\right\}^{\frac{1}{2}}\\[3ex]
       L_D&=& \sqrt{\frac{\epsilon_S V_T}{qN_A}}\;\; \rm{Debye\;\;length}
    \end{eqnarray*}

Now we want to find an approximate expression for Q when the system is in weak inversion. Assuming that $\Phi_P = \frac{kT}{q}ln\left(\frac{N_A}{n_i}\right) \approx 300 mV$ and $V_{TH} \approx 26 mV$:

	\begin{figure}[H]
	\centering
	\includegraphics[scale=0.35]{07_graph3}
	\caption{}
	\label{tau}
	\end{figure}

\begin{itemize}
\item $\frac{\Phi(0)}{V_{TH}} \approx 12$ when $\Phi(0) = \Phi_P$;
\item$\frac{\Phi(0)}{V_{TH}} \approx 24$ when $\Phi(0) = 2 \Phi_P$.
\end{itemize}

This assumption has a consequence on the terms that we can neglet in the expression of $Q_{S}$:


	\begin{eqnarray*}
      Q_S&=&-\frac{\sqrt{2} \epsilon_S V_T}{L_D}\cdot \\
         & &\displaystyle\left\{{e^{\frac{-\Psi(0)}{V_T}}}+ {\frac{\Psi(0)}{V_T}}{-1}+
             \frac{n_i^2}{N_A^2}\left[\displaystyle {e^{ \frac{\Psi(0)}{V_T}}} -
             {\frac{\Psi(0)}{V_T}}{-1}\right]\right\}^{\frac{1}{2}}\\[3ex]      
    \end{eqnarray*}


\begin{itemize}
\item $e^{-\frac{\Phi(0)}{V_{TH}}} \ll 1$ when $\Phi(0) = 12\ \text{or}\ 24$ so it is negligible;
\item -1 negligible;
\item$-\frac{\Phi(0)}{V_{TH}} -1 \ll e^{\frac{\Phi(0)}{V_{TH}}}$ when $\Phi(0) = 12\ \text{or}\ 24$ so it is negligible.
\end{itemize}

The total charge density for different substrate doping levels can be related to the $q\Phi_P$ values:

	\begin{figure}[H]
	\centering
	\includegraphics[width=0.75\textwidth]{figsubvt1}
	\caption{The quantity of charge in function of $V_{S}$ for different substrate doping levels.}
	\label{}
	\end{figure}

Remembering that:

\begin{equation}
\Phi_P = \frac{kT}{q}ln\left(\frac{N_A}{n_i}\right) \Rightarrow \frac{N_A}{n_i} = e^{\frac{q \Phi_P}{V_{TH}}} \Rightarrow \frac{n_{i}^2}{N_{A}^2} = e^{\frac{-2q \Phi_P}{V_{TH}}}
\end{equation}

Substituting the above expression of $\frac{n_{i}^2}{N_{A}^2}$ in the previous equation of $Q_{S}$ and considering only the terms not negligible, the approximated expression of $Q_{S}$ when the system works in weak inversion ($\Phi_P < \Phi(0) < 2\Phi_P$) is:

   \begin{eqnarray*}
      Q_S&=&-\displaystyle\frac{\sqrt{2}\epsilon_S V_T}{L_D}
            \left\{\frac{\Psi(0)}{V_T} + 
          e^{\displaystyle \frac{\Psi(0)-2\Phi_P}{V_T}}\right\}^{\frac{1}{2}}
    \end{eqnarray*}

Now we replace the expression of the Debye length in $Q_{S}$ obtaining:

    \begin{eqnarray*}
      Q_S&=&-V_T \sqrt{\frac {2q \epsilon_S N_A}{V_T} \cdot \frac {\Psi(0)}{V_T}}\cdot \left\{ {\frac{V_T}{\Psi(0)} e^{\displaystyle \frac{\Psi(0)-2\Phi_P}{V_T}}} +1 \right\}^{\frac{1}{2}}
     \end{eqnarray*}

Since we are in weak inversion (under the threshold $V_{T}$), $\Psi(0) - 2\Phi_P < 0$ (because $\Phi_P < \Psi(0) < 2\Phi_P$); so $\frac{V_T}{\Psi(0)}e^{\frac{\Psi(0) - 2\Phi_P}{V_T}} < 1$ and we can apply the approximation $\sqrt{x+1} \approx 1+ \frac{x}{2}$ obtaining:

 \begin{eqnarray*}
      Q_S &=&-\sqrt{2q \epsilon_S N_A \Psi(0)}\cdot \left\{ 1 +\frac{1}{2}\left( \frac{V_T}{\Psi(0)} e^{\displaystyle\frac{\Psi(0)-2\Phi_P}{V_T}}\right) \right\}
  \end{eqnarray*}

Now, since $Q_S = Q_n + Q_d$, we can obtain $Q_{n}$ as $Q_n = Q_S - Q_d$, where:

\begin{itemize}
\item $Q_{S}$ is the \textbf{total charge in the semiconductor} (in this case we are considering a p-type substrate);
\item $Q_{n}$ is the \textbf{inversion layer charge};
\item $Q_{d}$ is the \textbf{depletion layer charge}.
\end{itemize}


	\begin{figure}[H]
	\centering
	\includegraphics[scale=0.45]{08_charge_regions}
	\caption{}
	\label{}
	\end{figure}





Observing the above figure we can say that:

\begin{itemize}
\item in the \textbf{accumulation region} the charge is only due to \textbf{holes};
\item in the \textbf{depletion region} the charge is due to \textbf{ions};
\item in the \textbf{weak inversion region} the charge is mainly due to \textbf{ions} plus a \textbf{small quantity of electrons};
\item in the \textbf{strong inversion region} the total charge is due to \textbf{electrons} plus a negligible part of ions.
\end{itemize}

So, in weak inversion, to get the inversion charge we have to subtract the depletion charge from the total charge, as we wrote before ($Q_n = Q_S - Q_d$). We know the expression for $Q_{S}$ but we need to evaluate the expression of the  depletion charge under the gate. $Q_{d}$ can be obtained evaluating the thickness of the depleted region: in weak inversion the channel is uniform and it presents a voltage drop that is exactly $\Psi(0)$. Having an ideal voltmeter, the voltage drop can be measured taking as reference the voltage of the substrate and measuring the surface potential that is the value that can be measured at the interface between silicon and oxide:

	\begin{figure}[H]
	\centering
	\includegraphics[scale=0.6]{08_channel_voltage_meas}
	\caption{}
	\label{}
	\end{figure}

The general expression for the potential across the depleted region is:

\begin{equation}
\Psi(0) = \frac{qN_A}{2 \epsilon_S} x_d^2
\label{depl_pot}
\end{equation}

From (\ref{depl_pot}) we can derive $x_{d}$:

\begin{equation}
x_d = \sqrt{\frac{2 \epsilon_S \Psi(0)}{qN_A}}
\label{xd}
\end{equation}

Considering a \textbf{uniform doped substrate}, we can write the expression of the depleted charge as:

\begin{equation}
Q_d = -q N_A x_d
\label{Qd}
\end{equation}

Substituting (\ref{xd}) in (\ref{Qd}) we get:

\begin{equation}
Q_d = -\sqrt{q N_A 2 \epsilon_S \Psi(0)}
\label{}
\end{equation}

Now, we can derive $Q_{n}$ as $Q_n = Q_S - Q_d$ (the expression of $Q_{S}$ is the one that we have derived after applying the expansion $\sqrt{x+1} \approx 1+ \frac{x}{2}$):

  \begin{eqnarray*}
      Q_n &=&-\sqrt{2q \epsilon_S N_A \Psi(0)} \cdot \left\{\frac{1}{2}\left( \frac{V_T}{\Psi(0)} e^{\displaystyle 			\frac{\Psi(0)-2\Phi_P}{V_T}}\right) \right\}
  \end{eqnarray*}

The chart below compares three curves associated to:

\begin{enumerate}
\item $Q_{S}$ not approximated;
\item $Q_{S}$ approximated (Qs\_a);
\item $Q_{S}$ approximated and expanded (Qs\_as);
\end{enumerate}

	\begin{figure}[H]
	\centering
	\includegraphics[width=0.70\textwidth]{figsubvt2}
	\caption{}
	\label{}
	\end{figure}

It can be seen that the approximation that we did on $Q_{S}$ is really good; infact, for $\Phi_P \approx 0.54 V$, the \% errors of the approximate and approximate-expanded functions with respect to the original one are around 1-2\%:

	\begin{figure}[H]
	\centering
	\includegraphics[width=0.70\textwidth]{figsubvt3}
	\caption{}
	\label{}
	\end{figure}

The expressions that we have derived since now are valid only if the \textbf{channel is uniform ($V_{DS} = 0$)}.

Now we have to consider that the \textbf{channel is not uniform (drain biased $\Rightarrow$ channel not uniform)}.\\
When we measure $I_{OFF}$ we are in this situation:

	\begin{figure}[H]
	\centering
	\includegraphics[scale=0.36]{08_Ioff_meas}
	\caption{The MOS is switched off while the drain is biased.}
	\label{}
	\end{figure}

We consider a bi-dimensional system (x, y): the origin is the interface between Si and oxide, y is the direction along the channel. We define a bi-dimensional potential $\Psi(x,y)$; when $x = 0 \rightarrow \Psi(0, y) = \Psi_S(y)$, where $\Psi_S(y)$ is the surface potential  function of the position (y) in the channel.\\
Using the quasi-Fermi level for the concentration of the electrons along the channel we can write:

  \begin{eqnarray*}
      Q_n(y)&=&-\frac{\sqrt{2q \epsilon_S N_A \Psi_S(y)}}{2} \frac{V_T}{\Psi_S(y)}
             \left(e^{\displaystyle \frac{\Psi_S(y)-2\Phi_P}{V_T}}
              \right)\\[2ex]
      Q_n(y)&=&-\sqrt{\frac{q \epsilon_S N_A}{2\Psi_S(y)}}~~ V_T
             \left(e^{\displaystyle \frac{\Psi_S(y)-2\Phi_P}{V_T}}
              \right)
     \end{eqnarray*}

We can see that the inversion charge along the channel decreases moving from source to drain. Moreover, the inversion charge, in weak inversion, depends on both $\Psi_S(y)$ in the squareroot and in the exponential, but the \emph{main dependency is the exponential one}. The term $1/\sqrt{\Psi_S(y)}$ can be approximated with $1/\sqrt{\Psi_S(0)}$ where, since we are interested to the \emph{system near the strong inversion (just below the threshold)}, we can approximate $\Psi_S(0) \approx 2\Phi_P$.
This is done in order to obtain a simpler expression of $Q_{n}(y)$.

	\begin{eqnarray*}
	      Q_n(y)&\approx &-\sqrt{\frac{q \epsilon_S N_A}{4\Phi_P}}~~V_T
	             \left(e^{\displaystyle \frac{\Psi_S(y)-2\Phi_P}{V_T}}
	              \right)\\[2ex]
	       C_{dep}&=& {\sqrt{\frac{q \epsilon_S N_A}{ 4\Phi_P}}}
	\end{eqnarray*}

Summirizing, the steps to obtain the approximate expression of $Q_{n}(y)$ are:
\begin{enumerate}
\item approximating $\Psi_S(y)$ with the value at the surface $\Psi_S(0)$;
\item approximating $\Psi_S(0)$ with the value near the threshold $2\Phi_P$.
\end{enumerate}

$C_{dep}$ is the \textbf{depletion capacitance} (actually, it is a density). Since it is a capacitance, it can be written as:

\begin{equation}
C_{dep} = \frac{\epsilon_S}{x_d}
\label{C_dep}
\end{equation}

Infact, in the depleted region of Si the capacitance can be evaluated by considering the depleted silicon as a dielectric material:

	\begin{figure}[H]
	\centering
	\includegraphics[scale=0.4]{08_C_dep}
	\caption{}
	\label{}
	\end{figure}

We recall that in weak inversion:

\begin{equation}
x_d = \sqrt{\frac{2 \epsilon_S \Psi(0)}{qN_A}},\ \Psi(0) \approx 2\Phi_P \Rightarrow x_d = \sqrt{\frac{2 \epsilon_S 2\Phi_P}{qN_A}}
\label{xd_2}
\end{equation}

Substituting (\ref{xd_2}) in (\ref{C_dep}) we get:

\begin{equation}
C_{dep} = \frac{\epsilon_S}{\sqrt{\frac{4 \epsilon_S \Phi_P}{qN_A}}}
\label{}
\end{equation}

This is the expression of the capacitance of the depleted region below the channel when the system is near the threshold; this expression is the same in the strong inversion case.

After defining $C_{dep}$ we can write the expression of $Q_{n}(y)$ as:

	\begin{eqnarray*}
	Q_n(y)&\approx& -C_{dep} V_T \left(e^{\displaystyle \frac{\Psi_S(y)-2\Phi_P}{V_T}}\right)
	\end{eqnarray*}

Assuming a linear variation of the $\Psi_S(y)$ between Source and Drain, we can be compare the curve associated with the original expression of $Q_{n}(y)$ with the curve associated to the approximated one:

	\begin{figure}[H]
	\centering
	\includegraphics[width=0.65\textwidth]{figsubvt4}
	\caption{}
	\label{}
	\end{figure}

It can be seen that the approximation gives very good results as the approximated curve (Qn\_a) has a trend which is really similar to the trend of the original curve (Qn). Moreover we can observe that the inversion charge, when the transistor is in the subthreshold region, decays exponentially along the channel. It is worth noticing that the graph above shows the trend of the inversion charge density and it assumes really small values (few electrons). So, the inversion charge in subthreshold condition is too small to produce significant drift current and the total current is mainly due to the diffusion current (that represent a leakage) which, instead, depends on the gradient.

We define:

	\begin{eqnarray*}
	Q_n(0)&\approx&-C_{dep} V_T
	             \left(e^{\displaystyle \frac{\Psi_S(0)-2\Phi_P}{V_T}}
	               \right)\\[2ex]
	Q_n(L)&\approx&-C_{dep} V_T
	             \left(e^{\displaystyle \frac{\Psi_S(0)- {V_{DS}}-2\Phi_P}{V_T}}
	              \right)
	\end{eqnarray*}

where $Q_{n}(0)$ is the quantity of the inversion charge at the source and $Q_{n}(L)$ is the concentration of the inversion charge at the drain, both in WI.

We can approximate the diffusion current in this way:

	 \begin{eqnarray*}
	  I_{ndiff}(y) &=& D_nW \frac{\partial  {qt_{ch} n(y)}}{\partial y} D_nW \frac{\partial Q_n(y)}{\partial y}
	 \end{eqnarray*}

The gradient can be approximated as:

	 \begin{eqnarray*}
	  \frac{\partial Q_n(y)}{\partial y} &\approx& - \frac{Q_n(L)-Q_n(0)}{L}
	 \end{eqnarray*}

Now we can write that:

	 \begin{eqnarray*}
	  I_{ndiff} &=& -I_{DS}\\[2ex]\\
	  I_{DS}    &=& \frac{W D_n C_{dep} V_T}{L}  
	            e^{\displaystyle \frac{ {\Psi_S(0)-2\Phi_P}}{V_T}}
	           \left( 1-e^{\displaystyle\frac{-V_{DS}}{V_T}}\right)\\[2ex]
	  I_{DS}    &=& \frac{W \mu_n C_{dep} {V_T}^2}{L}  
	            e^{\displaystyle \frac{ {\Psi_S(0)-2\Phi_P}}{V_T}}
	           \left( 1-e^{\displaystyle\frac{-V_{DS}}{V_T}}\right)\\[2ex]
	  \end{eqnarray*}

Regarding the last expression ($I_{DS}$), we can note that:

\begin{itemize}
\item when $V_{DS} = 0 \Rightarrow I_{DS} = 0$ (no voltage drop between source and drain means no current flux);
\item when $V_{DS} = V_T \Rightarrow I_{DS} = \text{const.}$
\end{itemize}

In the second case what happens is that, increasing $V_{DS}$, even if we are in the subthreshold region, leads to the channel pinch-off at the drain (the transistor is in saturation).

Moreover, it is important to notice the dependency of $I_{DS}$ on the channel length L (\textbf{channel length modulation}): reducing the dimensions of the transistor (in particular L) the current increases because the gradient of the charge along the channel is increased. This explains why scaling has the effect of increasing the subthreshold current.

Now, looking at the $I_{DS}$ equation that we have derived before, we can observe that there is an exponential dependency on $\Psi_S(0)$. However, the value of $\Psi_S(0)$ is unknown, it depends on the gate voltage. Thus, we have to \emph{put in relation $\Psi_S(0)$ with the external gate voltage}.

The band diagram of an MOS structure is the following:

	\begin{figure}[H]
	\centering
	\includegraphics[scale=0.5]{08_band_diag}
	\caption{The band diagram of an MOS structure.}
	\label{}
	\end{figure}

Observing it we can write:

\begin{equation}
V_G - V_{FB} = \Psi_S(0) + V_{OX} \Rightarrow V_G = V_{FB} + \Psi_S(0) + V_{OX}
\label{Vg}
\end{equation}

In weak inversion the charge distribution is:

	\begin{figure}[H]
	\centering
	\includegraphics[scale=0.4]{08_charge_distrib}
	\caption{The charge distribution in weak inversion.}
	\label{}
	\end{figure}

\begin{equation}
\rho(x) = -qN_A
\label{}
\end{equation}

The electric field is:

	\begin{figure}[H]
	\centering
	\includegraphics[scale=0.3]{08_electric_field}
	\caption{The electric field in weak inversion.}
	\label{}
	\end{figure}

\begin{equation}
E_S(0) = \frac{qN_Ax_d}{\epsilon_S}
\label{electric_field}
\end{equation}

$E_{S}(0)$ is the electric field at the interface between silicon and oxide.

The potential is:

	\begin{figure}[H]
	\centering
	\includegraphics[scale=0.3]{08_potential}
	\caption{The potential in weak inversion.}
	\label{}
	\end{figure}

\begin{equation}
\Phi_S(0) = \frac{qN_Ax_d^2}{\epsilon_S}
\label{}
\end{equation}

$\Phi_S(0)$ is the potential at the interface between silicon and oxide.

Substituting the expression of $x_{d}$ in \ref{electric_field} we get:

\begin{equation}
E_S(0) = \frac{qN_Ax_d}{\epsilon_S} = \frac{qN_A}{\epsilon_S} \sqrt{\frac{2\epsilon_S\Psi_S(0)}{qN_A}} = \sqrt{\frac{2 qN_A \Psi_S(0)}{\epsilon_S}}
\label{}
\end{equation}

Now we can evaluate the voltage drop across the oxide:

\begin{equation}
E_S(0) \epsilon_S = E_{OX} \epsilon_{OX} \Rightarrow E_{OX} = \frac{\epsilon_S}{\epsilon_{OX}}E_S(0) = \frac{1}{\epsilon_{OX}} \sqrt{2q \epsilon_S N_A \Psi_S(0)}
\label{}
\end{equation}

\begin{equation}
V_{OX} = t_{OX} E_{OX} = \frac{1}{C_{OX}} \sqrt{2q \epsilon_S N_A \Psi_S(0)}
\label{VOX}
\end{equation}

Now we have an expression of $V_{OX}$ in function of $\Psi_S(0)$.

Substituting (\ref{VOX}) in (\ref{Vg}) we get:

\begin{equation}
V_G = V_{FB} + \Psi_S(0) + \frac{1}{C_{OX}} \sqrt{2q \epsilon_S N_A \Psi_S(0)}
\label{}
\end{equation}

This expression is too complex and we want to approximate it performing an expansion of $V_{G}$ around $2\Phi_P$:

\begin{equation}
f(\Psi_S(0)) \approx f(2\Phi_P) + f'(2\Phi_P)(\Psi_S(0) - 2\Phi_P)
\label{}
\end{equation}

The expanded version of $V_{G}$ is:

	\begin{eqnarray*}
	   V_G &=&  \underbrace{V_{FG}+2\Phi_P+\frac{1}{C_{ox}}\sqrt{2q\epsilon_S N_A 2\Phi_P}}_{f(2\Phi_P)} +  \underbrace{\left( 1+ \frac{1}{C_{ox}}\sqrt{\frac{q\epsilon_SN_A}{4\Phi_P}}\right)}_{f'(2\Phi_P)} \left(\Psi_S(0)-2\Phi_p\right)
	\end{eqnarray*}

The first part of the equation (f(2$\Phi_P$)) is indeed the \textbf{nominal threshold voltage} of the MOSFET:

	\begin{eqnarray*}
	   V_{Th} &=& V_{FB}+2\Phi_P+\frac{1}{C_{ox}}\sqrt{2q\epsilon_S N_A 2\Phi_P}
	\end{eqnarray*}

So we can arrange the expression of $V_{G}$ as:

\begin{equation}
V_G = V_{th} + (1+ \dots) (\Psi_S(0) - 2\Phi_P)
\label{Vg_arr}
\end{equation}



From (\ref{Vg_arr}) we can derive:

          \begin{eqnarray*}
             \Psi_S(0)-2\Phi_P=\frac{V_G-V_{Th}}{1 +\frac{C_{dep}}{C_{ox}}}
          \end{eqnarray*}

This expression can be substituted in the expression of $I_{DS}$ obtaining:

    \begin{eqnarray*}
    I_{DS}    &=& \frac{W \mu_n C_{dep} {V_T}^2}{L}
                  \left( 1-e^{\displaystyle\frac{-V_{DS}}{V_T}}\right) 
            e^{\displaystyle \frac{ {V_G-V_{Th}}}{V_T\left(1 +\frac{C_{dep}}{C_{ox}}\right)}} 
     \end{eqnarray*}

When we are interested in evaluating the threshold voltage, we have that $V_{DS} \approx V_{DD}$. For this reason, setting $V_{DS} = V_{DD}$, the term $e^{\frac{-V_{DS}}{V_T}}$ can be negleted since it is $\ll 1$.

Defining \textbf{m}, we finally have the complete \textbf{expression for the subthreshold current}:

    \begin{eqnarray*}
       {m}&=& \left(1 +\frac{C_{dep}}{C_{ox}}\right)
     \end{eqnarray*}

   \begin{eqnarray*}
    I_{DS}    &\approx & \frac{\mu_n W \left(m-1 \right) C_{ox} {V_T}^2}{L}\;
            e^{\displaystyle \frac{ {V_G-V_{Th}}}{ mV_T}} 
     \end{eqnarray*}

This last equation can also be written as:

\begin{equation*}[box=\fbox]{align}
I_{DS} \approx \frac{\mu_n W C_{dep} V_T^2}{L} e^{\frac{V_G - V_T}{m V_T}}
\label{final_Ids}
\end{equation*}

The factor m is the \emph{most used parameter to optimize technological processes}.

We can comment the equation (\ref{final_Ids}) saying that when the transistor is in subthreshold region ($V_G < V_T$), the term $e^{\frac{V_G - V_T}{m V_T}}$ is negative; this means that the subthreshold current decays exponentially when the transistor is switched off.

The following chart shows the trend of $I_{DS}$.

	\begin{figure}[H]
	\centering
	\includegraphics[scale=0.45]{08_Ids_Vgs_graph}
	\caption{}
	\label{}
	\end{figure}

\begin{itemize}
\item In WI, as said before, $I_{DS}$ decays exponentially; since we are using a semilogarithmic chart, the current has a linear trend;
\item in SI,  $I_{DS}$ increases in a logarithmic way assuming, then, a constant value.
\end{itemize}


We recall that the final approximated expression of the subthreshold current is the following (for $V_{DS} \approx V_{DD}$):

   \begin{eqnarray*}
    I_{DS}    &\approx & \frac{\mu_n W \left(m-1 \right) C_{ox} {V_T}^2}{L}\;
            e^{\displaystyle \frac{ {V_G-V_{Th}}}{ mV_T}} 
     \end{eqnarray*}

Where `m' is:

    \begin{eqnarray*}
       {m}&=& \left(1 +\frac{C_{dep}}{C_{ox}}\right)
     \end{eqnarray*}

We remember that $C_{dep}$ and $C_{OX}$ are called depletion and oxide capacitance, respectively, but they are \emph{densities} measured in fF/cm$^{2}$.

The behaviour of the sub-threshold current for HP100 HP65 HP32 technologies is:

	\begin{figure}[H]
	\centering
	\includegraphics[width=0.70\textwidth]{figsubvt5}
	\caption{}
	\label{}
	\end{figure}

We can notice that:

\begin{itemize}
\item the subthreshold current decades exponentially when we move in the subthreshold region;
\item long channel (HP100) devices have a lower subthreshold current for the same value of $V_{GS} - V_{TH}$ with respect to small channel (HP32) devices.
\end{itemize}

If we compare the subthreshold current evaluated using the approximated expression and the one measured on a real device, we can assert that the theoretical model exploited to get the approximated equation is quite accurate. We can refer to this model as \textbf{Taur model}.

Generally, it is difficult to say if a process is good or not just analizing the $I_{OFF}$ current since this parameter is not enough and it can be misleading. Now we define another parameter that can be used to classify a process. It is called \textbf{SS (Subthreshold Swing/Slope)}:

	\begin{figure}[H]
	\centering
	\includegraphics[scale=0.5]{09_SS}
	\caption{}
	\label{}
	\end{figure}

For example, considering the $T_{1}$ curve, $SS_{1}$ can be defined as the variation of the gate voltage required to get a reduction of one order of magnitude of the leakage current (from 10I' to I').

Supposing that $T_{1}$ and $T_{2}$ are two different technological processes (with $I_{OFF,\ T2} < I_{OFF,\ T1}$), which one is better? If we considered only the subthreshold current we could say that $T_{2}$ is better than $T_{1}$. But, we can observe that $SS2 > SS1$, which means that, in the $T_{2}$ case, we have to apply a bigger $\Delta V_G$ to get a reduction of one order of magnitude of the leakage current.

So, the SS tells us how easily a transistor can be switched off. How we can evaluate SS? We can apply the definition of slope which is the inverse of the derivative of the $log_{10}I_{DS}$ curve:

	\begin{equation}
	SS = \left( \frac{d(log_{10}I_{SUBV_T})}{dV_G} \right)^{-1}
	\label{}
	\end{equation}
	
	\begin{equation}
	SS = \left \{ \frac{d}{dV_G} \left[ log_{10} \left( \frac{\mu_n W C_{dep} V_T^2}{L} e^{\frac{V_G - V_T}{m V_T}} \right) \right] \right \}^{-1}
	\label{}
	\end{equation}
	
	\begin{equation}
	SS = \left \{ \frac{d}{dV_G} \left[ log_{10} \left( \frac{\mu_n W C_{dep} V_T^2}{L} \right) + \frac{V_G - V_T}{m V_T} \cdot \frac{1}{ln10} \right] \right \}^{-1}
	\label{}
	\end{equation}
	
	\begin{equation}
	SS = \left( \frac{1}{m V_T} \cdot \frac{1}{ln10} \right)^{-1} = m V_T ln10
	\label{}
	\end{equation}
	
	\begin{equation}
	SS = 2,3 \cdot V_T \cdot m
	\label{}
	\end{equation}
	
	\begin{equation}
	SS|_{300K}  = 2,3 \cdot 26mV \cdot m \approx 59.8 mV \cdot m
	\label{}
	\end{equation}

SS is measured in \textbf{mV/dec}.

The lower is the value of SS the better is the technological process. \textbf{The minimum reachable value for SS is 59.8 mV/dec}, when $m = 1$. So, from the technological point of view, all the efforts are concentrated in the direction that allows to have $m \approx 1$. Typical SS values for the bulk CMOS technology are in the range 80 - 100 mV. More advanced technologies have a lower SS and in some cases it is possible to reach values near the theoretical limit.\\
Since $m = 1+ \frac{C_{dep}}{C_{OX}}$, its value can be reduced lowering the ratio $C_{dep}/C_{OX}$ and this can be done in two ways:

\begin{enumerate}
\item $C_{dep} \rightarrow 0$;
\item $C_{OX} \rightarrow \infty$.
\end{enumerate}

The best choice is the first one from the technological point of view.

In the graph below are shown different curves related to SS:



	\begin{figure}[H]
	\centering
	\includegraphics[width=0.65\textwidth]{figsubvt6}
	\caption{}
	\label{}
	\end{figure}

We can observe that long channel devices (HP100) have a lower subthreshold current but a higher SS while small channel devices (HP32) have a higher subthreshold current but a smaller SS. So, in this case scaling is working in the right direction since it allows to reduce SS.

\subsection{Drive current}

The simplest expression of the drive current is the following:

\begin{equation}
I_{DS,\ sat} = \frac{\mu_n C_{OX} W}{2L} (V_{GS} - V_{TH})^2
\label{}
\end{equation}

However, this model \emph{overestimates} the current. Why? Because we consider just a linear dependency of the inversion charge on the gate voltage neglecting other squareroots terms present in the charge expression. Clearly, this leads to an unprecise approximation.

In general, we can use three models to estimate the drive current:
\begin{enumerate}
\item the expression obtained from the linear approximation of $Q_{n}(y)$

	\begin{eqnarray*}
	 I_{DS} &=&\frac{W \mu_n C_{ox}}{L}\left[\left(V_G-V_{Tn}\right)V_{DS}-\frac{V_{DS}^2}{2}\right]
	\end{eqnarray*}

\item the expression obtained using the `m' factor

	\begin{eqnarray*}
	 I_{DS} &=&\frac{W \mu_n C_{ox}}{L}\left[\left(V_G-V_{Tn}\right)V_{DS}-\frac{ {m}}{2}V_{DS}^2\right]
	\end{eqnarray*}

\item the expression derived using the complete $Q_{n}(y)$ (spice model)

	\begin{eqnarray*}
	I_{DS} &=&\frac{W \mu_n C_{ox}}{L}\left[\left(V_G-V_{Tn}\right)V_{DS}-\frac{V_{DS}^2}{2}\right]
	        -\gamma_B \frac{W}{L}\mu_n C_{ox}\cdot\\
	        & &\cdot\left\{\frac{2}{3}\left[\sqrt{\left(2\Phi_P +V_{DS}\right)^3}
	            -\sqrt{\left(2\Phi_P\right)^3}\right]-\sqrt{2\Phi_P}\cdot V_{DS}\right\}
	\end{eqnarray*}

\end{enumerate}

We can observe that the second expression is equal to the first one except for \textbf{`m'} that acts as a \textbf{correction factor}.

Comparing the three equations it can be noted that the `m' factor expression represents a very good approximation and it has one big advantage: it is a very simple expression that can be easily used for calculations.


	\begin{figure}[H]
	\centering
	\includegraphics[width=0.55\textwidth]{figidson1}
	\caption{Comparison of the three different expression of $I_{DS}(V_{DS})$}
	\label{}
	\end{figure}

Regarding the `m' factor equation for the drive current we can say that, the smaller m is, the higher $I_{DS}$ is. Again, the ideal value of m is one. So, improving m ($m \approx 1$), we can get a lower subthreshold current and, at the same time, a higher drive current.

The \textbf{maximum drive current} (saturation current) can be obtained imposing that $V_{DSmax} = \frac{V_{GS} - V_{TH}}{m}$:

   \begin{eqnarray*}    
       I_{DSmax}&=& \frac{\mu_n W C_{ox}}{2L \cdot  {m}}\left(V_{GS}-V_{Tn}\right)^2
    \end{eqnarray*}

Again, this equation is equal equal to the one that can be obtained from $I_{DS}$ approximated except for `m' that acts as a correction factor.

The maximum speed of carriers in the channel is limited by \textbf{velocity saturation} and this has a strong effect on the maximum current that a trasistor can drive. The effect of velocity saturation due to the longitudinal field can be modeled using the following expression ($I_{DS}$ is expressed using `m'):

	\begin{eqnarray*}
	     \mu_n&=&\frac{\mu_{n0}}{1+\frac{\mu_{n0}V_{DS}}{Lv_{sat}}}\\[2ex]
	   I_{DS} &=&\frac{W \mu_{n0} C_{ox}Wv_{sat}}{Lv_{sat}+\mu_{n0}V_{DS}}
	             \left[\left(V_G-V_{Tn}\right)V_{DS}-\frac{ {m}}{2}V_{DS}^2\right]
	\end{eqnarray*}

(\emph{\textbf{Note}: it is not requested to know the expression of $I_{DS}$ but just to remember what velocity saturation is.})







\section{Scaling policies}

There are three scaling policies:

\begin{enumerate}
\item \textbf{Constant voltage scaling};
\item \textbf{Constant field scaling};
\item \textbf{Generalized scaling}.
\end{enumerate}

We will analize all these three policies.

Indepentently from the policy adopted, in \textbf{planar technology} scaling means reducing dimensions: transistors and interconnects are scaled by a different factor that for the former is \textbf{K} and for the latter is \textbf{S}.

The transistors dimensions are reduced by K:

\begin{align*}
&W \xrightarrow[]{scaling} W' & &L \xrightarrow[]{scaling} L' \\
&W' = W \cdot \frac{1}{K} & &L' = L \cdot \frac{1}{K}\\
&\frac{W}{W'} = K & &\frac{L}{L'} = K
\end{align*}

For interconnects we have (W is the width of the interconnect):

\begin{align*}
&W \xrightarrow[]{scaling} W'\\
&W' = W \cdot \frac{1}{S}\\
&\frac{W}{W'} = S
\end{align*}

Transistors scaling is more aggressive than interconnects scaling: $\text{\textbf{K}} > \text{\textbf{S}}$.

\subsection{Constant voltage scaling}

Constant voltage scaling features:

\begin{itemize}
\item All the dimensions are reduced by K;
\item VDD is kept constant.
\end{itemize}

For many years, during the golden age of the CMOS technology, $V_{DD}$ has \textbf{not changed} (5 V) even if the dimensions of the transistors were scaled. This choice has been done in order to \textbf{guarantee compatibility}: same logic signals. We know that a fundamental parameter useful to distinguish different technological processes is the \textbf{ratio} $\frac{V_{DD}}{V_{TH}}$; \textbf{in constant voltage scaling this ratio too has to be maintained constant}. This means that even $V_{TH}$ has to be kept constant.

But what are the consequences, from the technological point of view, of these choices?

Let's analyze the following table:

 \begin{table}[H]
 \begin{center}
 {
  {
     \begin{tabular}{||l||c|c||}\hline
      V$_{DD}$ & -& $ {1}$\\
      \hline
      L,W,$x_J$       &- & $ {1/K}$\\
      \hline
      N & $x_d=\sqrt{\frac{2\epsilon_S(\Phi_i+V_{DD})}{qN}}$&$ {K^2}$ \\
      \hline
      V$_{DD}/V_{Th}$ &- & ${1}$\\
      \hline
      V$_{Th}$ & $V_{FB}+2\Phi_P+\frac{1}{C_{ox}}\sqrt{2qN_A\epsilon_S 2\Phi_p}$& $ {1}$\\
      \hline
       $C_{ox}$&  $\epsilon_{ox}/Tox$ &$ {K}$\\ 
      \hline
      $T_{ox}$       &- & $ {1/K}$\\
      \hline
       $C_{dep}$&  $\sqrt{\frac{q\epsilon_SN_A}{4\Phi_P}}$ & ${K}$\\
      \hline
       $m$&  $1 +\frac{C_{dep}}{C_{ox}}$ &$ {1}$\\ 
      \hline
      ${\cal E}_y$ & $V_{DD}/L$ & $ {K}$\\
      \hline
      ${\cal E}_x$ & $V_{DD}/Tox$ & $ {K}$\\
      \hline
      $I_{DSsat}$      &$\frac{\mu_n Cox W}{2L}(V_{GS}-V_T)^2$ &$ {K}$\\ 
      \hline  
       $I_{ON}$      &- &$ {K}^2$\\ 
      \hline
       $I_{subVt}$      &$\frac{\mu_n W \left(m-1 \right) C_{ox} {V_T}^2}{L}\;
            e^{\displaystyle \frac{ {V_G-V_{Th}}}{ mV_T}}$ &$ {K}$\\ 
      \hline  
       $I_{OFF}$      &- &$ {K}^2$\\ 
      \hline
      $\tau$     &$Cox\cdot WL\cdot V_{DD}/Ion$ &$ {1/K^2}$\\ 
      \hline 
      $P_{Dyn}^{tr}$      &$fC_LV_{DD}^2$  &$ {K}$\\ 
      \hline 
      $P_{Dyn}^{Den}$      &-&$ {K^3}$\\ 
      \hline
       $P_{Isub}^{tr}$      &$I_{subVt}V_{DD}$  &$ {K}$\\ 
      \hline 
       $P_{Isub}^{Den}$      &-&$ {K^3}$\\ 
      \hline
     \end{tabular}
     }
    }
    \end{center}
    \end{table}

We can see that L and W are scaled as 1/K. $x_{J}$ is the \textbf{juction depth} of the source-drain junction near the channel and it is scaled as well of the same factor K:

	\begin{figure}[!h]
	\centering
	\includegraphics[scale=0.5]{09_junction_depth_scaling}
	\caption{Geometry dimensions scaling.}
	\label{}
	\end{figure}

Why $x_{J}$ needs to be scaled? At first glance we can say that if the thickness of the highly doped regions is smaller their resistance increase and this is a drawback. So, why, despite this disadvantage, $x_{J}$ is scaled? Scaling $x_{J}$ helps to control lateral diffusions. After source and drain are implanted, it is necessary to perform thermal annealing. During this phase lateral diffusions are created and this is a problem because they increase capacitances and create other problems. In order to reduce lateral diffusions it is necessary to implant less deeply the dopant. So, when we reduce planar dimensions (W, L) we have also to reduce the vertical dimension ($x_{J}$) otherwise it becomes impossible to control lateral diffusions. Moreover, scaling $x_{J}$ by the same factor of the channel helps to \textbf{keep under control the short channel effect (SCE) avoiding roll-off} (the threshold voltage is function of the channel length).

So, it is important to technologically control the doped regions, in terms of rediffusion, and the SCE (maintaining the dependece of $V_{TH}$ on L limited to avoid roll-off).

Now we want to define how and why the doping level (N) changes when we scale a device. 

In order to keep SCE limited, the depletion width have to be reduced.

\newpage

	\begin{figure}[H]
	\centering
	\includegraphics[scale=0.5]{09_depl_width}
	\caption{}
	\label{}
	\end{figure}

The depletion layer visible in the picture above is a good approximation when $V_{DS}$ is equal to 0 V. So, the depletion width at source and drain is the same. Moreover, there is the depletion layer of the channel and its width is known when the system is in SI or near SI:

	\begin{equation}
	x_{d \approx SI} = \sqrt{\frac{2 \epsilon_S 2 \Phi_P}{q N_A}}
	\label{}
	\end{equation}

The depletion width at the source junction (n\super{+} - p) can be estimated as:

	\begin{equation}
	x_{dSJ} = \sqrt{\frac{2 \epsilon_S \Phi_i}{q N_A}},\ \Phi_i \rightarrow \text{built-in voltage}
	\label{}
	\end{equation}

For an $n+p$ junction $2 \Phi_P \approx 0.7 V - 0.8 V$ and $2 \Phi_i \approx 0.7 V - 0.8 V$. In other words these two width can be confused and, as a consequence, $x_{d}$ and $x_{dSJ}$ can be confused as well. This is true only when $V_{DS} = 0$.

When $V_{DS} > 0$ what happens is that the depletion layer around the drain has a higher width than the depletion layer around the source.

	\begin{equation}
	x_{dDJ} = \sqrt{\frac{2 \epsilon_S (\Phi_i + V_{DD})}{q N_A}}
	\label{x_D}
	\end{equation}

As we said, in order to control and maintain limited the SCE it is necessary to scale the depletion width. But which depletion width? The worst depletion region is the one near the drain: $x_{dDJ} \gg x_{dSJ}$. This large depletion layer is completely controlled not by the gate voltage but by the drain voltage. The worst case is when  $V_{DS} = V_{DD}$. So we need to reduce $x_{dDJ}$: observing the equation (\ref{x_D}) we can say that $V_{DD}$ cannot be scaled (we are analyzing the constant voltage scaling), $\Phi_i$ depends on the logarithm of the concentration and its influence is not so strong in scaling, thus, the only way to reduce $x_{dDJ}$ is to increase $N_{A}$ like $K^{2}$.

	\begin{equation}
	x_{dDJ}' = \frac{x_{dDJ}}{K} \Rightarrow N_A' = K^2 \cdot N
	\label{x_D}
	\end{equation}

In constant voltage scaling, the doping level of the substrate is, node by node, increased in a non-linear way (strong variation of the doping level).

The increase of the doping level has consequences on the threshold voltage:

	\begin{equation}
	V_{Th} = V_{FB}+2\Phi_P+\frac{1}{C_{ox}}\sqrt{2qN_A\epsilon_S 2\Phi_p}
	\end{equation}

$V_{FB}$ and $\Phi_P$ depend both on $N_{A}$ but they change in an opposite direction because: 

\begin{itemize}
\item $\Phi_P = kTln\frac{N_A}{n_i}$;
\item the expression of $V_{FB}$ contains the working function of the silicon that depends on $-\Phi_P$.
\end{itemize}

so these two terms partially compensate. Moreover, the variation of $\Phi_P$ is not so strong since it depends on the logarithm of $N_{A}$ thus, under the sqaureroot, what prevails is $N_{A}$ that is scaled by $K^{2}$. This means that, overall, $V_{TH}$ should scale like K. But, since, as we said, we want to maintain the ratio $\frac{V_{DD}}{V_{TH}}$ constant, $V_{TH}$ cannot scale. As a result, $C_{OX}$ must increase of a factor K. In this way $V_{TH}$ remains constant.

	\begin{equation}
	V_{TH}' \approx V_{TH} \Rightarrow C_{OX}' = K \cdot C_{OX}
	\label{}
	\end{equation}

How can we increase the oxide capacitance? Reducing the thickness of the oxide ($C_{OX} = \epsilon_{OX}/T_{OX}$):

	\begin{equation}
	T_{OX}' = \frac{1}{K} \cdot T_{OX}
	\label{}
	\end{equation}

This is the set of rules to implement the constant voltage scaling. Now we have to analyze what are the consequences of applying such rules on the electrical parameters.

Let's start from the depletion capacitance:

	\begin{equation}
	C_{dep}' = \sqrt{\frac{q\epsilon_S N_A'}{4\Phi_P}},\ N_A' = K^2 \cdot N \Rightarrow C_{dep}' = K \cdot C_{dep}
	\label{}
	\end{equation}

The `m' factor doesn't change:

	\begin{equation}
	m' = 1+\frac{C_{dep}'}{C_{OX}'} = 1+ \frac{KC_{dep}}{KC_{OX}} \Rightarrow m' = m
	\label{}
	\end{equation}

The average electric field along the channel ($E_{y}$) and the vertical one ($E_{x}$) will increase:

	\begin{equation}
	E_y' = \frac{V_{DD}}{L'} = \frac{V_{DD}}{\frac{1}{K}L} \Rightarrow E_y' = K \cdot E_y
	\label{}
	\end{equation}

	\begin{equation}
	E_x' = \frac{V_{DD}}{T_{OX}'} = \frac{V_{DD}}{\frac{1}{K}T_{OX}} \Rightarrow E_x' = K \cdot E_x
	\label{}
	\end{equation}

Both these results are not good. Now we try to understand why.

First of all, the expression of the electric field $E_y = \frac{V_{DD}}{L}$ is an acceptable approximation only when $V_{DS} \approx 0$ because in this case the behaviour of the channel is resistive. Infact:

	\begin{figure}[H]
	\centering
	\includegraphics[scale=0.4]{09_Ids_char}
	\caption{Drain current in function of $V_{DS}$.}
	\label{}
	\end{figure}

For small values of $V_{DS}$ the behaviour of $I_{DS}$ is linear. This means that when the device is in the linear region the channel is resistive. This implies that the electric field along the channel is roughly uniform. So, the trend of the channel potential for small values of $V_{DS}$ is:

	\begin{figure}[H]
	\centering
	\includegraphics[scale=0.4]{09_channel_pot}
	\caption{}
	\label{}
	\end{figure}

Since $\Phi_{ch}$ is linear the electric field can be simply evaluated as the ratio between $V_{DD}$ and L.

This is not true for high values of $V_{DS}$ when the transistor is in saturation:

	\begin{figure}[H]
	\centering
	\includegraphics[scale=0.4]{09_channel_pot_2}
	\caption{For high values of $V_{DS}$ the depletion region around the drain is wider.}
	\label{}
	\end{figure}

The inversion charge ($Q_{n}$) is pinched-off (channel length modulation). The channel potential is approximately linear when $V_{DS} < V_{DSsat}$, while, when $V_{DS} > V_{DSsat}$, its trend strongly changes. So, the electric field assumes a lower value in the resistive part of the channel and a very high value in the region near the drain since the excess voltage with respect to the saturation voltage drops on a very short part of the channel. This causes the presence of \textbf{hot electrons}:

\newpage

	\begin{figure}[H]
	\centering
	\includegraphics[scale=0.45]{09_hot_electr}
	\caption{}
	\label{}
	\end{figure}

Electrons are travelling by mean of the electric field (drift current): near the source the el. field is not so high so it doesn't cause problems; electrons are confined between two high energy barriers, one towards the oxide (3 eV) and one electrostatic barrier due to the depletion region; inside the channel there are collisions with the interface and with the lattice; the energy that electrons receive from the electric field, in this region, is not enough high to make electrons jump the barriers in which they are confined. Near the drain electrons receive an overshoot and their speed is further increased. In addition, they receive such a high energy that they can overcome the barrier with the oxide and towards the substrate. Electrons become hot electrons and thay cause different problems:

\begin{itemize}
\item current loss since some electrons don't reach the drain (gate current);
\item trapped charges in the oxide (ions in the oxide) that cause a change in $V_{FB}$ and, as a consequence, $V_{TH}$.
\end{itemize}

If the threshold voltage shifts becoming too high or too low (it depends on the type of ions created in the oxide), we are no more able to either switch on or switch off the transistor.

How we can limit the hot electrons phenomenon?

\begin{itemize}
\item Reducing $V_{DD}$;
\item Reducing $T_{OX}$ (with a very thin oxide the charge capture is strongly limited); this helps improving robustness;
\item Changing the width of the depletion region near the drain.
\end{itemize}

Let's focus on the last point. Now we consider only the channel-drain junction. It is reverse biased since $V_{DD}$ is applied on $n+$. The electric field varies linearly in the p region and has an abrupt trend in the $n+$ region. The problem is represented by $E_{max}$.

	\begin{figure}[H]
	\centering
	\includegraphics[scale=0.4]{09_LDD_proc}
	\caption{The behaviour of the electric field with a standard process MOS and with a LDD process MOS.}
	\label{}
	\end{figure}

Reducing the doping level in the drain, the electric field varies linearly in both the regions. The built-in voltage must be kept constant, which means that the area of the triangular regions must be constant. The depletion region extends not only in the channel but also in the drain region. The result is that the maximum electric field is reduced. This solution is called \textbf{LDD (Lightly Doped Drain) technology}.

Starting from the middle of 80's all processes for CMOS technology became LDD: this means that the channel is not more out directly in contact with highly doped source/drain regions but it is necessary a buffer region characterized by a lower doping level.

 	\begin{figure}[H]
	\centering
	\includegraphics[scale=0.45]{09_LDD_device}
	\caption{A LDD device.}
	\label{}
	\end{figure}

The buffer regions are the \textbf{extension regions} that represent the true source and drain. Although the source region is not affected by the hot electrons problem, it has its extension region because it is not possible to distinguish the source contact from the drain one so the device is realized in a symmetrical way.

Now, the vertical electric field ($E_{x}$) represents a problem as well.

\begin{equation}
E_{x,\ MAX} = \frac{V_{DD}}{T_{OX}}
\end{equation}

$E_{x, MAX}$ is the maximum electric field that can be applied at the oxide and it is called \textbf{dielectric rigidity}. The electric field cannot overcome this value otherwise the oxide would be permanently damaged. The rigidity of $SiO_{2}$ is equal to 8 MV/cm. This value seems to be high but, actually, it is not because when we apply 5 V on few nanometers (oxide thickness) it is very easy to reach this value. This a problem because, on the basis of what we said before, the oxide thickness has to be scaled by a factor K and, since $V_{DD}$ isn't scaled, it is very easy to overcome the $SiO_{2}$ rigidity. \\


Now we analyze the effect of constant voltage scaling on electrical parameters such as $I_{ON}$, $I_{OFF}$ end power density.

We can estimate how the \textbf{saturation current} for a long channel transistor varies applying CV scaling rules:

\begin{equation}
I_{DS,\ sat}' = \frac{\mu_n K C_{OX} W/K}{2 L/K}(V_{GS} - V_{TH})^2 = K \cdot I_{DS,\ sat}
\end{equation}

This equation tells us that we have a higher saturation current using a smaller transistor.

Example:

	\begin{figure}[H]
	\centering
	\includegraphics[scale=0.5]{10_CVscaling_example}
	\caption{}
	\label{}
	\end{figure}

Smaller transistor means \textbf{smaller parasitics} and, thus, \textbf{faster technology} in addition to a higher saturation current. For these reasons, for many years, CV scaling has been strongly exploited: it guaratees higher performances.

Now let's see how  $I_{ON}$ changes rembering that this parameter is measured considering \emph{W fixed} ($W_{ref}$ that, according to the Roadmap is $1 \mu m$).

	\begin{equation}
	I_{ON}' = \frac{\mu_n K C_{OX} W_{ref}}{2 L/K}(V_{GS} - V_{TH})^2 = K^2 \cdot I_{ON}
	\end{equation}

After this we can find out how $I_{subV_{TH}}$ varies:

	\begin{equation}
	I_{subV_{TH}}' = \frac{\mu_n K C_{dep} W/K}{L/K} V_{T}^2 \left( e^{\frac{V_{GS} - V_{TH}}{mV_T}} \right) = K \cdot I_{subV_{TH}}
	\end{equation}

This is not a good result: the subthreshold current is subjected to an increase.

The same can be done for $I_{OFF}$ (this current, as $I_{ON}$, is evaluated considering $W_{ref}$):

	\begin{equation}
	I_{OFF}' = K^2 \cdot I_{OFF}
	\end{equation}

Now we can evaluate the dynamic power $P_{DYN}$ for one transistor. We recall that

	\begin{equation}
	P_{DYN \ Tr} =f C_L V_{DD}^2
	\end{equation}

We need an estimation of the frequency f starting from the \textbf{intrinsic time $\tau$}. According to the Roadmap, $\tau$ can be evaluated in this way:

	\begin{figure}[H]
	\centering
	\includegraphics[scale=0.5]{10_tau_meas}
	\caption{}
	\label{}
	\end{figure}

$C_{L}$ has the same value of the input capacitance of a transistor with the same characteristics of the one in the figure:

	\begin{equation}
	C_{L} = C_{OX} W_{ref} L
	\label{CL}
	\end{equation}

T1 is discharging the capacitance $C_{L}$ with its maximum current $I_{ON}$. $\tau$ can be written as:

	\begin{equation}
	\tau = \frac{C_L V_{DD}}{I_{ON}}
	\label{tau}
	\end{equation}

After scaling $\tau$ will be (substituting (\ref{CL}) in (\ref{tau})):

	\begin{equation}
	\tau ' = \frac{K C_{OX} W_{ref} L/K V_{DD}}{K^2 I_{ON}} = \frac{1}{K^2} \cdot \tau
	\end{equation}

This represents a very good result: devices are K\super{2} times faster thanks to scaling.

The frequency can be expressed as $f = 1/\tau$. So:

	\begin{equation}
	f' = K^2 f
	\end{equation}

f' is higher compared to unscaled technology.

Now we can use $\tau'$ and f' to understand how the dynamic power changes. It is important to underline that we want to evaluate the dynamic power dissipated by a \emph{real transistor} characterized by a width equal to W, not $W_{ref}$, so, the expression of $C_{L}$, in this case, becomes $C_L = C_{OX} W L$ and after scaling

	\begin{equation}
	C_L ' = K C_{OX} W/K L/K = C_L/K
	\end{equation}

The \textbf{dynamic power} after scaling is:

	\begin{equation}
	P_{DYN \ Tr}' =  K^2 f C_L/K V_{DD}^2 = K \cdot P_{DYN \ Tr}
	\end{equation}

This a very bad result.

Let's consider a scaling factor $K = 2$:

	\begin{figure}[H]
	\centering
	\includegraphics[scale=0.35]{10_power_density}
	\caption{}
	\label{}
	\end{figure}

After scaling we have 4 transistors that occupy the same area occupied by the unscaled device: so, same area but K\super{2} scaled devices each of which dissipate a power equal to $K \cdot P_{DYN \ Tr}$. If we evaluate the \textbf{power density per unit of area} what we get is:

	\begin{equation}
	P^{'den}_{DYN} = K^2 \cdot KP_{DYN}^{den} = K^3 \cdot P^{den}_{DYN}
	\end{equation}

This is the worst result that we have obtained until now; it has a strong influence on the circuit reliability since it is difficult to dissipate tens of W/cm\super{2}.

Regarding the \textbf{static power}, it can be expressed as $P_{STAT \ Tr} = I_{subV_{TH}} V_{DD}$.

	\begin{equation}
	P_{STAT \ Tr}' = K I_{subV_{TH}} V_{DD} = K \cdot P_{STAT \ Tr}
	\end{equation}

The \textbf{power density per unit of area} is:

	\begin{equation}
	P^{'den}_{STAT} = K^3 \cdot P^{den}_{STAT}
	\end{equation}

Both the static power per transistor and the static power density per unit of area increase like K and K\super{3} respectively. This is not a true problem since the level of leakages is low enough to guarantee an acceptable power dissipation.

Summarizing:

\begin{itemize}
\item \textbf{CV scaling pro}:
	
	\begin{itemize}
	\item faster devices.
	\end{itemize}

\item \textbf{CV scaling cons}:
	
	\begin{itemize}
	\item low reliability;
	\item unacceptable power consumption.
	\end{itemize}

\end{itemize}

Constant voltage scaling curves:

	\begin{figure}[H]
	\centering
	\includegraphics[width=1.0\textwidth]{figscaVDD}
	\caption{}
	\label{}
	\end{figure}

Graph on the left: the yellow line is representative of the CMOS technology golden age: for a very long period (from the 80s to the middle of the 90s)$ V_{DD}$ was 5 V.\\
Graph on the right: from the 80s to the middle of the 90s transistors were scaled but the supply voltage was kept constant; as a result, the average electric field in the channel increased like 1/year following the yellow curve trend.

The 5 V power supply was substituted by a 3.3 $V_{DD}$. Then 2.5 V and so on. 

It is fundamental to observe that the time period in which a power supply value is kept constant has become increasingly tight. Moving towards the most recent years $V_{DD}$ has been scaled year by year and this is true for HP, LOP and LSTP processes. Dimensions have been scaled as well so the electric field step becomes smaller. This means that we are moving towards a constant field scaling.

\newpage

\subsection{Constant field scaling}

This policy involves the reduction of the transistor dimensions in order to maintain constant the electric field. As a conseuqence, if dimensions are scaled by a factor K, $V_{DD}$ also has to be reduced like K.

  \begin{table}[H]
  {
  \begin{center}
   {
     \begin{tabular}{||l||c|c||}\hline
      V$_{DD}$ & -& $ {1/K}$\\
      \hline
      L,W,$x_J$       &- & $ {1/K}$\\
      \hline
      N & $x_d=\sqrt{\frac{2\epsilon_S(\Phi_i+V_{DD})}{qN}}$&$ {K}$ \\
      \hline
      V$_{DD}/V_{Th}$ &- & ${1}$\\
      \hline
      V$_{Th}$ & $V_{FB}+2\Phi_P+\frac{1}{C_{ox}}\sqrt{2qN_A\epsilon_S 2\Phi_p}$  & $ {1/K}$\\
      \hline
       $C_{ox}$&  $\epsilon_{ox}/T_{ox}$ &$ {K}^{+3/2}\approx K $\\ 
      \hline
      $T_{ox}$       &- & $ {1/{K}^{3/2}}\approx 1/K$\\
      \hline
       $C_{dep}$&   $\sqrt{\frac{q\epsilon_SN_A}{4\Phi_P}}$ &$K^{1/2}$\\
      \hline
       $ m-1 $&  $ \frac{C_{dep}}{C_{ox}}$ &$ {1/K^{1/2}} \approx 1 $\\ 
      \hline
      ${\cal E}_y$ & $V_{DD}/L$ & $ {1}$\\
      \hline
      ${\cal E}_x$ & $V_{DD}/Tox$ & $ {\approx 1}$\\
      \hline
       $I_{DSsat}$      &$\frac{\mu_n Cox W}{2L}(V_{GS}-V_T)^2$ &$ {1/K}$\\ 
      \hline  
       $I_{ON}$      &- &$ {1}$\\ 
      \hline
       $I_{subVt}$      &$\frac{\mu_n W \left(m-1 \right) C_{ox} {V_T}^2}{L}\;
            e^{\displaystyle \frac{ {V_G-V_{Th}}}{ mV_T}}$ &$ K^{1/2} e^{\frac{V_{Th}}{mV_T}
               \left( \frac{K-1}{K}\right)}$\\ 
      \hline  
       $I_{OFF}$      &- & $ K^{3/2} e^{\frac{V_{Th}}{mV_T}
               \left( \frac{K-1}{K}\right)} $\\ 
      \hline
       $\tau$     &$Cox\cdot WL\cdot V_{DD}/Ion$ &$ {1/K}$\\ 
      \hline 
       $P_{Dyn}^{tr}$      &$fC_LV_{DD}^2$  &$ {1/K^{2}}$\\ 
      \hline 
       $P_{Dyn}^{Den}$      &-&$ {1}$\\ 
      \hline
       $P_{Isub}^{tr}$     &$I_{subVt}V_{DD}$  &$ K^{-1/2} e^{\frac{V_{Th}}{mV_T}
               \left( \frac{K-1}{K}\right)} $\\ 
      \hline 
       $P_{Isub}^{Den}$      &-&$ K^{3/2} e^{\frac{V_{Th}}{mV_T}
               \left( \frac{K-1}{K}\right)} $\\ 
      \hline
     \end{tabular}
  }
    \end{center}
}
    \end{table}

We can notice that $E_{y}$ and $E_{x}$ are kept constant solving the problem of the presence of high energy electrons.

The next step is try to understand the consequences of the constant field scaling. Again, we want to maintain the ratio $V_{DD}/V_{TH}$ unscaled because, if, for instance, the starting process is the LOP one, after scaling the process must be same and in order to do this we have to keep constant the ratio. As a result $V_{TH}$ should be scaled like 1/K and this has a main downside: higher leakage current.

\newpage

\underline{Example}

	\begin{itemize}
	\item Before scaling: let's consider a technology with $V_{TH} = 200 mV$; in ideal conditions when $V_{GS} = 0$ the transistor is off; $V_{GS} - V_{TH} = -200 mV$;
	\item after scaling ($K = 2$): $V'_{TH} = V_{TH}/K = 100 mV$; $V'_{GS} - V'_{TH} = -100 mV$.
	\end{itemize}

	\begin{figure}[H]
	\centering
	\includegraphics[scale=0.45]{10_higher_leak_current}
	\caption{}
	\label{}
	\end{figure}

In the unscaled situation the subthreshold current is $I_{subVT}$ while in the scaled case the leakage current is $I'_{subVT}$ which is quite higher than $I_{subVT}$. In the last case it becomes difficult to switch off the transistor. This is the problem of the CF scaling.

Now we can analize how the technological parameters change applying the CF scaling.

\textbf{Doping level}: we know that, in order to limit the SCE, the drain depletion width has to be reduced.

	\begin{equation}
	x_d=\sqrt{\frac{2\epsilon_S(\Phi_i+V_{DD})}{qN}}
	\end{equation}

Since $V_{DD}$ is scaled by a factor K, it is sufficient to increase N by the same factor (this is an approximation) to reduce $x_{d}$ of a quantity equal to K. So:

	\begin{equation}
	N' = K \cdot N
	\end{equation}

Differently from CV scaling, in this case is sufficient a smaller increase of the substrate doping level.

\textbf{Threshold voltage}: we want to maintain $V_{DD}/V_{TH}$ constant; since $V_{DD}$ is scaled like 1/K it is necessary to scale $V_{TH}$ like 1/K as well. For this purpose, considering that $N_{A}$ increases like K, $C_{OX}$ has to be increased like $K^{3/2}$.

	\begin{equation}
	V'_{TH} = V_{FB} + 2\Phi_P + \frac{1}{K^{3/2} C_{OX}} \sqrt{2q \epsilon_S K N_A 2\Phi_P}
	\end{equation}
	
	\begin{equation}
	\frac{V'_{TH}}{V_{TH}} \approx \frac{K^{1/2}} {K^{3/2}} = \frac{1}{K}
	\end{equation}

As we said the \textbf{gate oxide capacitance} becomes:

	\begin{equation}
	C'_{OX} = K^{3/2} \cdot C_{OX}  \approx K \cdot C_{OX}
	\label{cox}
	\end{equation}

As a consequence we get the \textbf{oxide thickness}:

	\begin{equation}
	C'_{OX} = \frac{\epsilon_{OX}}{T'_{OX}} \Rightarrow T'_{OX} = \frac{\epsilon_{OX}}{K^{3/2} \cdot C_{OX}} \approx \frac{1}{K} \cdot T_{OX}
	\label{tox}
	\end{equation}

In the last two equations ((\ref{cox}) and (\ref{tox})) $K^{3/2}$ has been approximated as K to reduce the complexity of the expressions, on one hand; on the other hand, the oxide thickness is already very thin: scaling it like 1/K is a hard task, scaling it like $1/K^{3/2}$ is even harder and, in many cases, it is impossible. So, it is preferable to reduce $T_{OX}$ of a factor K and this, clearly, has a consequence: $V_{TH}$ will not scale exactly as expected but this is not a problem.

\textbf{Depletion capacitance}:

	\begin{equation}
	C'_{dep} = \sqrt{\frac{q \epsilon_S K N_A}{4 \Phi_P}} = K^{1/2} \cdot C_{dep} 
	\label{}
	\end{equation}

\textbf{Ratio $\frac{C'_{dep}}{C'_{OX} }$ (m - 1)}:

	\begin{equation}
	\frac{C'_{dep}}{C'_{OX}} = \frac{1}{K^{1/2}} \approx 1
	\label{}
	\end{equation}

Now we analyze the electrical parameters.

\textbf{Saturation current} (evaluated for $V_{GS} = V_{DD}$, maximum voltage applicable to a transistor):

	\begin{equation}
	I_{DS,\ sat}' = \frac{\mu_n K C_{OX} W/K}{2 L/K}\frac{(V_{DD} - V_{TH})^2}{K^2} = \frac{1}{K} \cdot I_{DS,\ sat}
	\label{}
	\end{equation}

The saturation current is reduced so, the effect on the reliability of the device is improved.

$I_{ON}$ current:

	\begin{equation}
	I_{ON}' = \frac{\mu_n K C_{OX} W_{ref}}{2 L/K}\frac{(V_{DD} - V_{TH})^2}{K^2} = 1 \cdot I_{ON}
	\end{equation}

\textbf{Instrinsic time}:

	\begin{equation}
	\tau ' = \frac{K C_{OX} W_{ref} L/K V_{DD}/K}{I'_{ON}} = \frac{1}{K} \cdot \tau
	\end{equation}

With respect to the CV scaling, now the performance are less good (devices are just K times faster, not K\super{2}).

\textbf{Dynamic power per transistor}:

	\begin{equation}
	P'_{DYN \ Tr} =f' C'_L V_{DD}^{'2}
	\end{equation}

where:

	\begin{align}
	&f' = \frac{1}{\tau'} = K \cdot f \\
	&C_L ' = K C_{OX} W/K L/K = C_L/K
	\end{align}

So $P'_{DYN} \ Tr$ becomes:

	\begin{equation}
	P_{DYN \ Tr}' =  K f C_L/K \frac{V_{DD}^2}{K^2} = \frac{P_{DYN \ Tr}}{K^2}
	\end{equation}

\textbf{Dynamic power density per unit area}:

	\begin{equation}
	P^{'den}_{DYN} = K^2 \cdot \frac{P^{den}_{DYN}}{K^2} = 1 \cdot P^{den}_{DYN}
	\end{equation}

These last two expressions represent an excellent result: the dynamic power consumption per transistor is reduced by $1/K^{2}$ (while in the CV scaling it increases like K) and the dynamic power density remains constant (while in the CV scaling it increases like $K^{3}$).



The subthreshold current is evaluated with $V_{GS} = 0$:

	\begin{eqnarray*}
	I_{subVt} & = &\frac{\mu_n W C_{dep} {V_T}^2}{L}\;e^{\displaystyle \frac{ {V_G-V_{Th}}}{ mV_T}}
	\end{eqnarray*}

After scaling, approximating m' with m, it becomes:

	\begin{eqnarray*}
	I_{subVt}^{\prime} & = &\frac{K \mu_n W K^{1/2} C_{dep} {V_T}^2}{K L}\;e^{\displaystyle \frac{ {V_G-(V_{Th}/K)}}{ m^{\prime}V_T}}
	\end{eqnarray*}

The ratio between $I'_{subVT}$ and $I_{subVT}$ is:

	\begin{eqnarray*}
	\frac{I_{subVt}^{\prime}}{I_{subVt}} & =& K^{1/2} e^{\frac{V_{Th}}{mV_T}\left( \frac{K-1}{K}\right)}
	\end{eqnarray*}

The ratio increases exponentially with K:

	\begin{figure}[H]
	\centering
	\includegraphics[scale=0.5]{11_I'sub_Isub_graph}
	\caption{}
	\label{}
	\end{figure}

Summarizing:

\begin{itemize}
\item Constant field scaling pros:
	\begin{itemize}
	\item limited electric field;
	\item limited dynamic power.
	\end{itemize}
\item Constant field scaling cons:
	\begin{itemize}
	\item high subthreshold leakages (they increase exponentially with scaling).
	\end{itemize}
\end{itemize}

It is not possible to scale $V_{DD}$ of the same scaling factor that we use for dimensions at each technological node since, doing in such a way, the power supply now should be equal to few millivolts and this, clearly, is not possible. In other words, we cannot apply only the constant field scaling because we would need a very small power supply and, as a consequence, a very low threshold voltage and it wouldn't be possible to operate transistors for two reasons: high leakages and noise (few mV of noise signal could erroneously switch on/off a transistor). On the other hand we cannot apply only constant voltage field because of the low reliability. We have to mix these two policies creating the \textbf{Generalized scaling}.

\subsection{Generalized scaling}

In generalized scaling are used two different parameters:

	\begin{enumerate}
	\item K for dimensions (W, L, $x_{j}$);
	\item $\alpha \rightarrow 1< \alpha < K$.  
	\end{enumerate}

How do we scale $V_{DD}$? As we said, scaling $V_{DD}$ by a factor K is too much, so, in generalized scaling, we correct it using the factor $\alpha$:

	\begin{equation}
	V'_{DD} = V_{DD} \cdot \frac{\alpha}{K}
	\end{equation}

$\alpha$ can be used as a parameter to measure the type of scaling:

	\begin{figure}[H]
	\centering
	\includegraphics[scale=0.4]{11_alpha}
	\caption{}
	\label{}
	\end{figure}

Varying $\alpha$ we modify the kind of scaling.

In the table below is summarized how technological and electrical parameters change applying the generalized scaling:

\newpage

\begin{table}[H]
{
   \begin{center}
  {
     \begin{tabular}{||l||c|c||}\hline
      V$_{DD}$ & -& $ {\alpha/K}$\\
      \hline
      L,W,$x_J$       &- & $ {1/K}$\\
      \hline
      N & $x_d=\sqrt{\frac{2\epsilon_S(\Phi_i-V_{DD})}{qN}}$&$ {\alpha K}$ \\
      \hline
      V$_{DD}/V_{Th}$ &- & ${1}$\\
      \hline
      V$_{Th}$ & $V_{FB}+2\Phi_P+\frac{1}{C_{ox}}\sqrt{2qN_A\epsilon_S 2\Phi_p}$  & $ {\alpha/K}$\\
      \hline
       $C_{ox}$&  $\epsilon_{ox}/T_{ox}$ &$ {K}^{+3/2}/\alpha^{1/2}\approx K $\\ 
      \hline
      $T_{ox}$       &- & $ {\alpha^{1/2}/{K}^{3/2}}\approx 1/K$\\
      \hline
       $C_{dep}$&  $\sqrt{\frac{q\epsilon_SN_A}{4\Phi_P}}$ &$ {(\alpha K)}^{1/2}$\\ 
      \hline
      ${\cal E}_y$ & $V_{DD}/L$ & $ {\alpha}$\\
      \hline
      ${\cal E}_x$ & $V_{DD}/Tox$ & $ {\approx \alpha}$\\
      \hline
     \end{tabular}
}
    \end{center}
}
\end{table}

Now should be clear how to deduce all these rules.

\newpage

\section{Basic scalable MOSFET}

The most common MOSFET structure is the LDD (Lightly-Doped Drain) one:

	\begin{figure}[H]
	\centering
	\includegraphics[width=0.75\textwidth]{figbasmos}
	\caption{A LDD MOSFET.}
	\label{}
	\end{figure}

Fundamental parts that characterize this structure are:

\begin{itemize}
\item the gate $\rightarrow$ $T_{gate}$, $T_{OX}$;
\item source/drain extensions $\rightarrow $ $X_{j}$ (junction depth);
\item source/drain contacts $\rightarrow$ $X_{jc}$ (contact junction depth);
\item spacers (regions beside the gate) $\rightarrow$ $T_{SW}$.
\end{itemize}

In the basic CMOS process, the polysilicon gate is used to mask source/drain implantation. Now we want to realize extensions and contacts exploiting the \textbf{LDD CMOS process}:

	\begin{figure}[H]
	\centering
	\includegraphics[width=0.75\textwidth]{figldd}
	\caption{The steps to realize spacers.}
	\label{LDD_process}
	\end{figure}

In the standard process source and drain are implanted using a high dose, $N_D = 10^{15}$, and a high energy, $E = 80/100\ keV$. In the LDD process, the dose is reduced by three orders of magnitude and the energy is around 20/30 keV in order to realize very shallow junctions ($X_j \approx 10 nm$). So, looking at figure \ref{LDD_process}, we can distinguish four steps:

\begin{enumerate}
\item [\textbf{Step (1)}] implatation of shallow light-doped source/drain extensions ($N_D = 10^{12}$, $E = 20/30\ keV$);
\item [\textbf{Step (2)}] CVD deposition of a thick ($\approx 1\mu m$) layer of silicon oxide;
\item [\textbf{Step (3)}] anisotropic etching to remove $SiO_{2}$; part of the silicon oxide is not removed and it constitutes spacers;
\item [\textbf{Step (4)}] high energy implantation of source/drain contacts ($N_D = 10^{15}$, $E = 80/100\ keV$).
\end{enumerate}

Spacers can be realized in $SiO_{2}$ or in silicon nitride ($Si_{3}N_{4}$) (more often in $SiO_{2}$). These structures are used to define source and drain contacts.

When we realize S/D extensions it is necessary to perform thermal annealing for a short time since we want shallow junctions. On the contrary, to realize S/D contacts, that have to be deep, it is required to perform thermal annealing for a longer time and at a higher temperature than the previous case. The consequence of performing "heavy" thermal annealing is that it causes the lateral diffusion of extensions below the gate, affecting the effective length of the channel, and it increases parasitics (because of the source-gate and drain-gate coupling). There exist techniques to reduce the lateral diffusion: the main idea is to implant S/D contacts before extensions.

The \emph{great advantage} of the LDD technology is it can be \textbf{implemented without any additional mask and photolitography} because the oxide deposition and source/drain contacts implantation are obtained using the masking action of spacers. This means that all processes can be easily updated to LDD. Starting from the end of 80s, the reference device has become the LDD one. 

An example of the reference MOSFET is the following:

\newpage

	\begin{figure}[H]
	\centering
	\includegraphics[width=0.85\textwidth]{figscalablemos}
	\caption{}
	\label{}
	\end{figure}

We can see that this device has very large spacers and, as a consequence, very long extensions: this is negative since the resistance of the extension regions becomes very high because they are lightly-doped and long.

Moreover, we can notice the presence of silicide (the silicide is a compound of silicon and metal, in this case nichel) contacts useful to reduce the parasitic resistance.

We recall that source/drain extensions have been introduced in order to improve the reliability of the device limiting the high electric field near the drain. Extensions offer another advantage: \textbf{better scaling} because contact 
geometry does not depend on S/D extension dimensions. This means that we can scale independently $X_{j}$ and $X_{jc}$ so, we can have a thin junction depth (generally scaled by a factor K) and a thick contact region (less scaled than the junction depth) to reduce the resistance.

Why the junction depth $X_{j}$ has to be scaled by the same factor with which we scale the device dimensions? Because of the \textbf{short channel effects}.

\section{Short Channel Effects}
\subsection{Roll-off}


For a long channel device the threshold voltage can be written as:

	\begin{equation}
	V_{TH,\ LC} = V_{FB} + 2\Phi_P + \frac{1}{C_{OX}} \sqrt{2q \epsilon_s N_A 2 \Phi_P} = V_{FB} + 2\Phi_P - \frac{Q_d}{C_{OX}}
	\end{equation}
	$[Q_{d}] = C/cm^2$ ($Q_{d}$ is a charge density)\\
	$[C_{OX}] = F/cm^2$

We can rewrite the ratio between the depletion charge and the oxide capacitance multiplying and dividing it by $W \cdot L$; in this way we are considering the effective charge and capacitance, not their densities:

	\begin{equation}
	V_{TH,\ LC} = V_{FB} + 2\Phi_P - \frac{Q_d \cdot W \cdot L}{C_{OX} \cdot W \cdot L}
	\end{equation}

In a long channel device the quantity $W \cdot L$ is the same at the numerator and at the denominator, so it can be simplified; in addition, considering that $Q_d = -q N_A x_d$, we can simply write:

	\begin{equation}
	V_{TH,\ LC} = V_{FB} + 2\Phi_P + \frac{q N_A x_d}{C_{OX}}
	\end{equation}

But, in a short channel device these semplifications are no more valid:

	\begin{figure}[H]
	\centering
	\includegraphics[width=0.75\textwidth]{figsce}
	\caption{A short channel LDD MOS}
	\label{figsce}
	\end{figure}

The oxide capacitance does not change: $C_{OX} \cdot W \cdot L$. What varies is the depletion capacitance because, now, the geometry of the depleted region under the channel is changed.

The total depletion charge becomes:

\begin{equation}
Q_d = -q N_A x_d W \left( \frac{L + L - 2x_1}{2} \right)
\end{equation}

We want to evaluate $x_{1}$ making two assumptions:

	\begin{itemize}
	\item S/D junctions have a circular profile;
	\item the depletion region beside S/D junctions has the same width of the depletion region under the channel. Is this assumption reasonable?
	
	The voltage drop accross the depleted region is:
	
	\begin{equation}
	V = \frac{q N_A}{2 \epsilon_s} x_d^2
	\end{equation}
	
	In strong inversion:
		\begin{itemize}
		\item for the channel $V = 2\Phi_P \approx 0.8 V \Rightarrow x_{d,\ CH} = \sqrt{\frac{2 \Phi_P 2 \epsilon_s}{q N_A}}$;
		\item for source and drain $V = \Phi_i \approx 0.8 V \Rightarrow x_{d,\ S/D} = \sqrt{\frac{\Phi_i 2 \epsilon_s}{q N_A}}$.
		\end{itemize}
	
	Since $2\Phi_P$ and $\Phi_i$ are approximately equal we can say that $x_{d,\ CH} \approx x_{d,\ S/D}$. Thus, the assumption is correct.
	\end{itemize}

Now, considering figure \ref{figsce}, we can write:

	\begin{eqnarray*}
	 \left(x_1+ x_J\right)^2 + {x_d}^2 &=& \left(x_J+ x_d \right)^2
	\end{eqnarray*}

Developing the square terms we obtain:

	\begin{eqnarray*}
	   x_1&=& x_J \left(\sqrt{1+\frac{2x_d}{x_J}} -1 \right)
	\end{eqnarray*}

Substituting this expression in the threshold voltage expression for a short channel device we get:

	\begin{eqnarray*}
	   V_{ThLC}&=& V_{FB}+2\Phi_P-\frac{Q_d\;WL}{C_{ox}WL}\\
	           &=& V_{FB}+2\Phi_P+\frac{qN_A\;WL x_d}{C_{ox}WL}\\
	           &=& V_{FB}+2\Phi_P+\frac{qN_A x_d}{C_{ox}}
	\end{eqnarray*}

The quantity $- \frac{q N_A x_d x_1}{C_{OX}L}$ is a \textbf{correction factor}.

We said that the expression of the threshold voltage  for a short channel device is:

	\begin{eqnarray*}
	   V_{ThSC}&=& V_{FB}+2\Phi_P+\frac{qN_AW\left(L+L-2x_1\right)x_d}{2 C_{ox}WL}\\
	           &=& V_{ThLC} - \frac{qN_A x_d \dfrac{x_1}{L}}{C_{ox}}
	\end{eqnarray*}

We can observe that the SCE (represented by $- \frac{q N_A x_d x_1}{C_{OX}L}$) reduces the threshold voltage. When we scale a device, the ratio $\frac{x_1}{L}$ should be kept constant. If it doesn't remain constant, which means that L is scaled while $x_{1}$ (which is the junction depth) not, $\frac{x_1}{L}$ increases causing a further reduction of the threshold voltage. So, in order to keep limited the $V_{TH}$ reduction we have to scale $x_{1}$ that, however, is not a technological parameter.

	\begin{eqnarray*}
	   x_1&=& x_J \left(\sqrt{1+\frac{2x_d}{x_J}} -1 \right)
	\end{eqnarray*}

We can reduce $x_{1}$ scaling $x_{j}$ by the same factor of the channel length (that's why it is essential to scale S/D junctions like channel). But, looking at the expression of $x_{1}$, if we want to reduce it by a factor K, we have to \textbf{scale both the depletion width and the junction depth}:

	\begin{equation}
	x'_d = \frac{1}{K} x_d\ \text{and}\ x'_j = \frac{1}{K} x_j
	\end{equation}

So:

	\begin{equation}
	x'_1 = \frac{1}{K} x_j \left( \sqrt{1+ 2\frac{1/K \cdot x_d}{1/K \cdot x_j}} - 1 \right) = \frac{1}{K} \cdot x_1
	\end{equation}

Scaling $x_{d}$ and $x_{j}$ we are sure that the effect of the SCE on $V_{TH}$ is limited. If we don't respect this rule the consequence is \textbf{roll-off}. Roll-off means that the threshold voltage is dominated by the SCE:

\newpage

	\begin{figure}[H]
	\centering
	\includegraphics[width=\textwidth]{figrolloff}
	\caption{}
	\label{}
	\end{figure}

Looking at the graph, it can be seen that, in this case, with a channel length below 80 nm, the threshold voltage is subject to roll-off. 

\newpage

\subsection{DIBL effect}

The previous graph shows another effect called \textbf{DIBL (Drain Induced Barrier Lowering}; it is the influence of the drain voltage on the barrier): the red line is related to a higher value of $V_{DS}$; in this case the roll-off effect appears earlier.

It is clear that the reference channel length for a technology has to be carefully chosen in order to avoid the roll-off region:

	\begin{figure}[H]
	\centering
	\includegraphics[width=\textwidth]{12_L_opt}
	\caption{}
	\label{}
	\end{figure}

Because of process variations (the photolitography is not so accurate), the effective channel length can vary from a minimum value to a maximum one. It is important to choose the optimal length in such a way that, if its value changes, the roll-off region is still avoided and the threshold voltage remains the nominal one.

\end{document}
