\documentclass[a4paper, 12pt, twoside, openany]{report}
\usepackage[T1]{fontenc}
\usepackage[utf8]{inputenc}
\usepackage[english]{babel}
\usepackage{graphicx}
\usepackage[aboveskip = 5 pt]{caption}
%\usepackage{caption}
\usepackage{float}
\usepackage[amssymb]{SIunits}
\usepackage{amsmath, amssymb}
%\usepackage[makeroom]{cancel}
%\usepackage{subscript}
%\usepackage{empheq}
\usepackage{ltxtable}
\usepackage[table]{xcolor}
\usepackage{array}
\usepackage{booktabs}
\usepackage{subfig}
\usepackage{tabularx}
\usepackage{cite}
\graphicspath{{./}{Immagini/}}

\newcommand{\sub}{\textsubscript}
\newcommand{\super}{\textsuperscript}
\newcommand{\gr}{\cellcolor{gray}}
\newcommand{\ye}{\cellcolor{yellow}}
\newcommand{\re}{\cellcolor{red}}
\setlength{\topmargin}{-1.5cm}
\setlength{\textheight}{24.5cm}
\setlength{\textwidth}{17.5cm}
\setlength{\oddsidemargin}{-1.0cm}
\setlength{\evensidemargin}{-1.0cm}


\begin{document}

\tableofcontents


\chapter{Introduction to Scaling}

\section{Continuing Moore's Law}
Over the last decades, \textbf{Moore's Law} has represented the guideline for the evolution of \emph{Integrated Circuits (ICs)} as it predicts how the key features of ICs will behave in the following years. Moore's Law is based on the empirical observation of data related to ICs and it states that the number of transistors in a VLSI chip approximately doubles every two years; furthermore, it also predicts a reduction of cost per transistor, a growth in terms of die area and an overall greater integration on the chip.\\
In this report it will be analyzed how technology should be modified to allow an effective scaling of devices that can lead to \textbf{deeper integration}. In Figure (\ref{confronto}) a comparison between a very old process (3 um channel length) with a relatively newer one (250 nm channel length) is shown and it is possible to notice that device dimensions have become smaller and this fact introduces a lot of issues.

	\begin{figure}[h]
	\centering
	\includegraphics[width=0.85\textwidth]{Immagine}
	\caption{Comparison between two technological processes.}
	\label{confronto}
	\end{figure}

First it's important to say that devices have been scaled much more than interconnections; in other words, interconnects cannot be scaled with the same policy used for electronic devices. This could be a problem since in a technology where devices and interconnects have comparable dimensions, interconnects parasitics do not have influence on the overall performance; on the contrary, in a technology where the devices and the interconnections are not scaled with the same rate, as in this case (devices smaller than interconnects), \textbf{parasitics have a significant influence on the performance}. This is one of the consequences of scaling that can only be partially solved by technology, so it is essential to act from design point of view in order to limit the effect of parasitics. The other consequence of the scaling is related to \textbf{Short Channel Effects (SCE)}. Short channel effect is a phenomenon that is attributable to short channel devices, in other words when the channel length is of the same order of magnitude as the depletion-layer widths (xdD, xdS) of the source and drain junction. One of the most important consequences of this effect is a change of the threshold voltage and so a less strong electrostatic control of the channel. The SCE can be strongly limited by appropriate technological choices (this is the main target of integrated technologies); \\
The \textbf{ITRS (International Technology Roadmap for Semiconductors)} classifies technologies in the following way (2011 edition):

	\begin{itemize}
	\item \textbf{HP}: High Performance processes have as main aim to reduce the gate delay even if the power dissipation can be high; it is clear that this kind of processes cannot be used to produce portable devices, they can be used only for mainframes or data centres; in the last years HP processes have reached the natural end;
	\item \textbf{LOP}: Low Operating Power processes are used for relatively high performance mobile applications where the focus is on reduced operating power (acceptable performance and acceptable dynamic power consumption);
	\item \textbf{LSTP}: Low STandby Power processes have as main aim to reduce the static power for low performance and low battery capacity applications (for example mobile phones); so, LSTP processes try to reduce the power consumption when the device is in standby mode, paying in terms of performance (reduced clock frequency).
	\end{itemize}

Nowadays, HP is becoming marginal with respect to the other process families, because modern technology tends to pay attention in particular to the reduction of leakage (by avoiding to reduce too much the threshold voltage), since the supply voltage has been scaled more and more over the years. For these reasons, LOP technology is playing a key role: it has become the new HP family in \textbf{ITRS} (2013 edition), which will be used throughout this chapter for technological parameters. As a consequence, the old LSTP has become the new LP family. In conclusion, from now on this chapter will refer to two different technological families:

\begin{itemize}
\item \textbf{HP}: High Performance processes have as main goal to improve performance even though power consumption increases.
\item \textbf{LP}: Low Power processes have aims at reducing the static power for low performance and low battery capacity applications.
\end{itemize}

The \textbf{ITRS} makes previsions on which direction research will proceed regarding to different areas of technology up to about 12 years into the future. Until now, these previsions have been really accurate. In the following two tables, one per each technological family, there are all predicted parameters. For some of them the manufacturable solutions are known and they are being optimized, some others already have manufacturable solution, whereas for a few parameters there are not any manufacturable solutions available so far. A color legend will be used to distinguish among the different cases:

\begin{table}[h]
\centering
\begin{tabular}{|m {2cm} |l|}
\hline
&Manufacturable solutions exist and they are being optimized\\
\hline
\cellcolor{yellow}&Manufacturable solutions are known\\
\hline
\cellcolor{red}&Manufacturable solutions are NOT known\\
\hline
\cellcolor{gray}&End of technology\\
\hline
\end{tabular}
\caption{Color legend}
\label{tab:color}
\end{table}

As already stated above, there are two tables per technological family. That is due to different technological processes that can be exploited to implement both HP and LP families, which consist of Bulk technology and Multi-Gate (MG) technology. The former is going to disappear by the end of 2017 and the latter has become dominant in the last few years, whereas SOI technology has already become out-of-date. As a matter of fact, all the major foundaries in the world have already switched to MG technology, focusing their production on finFET-on-Bulk devices.\\
As Multi-Gate technology requires a different approach to scaling, in this part of the chapter only the tables related to the Bulk process will be reported. All data and graphs about MG devices will be showed and analyzed later in the chapter \ref{multigate}.


\chapter{Scaling on Bulk process}

\section{Main parameters for scaling} \label{sec:bulk}
All the parameters listed below in the two tables have been obtained by the ITRS roadmap. For each physical quantity will we consider both the values for High-Performance and Low-Power families. Moreover, since we are especially interested in the drive and subthreshold currents\footnote{They tell us how good a device is in conduction and off-state respectively.} a brief section is dedicated to their mathematical derivation.

\begin{table}[h]
\centering
\resizebox{\textwidth}{!}{\begin{tabular}{||l||c|c|c|c|c|c|c|c|c|c|c|c||}
\hline
Year&2013&2014&2015&2016&2017&2018&2019&2020&2021&2022&2023&2024\\
\hline
$L_{g}$ [nm]&20&18&16.7&15.2&13.9&12.7&11.6&10.6&\ye 9.7&\ye 8.8&\ye 8&\ye 7.3\\
\hline
$L_{ch}$ [nm]&16&14.4&13.4&12.2&11.1&10.2&9.3&8.5&\ye 7.8&\ye 7&\ye 6.4&\ye 5.8\\
\hline
$V_{dd}$ [V]&0.86&0.85&0.83&0.81&0.8&0.78&0.77&0.75&0.74&0.72&0.71&0.69\\
\hline
EOT [nm]&0.8&0.77&0.73&\ye 0.7&\ye 0.67&\ye 0.64&\ye 0.61&\ye 0.59&\ye 0.56&\ye 0.54&\ye 0.51&\re 0.49\\
\hline
$\epsilon_{HK}$&12.5&13&13.5&\ye 14&\ye 14.5&\ye 15&\ye 15.5&\ye 16&\ye 16.5&\ye 17&\ye 17.5&\re 18\\
\hline
$T_{HK}$ [nm]&2.56&2.57&2.53&\ye 2.51&\ye 2.49&\ye 2.46&\ye 2.42&\ye 2.42&\ye 2.37&\ye 2.35&\ye 2.29&\re 2.26\\
\hline
$Doping_{ch}$ [$10^{18}$/$cm^{3}$]&6&7&7.7&8.4&9&\gr &\gr &\gr &\gr &\gr  &\gr &\gr \\
\hline
CET [nm]&1.1&1.07&1.03&\ye 1&\ye 0.97&\ye 0.94&\ye 0.91&\ye 0.89&\ye 0.86&\ye 0.84&\ye 0.81&\re 0.79\\
\hline
$C_{ch}$ [fF/$\mu$m]&0.502&0.465&0.448&\ye 0.42&\ye 0.396&\ye 0.373&\ye 0.352&\ye 0.329&\ye 0.311&\ye 0.289&\ye 0.273&\re 0.255\\
\hline
$\mu$ [$cm^{2}$/V*s]&400&400&400&400&400&\gr &\gr &\gr &\gr &\gr &\gr &\gr \\
\hline
$I_{off}$ [pA/$\mu$m]&100&100&100&100&100&100&100&100&100&100&100&100\\
\hline
$I_{on}$ [$\mu$A/$\mu$m]&1348&1355&\ye 1340&\ye 1295&\ye 1267&\gr &\gr &\gr &\gr &\gr &\gr &\gr \\
\hline
$V_{t,lin}$ [V]&0.306&0.327&0.334&0.357&0.378&\gr &\gr &\gr &\gr &\gr &\gr &\gr \\
\hline
$V_{t,sat}$ [V]&0.19&0.2&0.206&0.218&0.23&\gr &\gr &\gr &\gr &\gr &\gr &\gr \\
\hline
S/D $R_{s}$ [$\omega$*$\mu$m]&188&179&171&\ye 162&\ye 156&\gr &\gr &\gr &\gr &\gr &\gr &\gr \\
\hline
$C_{fringing}/C_{intrinsic}$ &1.2&1.3&1.4&1.5&1.6&1.7&1.8&1.9&2&2&2&2\\
\hline
$C_{fringing}$ [fF/$\mu$m]&0.6&0.6&0.63&0.63&0.63&0.63&0.63&0.62&0.62&0.58&0.55&0.51\\
\hline
$C_{g,total}$ [fF/$\mu$m]&1.1&1.07&\ye 1.07&\ye 1.05&\ye 1.03&\ye 1.01&\ye 0.99&\ye 0.95&\ye 0.93&\ye 0.87&\ye 0.82&\re 0.77\\
\hline
$CV^{2}$ [fJ/$\mu$m&0.82&0.77&\ye 0.74&\ye 0.69&\ye 0.66&\ye 0.61&\ye 0.58&\ye 0.54&\ye 0.51&\ye 0.45&\ye 0.41&\re 0.36\\
\hline
$\tau$ [ps]&0.705&0.67&\ye 0.666&\ye 0.656&\ye 0.65&\gr &\gr &\gr &\gr &\gr &\gr &\gr \\
\hline
1/$\tau$ [1/ps]&1.42&1.49&\ye 1.5&\ye 1.52&\ye 1.54&\gr &\gr &\gr &\gr &\gr &\gr &\gr \\
\hline
$I_{d,satn}/I_{d,satp}$&1.25&1.24&1.22&1.21&1.2&1.19&1.18&1.16&1.15&1.14&1.13&1.12\\
\hline
\end{tabular}}
\caption{\textbf{Bulk process for HP (ITRS 2013)}}
\end{table}

\begin{table}[h]
\centering
\resizebox{\textwidth}{!}{\begin{tabular}{||l||c|c|c|c|c|c|c|c|c|c|c|c||}
\hline
Year&2013&2014&2015&2016&2017&2018&2019&2020&2021&2022&2023&2024\\
\hline
$L_{g}$ [nm]&23&21&19&18&16&14.6&13.3&12.2&11.1&10.1&\ye 9.3&\ye 8.5\\
\hline
$L_{ch}$ [nm]&18.4&16.8&15.2&14.4&12.8&11.7&10.6&9.8&8.9&8.1&\ye 7.4&\ye 6.8\\
\hline
$V_{dd}$ [V]&0.86&0.85&0.83&0.81&0.8&0.78&0.77&0.75&0.74&0.72&0.71&0.69\\
\hline
EOT [nm]&0.8&0.77&0.73&\ye 0.7&\ye 0.67&\ye 0.64&\ye 0.61&\ye 0.59&\ye 0.56&\ye 0.54&\ye 0.51&\re 0.49\\
\hline
$\epsilon_{HK}$&12.5&13&13.5&\ye 14&\ye 14.5&\ye 15&\ye 15.5&\ye 16&\ye 16.5&\ye 17&\ye 17.5&\re 18\\
\hline
$T_{HK}$ [nm]&2.56&2.57&2.53&\ye 2.51&\ye 2.49&\ye 2.46&\ye 2.42&\ye 2.42&\ye 2.37&\ye 2.35&\ye 2.29&\re 2.26\\
\hline
$Doping_{ch}$ [$10^{18}$/$cm^{3}$]&5&6&7&7.7&8.4&\gr &\gr &\gr &\gr &\gr &\gr &\gr \\
\hline
CET [nm]&1.1&1.07&1.03&\ye 1&\ye 0.97&\ye 0.94&\ye 0.91&\ye 0.89&\ye 0.86&\ye 0.84&\ye 0.81&\re 0.79\\
\hline
$C_{ch}$ [fF/$\mu$m]&0.577&0.542&0.509&\ye 0.497&\ye 0.455&\ye 0.429&\ye 0.404&\ye 0.379&\ye 0.356&\ye 0.332&\ye 0.317&\re 0.297\\
\hline
$\mu$ [$cm^{2}$/V*s]&400&400&400&400&400&\gr &\gr &\gr &\gr &\gr &\gr &\gr \\
\hline
$I_{off}$ [pA/$\mu$m]&10&10&20&20&50&\gr &\gr &\gr &\gr &\gr &\gr &\gr \\
\hline
$I_{on}$ [$\mu$A/$\mu$m]&490&459&456&\ye 485&\ye 422&\gr &\gr &\gr &\gr &\gr &\gr &\gr \\
\hline
$V_{t,lin}$ [V]&0.619&0.639&0.636&0.676&0.647&\gr &\gr &\gr &\gr &\gr &\gr &\gr \\
\hline
$V_{t,sat}$ [V]&0.528&0.543&0.533&0.54&0.53&\gr &\gr &\gr &\gr &\gr &\gr &\gr \\
\hline
S/D $R_{s}$ [$\omega$*$\mu$m]&188&179&171&\ye 162&\ye 156&\gr &\gr &\gr &\gr &\gr &\gr &\gr \\
\hline
$C_{fringing}/C_{intrinsic}$ &1.1&1.2&1.3&1.4&1.5&1.6&1.7&1.8&1.9&2&2&2\\
\hline
$C_{fringing}$ [fF/$\mu$m]&0.64&0.65&0.66&0.7&0.68&0.69&0.69&0.68&0.68&0.66&0.63&0.59\\
\hline
$C_{g,total}$ [fF/$\mu$m]&1.21&1.19&1.17&\ye 1.19&\ye 1.14&\ye 1.12&\ye 1.09&\ye 1.06&\ye 1.03&\ye 1&\ye 0.95&\re 0.89\\
\hline
$CV^{2}$ [fJ/$\mu$m]&0.9&0.86&0.81&\ye 0.78&\ye 0.73&\ye 0.68&\ye 0.65&\ye 0.6&\ye 0.57&\ye 0.52&\ye 0.48&\re 0.42\\
\hline
$\tau$ [ps]&2.128&2.08&2.132&\ye 1.992&\ye 2.159&\gr &\gr &\gr &\gr &\gr &\gr &\gr \\
\hline
1/$\tau$ [1/ps]&0.47&0.45&0.47&\ye 0.5&\ye 0.46&\gr &\gr &\gr &\gr &\gr &\gr &\gr \\
\hline
$I_{d,satn}/I_{d,satp}$&1.27&1.26&1.25&1.24&1.22&1.21&1.2&1.19&1.18&1.16&1.15&1.14\\
\hline
\end{tabular}}
\caption{\textbf{Bulk process for LP (ITRS 2013)}}
\end{table}

\noindent Let's start with the analysis, assuming roadmap definition for some while for others a comparison between the different families:

\paragraph{Gate Length $L_{g}$~:}The gate length is defined as the distance between the source and drain terminals. For example, taking into account the actual node (i.e. year 2016) for analysis, we can observe that the gate length is 15.2 nm for the HP logic, while in the LP logic reaches the value of 18 nm. What we can understand is that in order to get high performance it is necessary to reduce the channel length. First of all, reducing the channel length allows to have \emph{smaller parasitic capacitances} and, above all, a \emph{higher current}. So, generally a faster device is the combination of two main factors: smaller parasitic capacitances and larger $I_{ON}$ current.
This is true when considering the near-term period, defined approximately from 2013 up to 2020 in the Roadmap \textbf{PIDS}\footnote{\emph{Process Integration, devices and Structure}} chapter. Since it is coming to an end very soon, technology is going to move towards a new direction consisting of fabrication of advanced non-planar multi-gate and nanowire FETs below 10-nm gate length, dragging all the related issues: effective scaling of the device, control of the short channel effects and parasitic components.
Notice that in all the tables it is possible to distinguish also the effective channel length $L_{ch}$ defined as the distance between metallurgical source/drain junctions, and it is assumed equal to 80\% of the gate length; in this manner, the gate/source and gate/drain overlaps are 10\% of gate length at each side. This parameter is used in the computation of some quantities listed in the previous tables.

\paragraph{Power Supply Voltage $V_{DD}$~:} It is well known that $V_{dd}$ is more difficult to scale than other parameters, mainly because of the fundamental limit of the subthreshold swing $SS\simeq60 mV/decade$. This trend will continue and become more severe as $V_{DD}$ approaches the regime of 0.6 V. This fact, along with the continuous increase of current density (per unit of area) causes the dynamic power density to climb with scaling, soon reaching an unacceptable level.
For high-performance logic, dynamic power dissipation is expected to become particularly difficult to control due to greater chip complexity and larger transistor on-current, which aims to maintain high speed while scaling. The main indicator for power dissipation is reported in the table as $CV^{2}$:

\begin{itemize}
\item \textbf{C}: it is the total gate capacitance per micron device width while in inversion region ($C_{g,total}$). This is the sum of $C_{ch}$ and the gate fringing capacitance above.
\item \textbf{V}: here it is assumed equal to $V_{dd}$.
\end{itemize}

Historically, for the low-power family, devoted to the reduction of static power dissipation, hence the off-current, the main indicator had been the $I_{off}$ current: it has always been much lower than the $I_{off}$ for the HP technology. However, also the previous parameter, i.e. $CV^{2}$, is used to characterize the low-power family. A comparison between technological families tells us that for the HP family the obtained value is lower with respect to the LP family on account of a lower value of the total gate capacitance, while power supply voltage scales similarly both for HP and LP. As to understand this behavior in a more satisfactory way, let us consider to the following picture (\ref{CV2}):

	\begin{figure}[h]
	\centering
	\includegraphics[width=0.7\textwidth]{figCV2}
	\caption{Dynamic power $CV^{2}$ analysis.}
	\label{CV2}
	\end{figure}

\paragraph{Parameters $I_{on}$, $I_{off}$, $\tau$~:} These three important parameters are usually used to compare technologies and their main properties.

$I_{on}$ [$\mu A/\mu m$] is the maximum drive current per unit width of a transistor. It is defined by connecting MOSFET in saturation configuration: gate and drain terminals are tied to $V_{dd}$, while source and body ones are connected to the lowest potential (i.e. ground). Then, considering a $1 \mu m$ width transistor\footnote{In the following sections it will be referred to as $W_{ref}$}, we are able to measure the maximum saturation current flowing through the drain node:

\begin{equation}
I_{on} = I_{ds}(V_{gs}=V_{dd}, V_{ds}=V_{dd}, W=W_{ref})
\label{}
\end{equation} 

	\begin{figure}[h]
	\centering
	\includegraphics[width=0.85\textwidth]{figIon}
	\caption{Ion current as a function of time.}
	\label{}
	\end{figure}

Let us analyze the line in the table related to the $I_{on}$ values, for the two families HP and LP. \\In the former, the n-channel MOSFET saturation drive current is found to increase only for a few years and then it starts to drop. This behavior is drastically different from the previous projections where the current continued to grow every year. The reason for the drop of the on-current is mainly due to $V_{dd}$ scaling and significant source-drain tunneling for channel lengths below 10 nm, which makes a high threshold voltage necessary to keep the same $I_{off}$.\\
In the latter, the same trend as the one explained above is kept, with values of on-current almost equal to 35\% to the corresponding HP-node values in the table: let us recall that the main feature for the LP transistors is low static power at the expense of a reasonable speed reduction.

$I_{off}$ ($nA/\mu m$) is the current flowing out of the source node of the n-MOSFET when the gate is connected to ground and the drain to $V_{DD}$, as defined in the following equation:

\begin{equation}
I_{off} = I_{ds}(V_{gs}=0, V_{ds}=V_{dd}, W=W_{ref})
\label{}
\end{equation}

	\begin{figure}[h]
	\centering
	\includegraphics[width=0.85\textwidth]{figIoff}
	\caption{Ioff current as a function of time.}
	\label{}
	\end{figure}

In order to appreciate this small value, let us examine the transfer-characteristic with semi-logarithmic scale, which is necessary since the range of variation of $I_{off}$ is larger than the $I_{on}$ one. Let us take a look at the graph above.
In an ideal transistor $I_{off}$ should be zero. However, source-drain tunnelling phenomena, as mentioned before, make the device harder to turn off and increase the subthreshold swing (SS). Since for the HP family transistor the off-current is fixed to 100 nA/$\mu$m for all years, the tunnelling current requires the threshold voltage to be higher in order to maintain $I_{off}$ constant, and consequently leads to a reduction in the inversion charge.
In the case of low power devices this current is set to be four decades lower than that of HP, specifically 10 pA/$\mu m$. Note that to meet the leakage current requirement, the gate length scaling of LP logic lags behind that of HP logic.

An important parameter is represented by the ratio between $I_{on}$ and $I_{off}$: it reveals how good the electrostatic control of the channel performed by the gate is; the larger the ratio, the more effective it is the control. In an ideal technology this ratio should be infinite. Referring to the data analyzed so far, for the HP technology $\frac{I_{on}}{I_{off}} \approx 10^4$, while in the LOP logic it is roughly $10^7$. It is clear that the higher the ratio the higher it is $I_{on}$, boosting performance, and the lower it is $I_{off}$, leading to low static power. A detailed view for this parameter is showed in the Figure (\ref{OnOff}):

	\begin{figure}[!h]
	\centering
	\includegraphics[width=0.7\textwidth]{figOnOff}
	\caption{Ion/Ioff ratio.}
	\label{OnOff}
	\end{figure}

An efficient method to measure the performance of a process is to evaluate the intrinsic time $\tau$: it can be seen as the time required to remove the charge stored in the gate capacitance of a transistor with $W=W_{ref}$ by mean of the $I_{on}$ current.
The value listed in the tables are computed applying the following equation:

\begin{equation}
\tau = \frac{C_{g,total}*V_{dd}}{I_{on}}
\label{}
\end{equation}

	\begin{figure}[h]
	\centering
	\includegraphics[width=0.85\textwidth]{figtau}
	\caption{Intrinsic delay as a function of time.}
	\label{tau}
	\end{figure}

The dotted capacitor is the total gate capacitance per micron device width in inversion, as defined above in this chapter. It acts as load capacitance for the other transistor (the one of which we want to measure $\tau$). This expression is just an approximation since in a real transient the current varies during the discharge of the capacitor, and it is not constant. What we can state is that the intrinsic transistor delay can be reduced:

	\begin{enumerate}
		\item Reducing parasitics since in the gate capacitance, apart from $C_{ch}$, also the fringing value is considered;
		\item Increasing $I_{on}$: capacitor will be discharged faster.
	\end{enumerate}

Since the input capacitance depends mainly on the geometries, generally, if we want to improve the intrinsic time we have to act on $I_{on}$. This is another reason why, in advanced processes, the main aim is to increase $I_{on}$ as much as possible. $\tau$ and the clock frequency are not the same thing; however, increasing the former can probably lead to an improvement of the latter.
In generating the roadmap projection for logic technology, the guiding metric has been the transistor intrinsic speed, the inverse of $\tau$. Even if some new transistor delay metric has been developed, also in the 2013 edition original metric will be used, because it is sufficiently accurate to follow the key scaling trends, and for consistency with previous Roadmaps too.
Observing the graph in figure \ref{tau}, it can been seen that, for high-performance transistors, the intrinsic speed\footnote{It is the inverse of $\tau$ for a speed unit.} improves initially with a slope of 4\% per year, while in low-power family increases only at a slope of roughly 2\% per year. 


\paragraph{Threshold voltage $V_{T}$~:} To define how good a technology is, the threshold voltage can be considered as target: all the choices that can be made depend on $V_{T}$. Starting from the HP technology, we can observe that the value of $V_{T}$ is quite low (it is around or under 200 mV).\\
What has been kept constant during the years is the ratio $V_{DD}/V_{T}$. This ratio is the fundamental parameter to \textbf{distinguish a technological process} from another.

	\begin{figure}[H]
	\centering
	\includegraphics[width=0.75\textwidth]{figVddVt}
	\caption{$V_{DD}/V_{T}$ in function of time.}
	\label{tau}
	\end{figure}

Up to the 2011 edition of the \textbf{ITRS} there has been a significantly different value of this ratio for the three families. For HP, $V_{DD}/V_{th} \approx 6/7$. Generally, when the \textbf{ratio is high} it means that the considered process will be characterized by a \textbf{high} on-current. For many years (during the golden age of the CMOS technology: '70, '80, beginning of '90) $V_{DD}$ had been 5 V and $V_{th}$ 1 V, so their ratio was equal to 5.\\
In LOP the ratio is lower and it is, approximately, equal to 3 since $V_{th}$ is higher.\\
In LSTP the quantity $V_{DD}/V_{th}$ is slightly smaller, approximately less than 3 ($\approx$ 2.5).
By looking at the 2013 edition, instead, the following ratios are reported for the HP and LP family respectively: the former is supposed to constantly decrease from 5 down to 3 over the years, the latter will reduce from 2 down to approximately 1.5. 

\paragraph{Equivalent Oxide Thickness $T_{ox}$~:} The thickness of the gate oxide is reduced node by node. For example, in the HP process, in 2012, $T_{ox}$ is equal to 7 \r{A}; considering that the $SiO_{2}$ lattice constant is 2.7, having a thickness of 7 \r{A} means having roughly 3 atomic layers. This thickness is not enough to prevent \emph{tunneling} and it causes an \textbf{increase of the gate current}, which, instead, should be close to zero since the gate is supposed to be isolated. This could represent another problem for deeply scaled devices.\\
Especially in the last nodes, while still considering the three different technology families\footnote{Again, up to ITRS 2011 edition.}, $I_{gate}$ (expressed as current density, A/cm$^{2}$) is characterized by high values. In particular, HP has the highest $I_{gate}$ value among the three technologies. LSTP has the minimum value due to a thicker gate oxide.\\
\textbf{Why in a scaled device the gate oxide thickness must be reduced?} The answer is that reducing $t_{ox}$, although it causes an increase of parasitic capacitances, it also boosts $C_{ox}$ (gate capacitance). In fact:

\begin{equation}
C_{ox} = \frac{\epsilon_{ox}}{t_{ox}}
\label{}
\end{equation}

Moreover, as shown in the equation below, $I_{ON}$ linearly depends on $C_{ox}$ so, increasing the latter means increasing the current.

\begin{equation}
I_{ON} = \frac{\mu_n C_{ox} W}{2 L} (V_{DD} - V_{TH})^2
\label{}
\end{equation}

Therefore, with a \textbf{thinner oxide} we are able to control, in a better way, the quantity of charge in the channel and this means \textbf{higher transconductance}, that turns into \textbf{higher current}.\\
However, this is not the main motivation to reduce the oxide thickness in a scaled process. The main motivation is that reducing $t_{ox}$ allows to reduce $V_{TH}$:

\begin{equation}
V_{TH} = V_{FB} + 2 \Phi_{P} + \frac{1}{C_{ox}} \sqrt{2 q \epsilon_s N_A 2 \Phi_{P}}
\label{V_th}
\end{equation}

So, the followed sequence is:

\begin{equation}
\text{SCALING} \ \Rightarrow \ t_{ox} \ \downarrow \ \Rightarrow \ C_{ox} \ \downarrow \ \Rightarrow \ V_{TH} \ \uparrow \Rightarrow \ I_{ON} \ \uparrow
\end{equation}

Besides the effort of reducing $t_{ox}$ as to obtain a better electrostatic control of the channel to switch the device on and off more efficiently, there are other reasons why the gate oxide thickness has been reduced over and over with years. Two of them are the most significant:


\begin{itemize}
\item QM effects. For deeply-scaled devices the classical model that has been used for many year to describe the distribution of the inversion charge in the channel is not accurate any more. The inversion charge does \textbf{not} concentrate at the surface between gate oxide and channel, but it is pushed further away from the oxide, into the silicon. The distance between the silicon-oxide surface and the maximum concentration of carriers in the silicon is called \textbf{Dark Space (DS)}. As a consequence, there exists a new capacitance $C_{DS}=\epsilon_{s}/ DS$ that is in series with $C_{ox}$ and the total capacitance will be: \\ $C_{tot}=C_{ox}*C_{DS}/(C_{ox}+C_{DS})$.\\ It is straight-forward to notice that the gate electrostatic control will worsen on account of dark space.
\item Polydepletion effect. A similar phenomenon occurs in devices with polysilicon gates, since there exists a depletion layer in the gate itself due to the applied voltage. Once again, a second capacitance can be inserted in series to $C_{ox}$, thus reducing the total capacitance and worsening the device performance. However, the polydepletion effect has been resolved by replacing polysilicon with metals. Nowadays, metal-gate technology has completely taken over polysilicon-gate technology.
\end{itemize} 

%from http://www.electrical4u.com/mos-capacitor-mos-capacitance-c-v-curve/
\begin{figure}[h]
\centering
\includegraphics[width=0.4\textwidth]{c_ox_vs_vg}
\caption{Polydepletion and quantum mechanics effects reduce $C_{ox}$ significantly.}
\label{fig:Cox-reduction}
\end{figure}

As it has been stated at the beginning of the paragraph, the main drawback of the reduction of $t_{ox}$ consists of the strong increase of the \textbf{gate leakage} current: in order to avoid this undesired consequence, the gate structure and materials should be completely changed. Independently from the technological family \textbf{scaling means higher $I_{off}$} as long as the bulk planar process is concerned. New processes try to maintain limited, under control, $I_{off}$. For the moment we consider $I_{off}=I_{subV_{th}}$ for a n-MOS transistor whose drain is tied to $V_{dd}$ and its source, gate and body are tied to ground (dual condition for a p-MOS)\label{ioff_config}; therefore, we are not considering other leakage sources as junction leakages, gate leakages, etc\dots.


\subsection{Subthreshold current}

The subthreshold current expression is something that is of great interest to us since it allows to evaluate the $I_{off}$ current when the device is in the configuration explained above \ref{ioff_config}. The hypothesis under which we try to find the analytical expression for the subthreshold current are:

\begin{itemize}
\item In the Weak Inversion (WI) region, \textbf{diffusion current is prevailing over drift current}, where the latter is negligible compared to the former.
\item $0<V_{GS}<V_{th}$, $V_{DS}>0$ and $\Phi_{p}<\Psi(0)<2\Phi_{p}$ (where $\Psi(0)$ is the surface potential at the interface). 
\end{itemize}

For the sake of completeness, the main steps that lead to the $I_{subV_{th}}$ expression will be reported:

\begin{enumerate}
\item Find a good analytical expression for the inversion charge $Q_{n}(y)$ in WI by solving Poisson's equation.
\item Write $Q_{n}(y)$ as a function of the channel potential.
\item Find an expression to evaluate $\Psi(0)$ as a function of $V_{GS}$.
\item Write:
\begin{equation}\label{punto4}
I_{DS}=-I_{diff_{n}}=-\frac{\partial Q_{n}(y)}{\partial y}
\end{equation}
\end{enumerate}

The inversion charge expression in WI will be:

	\begin{eqnarray*}
	      Q_n(y)&\approx &-C_{dep}~~V_T
	             \left(e^{\displaystyle \frac{\Psi_S(y)-2\Phi_P}{V_T}}
	              \right)\\[2ex]
	       C_{dep}&=& {\sqrt{\frac{q \epsilon_S N_A}{ 4\Phi_P}}}
	\end{eqnarray*}

Just for completeness, in the picture (\ref{ChargeDist}) it has been plotted the charge density as a function of the surface potential in all the working region of the transistor (accumulation; depletion, WI; SI)

	\begin{figure}[H]
	\centering
	\includegraphics[width=0.75\textwidth]{figCharge}
	\caption{Charge density as a function of surface potential for complete model.}
	\label{ChargeDist}
	\end{figure}

The inversion charge at source and drain will be respectively:
\begin{eqnarray*}
Q_n(0)&\approx&-C_{dep} V_T
\left(e^{\displaystyle \frac{\Psi_S(0)-2\Phi_P}{V_T}}
\right)\\[2ex]
Q_n(L)&\approx&-C_{dep} V_T
\left(e^{\displaystyle \frac{\Psi_S(0)- {V_{DS}}-2\Phi_P}{V_T}}
\right)
\end{eqnarray*}

The gradient can be expressed through an approximation:

\begin{eqnarray*}
\frac{\partial Q_n(y)}{\partial y} &\approx& - \frac{Q_n(L)-Q_n(0)}{L}
\end{eqnarray*}

Finally, by applying the equation \ref{punto4} we get the following expression for the subthreshold current:

\begin{eqnarray}
I_{DS}&=&\frac{W \mu_n C_{dep} {V_T}^2}{L}
         \left( 1-e^{\displaystyle\frac{-V_{DS}}{V_T}}\right) 
         e^{\displaystyle \frac{ {V_G-V_{Th}}}{V_T\left(1 +\frac{C_{dep}}{C_{ox}}\right)}} 
\end{eqnarray}

The expression above can be further simplified by making two more assumptions:

\begin{itemize}
\item Let us define the parameter $m=1+\frac{C_{dep}}{C_{ox}}$.
\item For $V_{DS}>5V_{T}$ we can consider:
\begin{eqnarray}
\left( 1-e^{\displaystyle\frac{-V_{DS}}{V_T}}\right) \simeq 1
\end{eqnarray}
\end{itemize} 

Therefore, we finally obtain

\begin{equation} \label{final_Ids}
I_{DS} \approx \frac{\mu_n W C_{dep} V_T^2}{L} e^{\frac{V_{GS} - V_T}{m V_T}}
\end{equation}

The expression \ref{final_Ids} is the one that will be used to evaluate the effects of scaling on the subthreshold current and $I_{off}$ in particular. \\
Even though the subthreshold current is an important parameter for any device family, it is not sufficient to determine whether a certain technological process is a good process or not. A more reliable parameter is represented by the subthreshold swing (SS) that tells us how easily a transistor can be switched off. How we can evaluate SS? We can apply the definition of slope which is the inverse of the derivative of the $log_{10}I_{DS}$ curve:

	\begin{equation}
	SS = \left( \frac{d(log_{10}I_{subV_{th})}}{dV_G} \right)^{-1}
	\end{equation}
	
	\begin{equation}
	SS = \left \{ \frac{d}{dV_G} \left[ log_{10} \left( \frac{\mu_n W C_{dep} V_T^2}{L} e^{\frac{V_G - V_T}{m V_T}} \right) 	\right] \right \}^{-1}
	\end{equation}
	
	\begin{equation}
	SS = \left \{ \frac{d}{dV_G} \left[ log_{10} \left( \frac{\mu_n W C_{dep} V_T^2}{L} \right) + \frac{V_G - V_T}{m V_T} 		\cdot \frac{1}{ln10} \right] \right \}^{-1}
	\end{equation}
	
	\begin{equation}
	SS = \left( \frac{1}{m V_T} \cdot \frac{1}{ln10} \right)^{-1} = m V_T ln10
	\end{equation}
	
	\begin{equation}
	SS = 2,3 \cdot V_T \cdot m
	\end{equation}
	
	\begin{equation}
	SS|_{300K}  = 2,3 \cdot 26mV \cdot m \approx 59.8\cdot m mV
	\end{equation}

SS is measured in \textbf{mV/dec}, since it tells us how many milliVolts the gate voltage must change in order to decrease/increase the $I_{subV_{th}}$ by one order of magnitude. The lower is the value of SS the better is the technological process. \textbf{The minimum reachable value for SS is 59.8 mV/dec}, when $m = 1$. So, from the technological point of view, all the efforts are focused towards the direction that allows to have $m \approx 1$. Typical SS values for the bulk CMOS technology are in the range 80 - 100 $mV/dec$. More advanced technologies have a lower SS and in some cases it is possible to reach values near the theoretical limit. A good example is provided by the \textbf{Fully-Depleted Silicon-On-Insulator (FD-SOI)} process, where $SS=60mV/dec$. The FD-SOI process has a very good value of the SS due to the extremely reduced thickness of the silicon overlay, which reduces $C_{dep}$. FD-SOI devices are characterized by many advantages compared to standard Bulk transistors; in fact, one the main advantages is represented by the reduced SS value that makes the control on the device more effective.\\
As a matter of fact, $m = 1+ \frac{C_{dep}}{C_{OX}}$ and its value can be minimized by lowering the ratio $C_{dep}/C_{OX}$ in two possible ways:

\begin{enumerate}
\item $C_{dep} \rightarrow 0$ (best choice from the technological point of view);
\item $C_{OX} \rightarrow \infty$.
\end{enumerate}

In the graph below are shown different curves related to SS: 

	\begin{figure}[H]
	\centering
	\includegraphics[width=0.75\textwidth]{figsubvt6}
	\caption{Subthreshold slopes for different technology nodes.}
	\label{}
	\end{figure}


%%% QUESTA PARTE E' STATA RIDOTTA (SOPRA) PERCHE' LA TRATTAZIONE ERA LUNGA E NON CENTRATA SULLO SCALING. SERVIVA MOSTRARE UNA FORMULA PER LA CORRENTE DI SOTTOSOGLIA MA SENZA DILUNGARSI COME ERA STATO FATTO QUI. E' UN CORSO DI TECNOLOGIA, NON DI DISPOSITIVI!! %%%

%It is very important to understand which are the physical mechanisms that produce $I_{subV_{th}}$.
%
%	\begin{figure}[H]
%	\centering
%	\includegraphics[scale=0.44]{07_MOS_band_diag}
%	\caption{Band diagram of a n-MOS}
%	\label{}
%	\end{figure}
%
%$V_{S}$ is the voltage drop across silicon, when we describe a MOS system, that is equivalent to $\Psi(0)$.\\
%$V_{ox}$ is the oxide voltage drop.
%
%Considering this voltage references, we draw the most important figure for a MOS system which is $|Q_t|$ ($Q_{t}$ is the total charge of the silicon in the MOS system):
%
%	\begin{figure}[H]
%	\centering
%	\includegraphics[scale=0.43]{07_total_charge}
%	\caption{The quantity of charge in function of $V_{S}$.}
%	\label{}
%	\end{figure}
%
%This graph is related to a \textbf{p-type device}.\\
%The red curve corresponds to $Q_{t}$, not $|Q_{t}|$. $Q_{t}$ is the sum of ions (depletion charge) and free carriers (electrons and holes).\\
%Between the accumulation and the depletion regions there is a point where the total charge in the silicon is zero. At this point we have that $V_S = 0$ so, this should be the true origin of the graph. When $V_S = 0$ we are applying at the gate (so at the entire MOS system) a voltage equal to $V_{FB}$.\\
%If we want to increase the quantity of positive charge in a p-type device we need to increase the number of holes.\\
%The maximum voltage drop applicable to silicon is $2\Phi_P$ (the typical value of $\Phi_P$ is a p-type silicon is roughly 0.35-0.40). Starting from the point in which $V_S = 2\Phi_P$ the system is in strong inversion. In strong inversion region the channel is created and the quantity of charge in the channel increases linearly with the gate voltage. In the \textbf{strong inversion region} we evaluate $I_{on}$: infact when $V_S = 2\Phi_P$, we are applying at the gate a voltage equal to $V_{TH}$, at least. In the \textbf{weak inversion region}, where $ \Phi_P < V_S < 2\Phi_P$ and $V_G < V_{TH}$, we evaluate $I_{off}$. When the system is in weak inversion, $|Q_{t}|$ is not enough high to provide $I_{on}$ and not enough low to consider $I_{off}$ = 0 (the transistor is not completely off). It can be considered as a transition region.\\
%Only when $V_S < \Phi_P$ we can consider the transistor is completely off.
%
%Let us consider this figure:
%
%	\begin{figure}[H]
%	\centering
%	\includegraphics[scale=0.6]{07_graph1}
%	\caption{}
%	\label{}
%	\end{figure}
%
%When $V_G > V_{TH}$ we have $I_{on}$ current that is mainly drift current; when $V_G < V_{TH}$ we have $I_{off}$ current that is mainly diffusion current. This can be grater than zero even if the total quantity of carriers is very small because what is relevant is the variation of the concentration, not its absolute value. When we consider $I_{on}$ we generally neglect diffusion current, but, actually, it is present and it is due to the inversion charge (red) when the system is in weak inversion:
%
%	\begin{figure}[H]
%	\centering
%	\includegraphics[scale=0.5]{07_graph2}
%	\caption{Depletion and inversion layer in the channel.}
%	\label{}
%	\end{figure}
%
%The depletion charge increase moving from source to drain and the inversion charge decrease moving from S to D.\\
%The concentration $Q_{n}$ varies along the channel and its derivative $\frac{\partial Q_n (y)}{\partial y}$ becomes dominant.
%(We recall that the expression of $Q_{n}$ in strong inversion is $Q_{n} = C_{ox} (V_{GS} - V_{TH})$)\\
%In order to  $I_{off}$, we need to:
%
%\begin{enumerate}
%\item find out an expression for $Q_{n}$ in weak inversion;
%\item evaluate $I_{subV_{th}$ as diffusion current starting from $\frac{\partial Q_n (y)}{\partial y}$.
%\end{enumerate}
%
%A MOS system can be analyzed in terms of total charge density in the silicon below the gate oxide:
%
%\begin{itemize}
%\item writing the Poisson equation:
%	
%	\begin{eqnarray*}
%	\frac{d^2  {\Psi(x)}}{dx^2} &=& -\frac{q \left[ p(x)-n(x)-N_A^- \right]}{\epsilon_S}\\[1ex]
%	p(x)&=&N_A e^{\displaystyle\frac{- {\Psi(x)}}{kT}}\\[2ex]
%	n(x)&=&\frac{n_i^2}{N_A}e^{\displaystyle\frac{ {\Psi(x)}}{kT}}
%	\end{eqnarray*}
%
%\item integrating it and then applying the Gauss’s theorem at the silicon silicon dioxide interface.
%\end{itemize}
%
%The total charge density can be expressed as:
%
%    \begin{eqnarray*}
%      Q_S&=&-\frac{\sqrt{2} \epsilon_S V_T}{L_D}\cdot \\
%         & &\displaystyle\left\{e^{\frac{-\Psi(0)}{V_T}}+ {\frac{\Psi(0)}{V_T}}-1+
%             \frac{n_i^2}{N_A^2}\left[\displaystyle {e^{ \frac{\Psi(0)}{V_T}}} -
%             \frac{\Psi(0)}{V_T}-1\right]\right\}^{\frac{1}{2}}\\[3ex]
%       L_D&=& \sqrt{\frac{\epsilon_S V_T}{qN_A}}\;\; \rm{Debye\;\;length}
%    \end{eqnarray*}
%
%Now we want to find an approximate expression for Q when the system is in weak inversion. Assuming that $\Phi_P = \frac{kT}{q}ln\left(\frac{N_A}{n_i}\right) \approx 300 mV$ and $V_{TH} \approx 26 mV$:
%
%	\begin{figure}[H]
%	\centering
%	\includegraphics[scale=0.35]{07_graph3}
%	\caption{}
%	\label{tau}
%	\end{figure}
%
%\begin{itemize}
%\item $\frac{\Phi(0)}{V_{TH}} \approx 12$ when $\Phi(0) = \Phi_P$;
%\item$\frac{\Phi(0)}{V_{TH}} \approx 24$ when $\Phi(0) = 2 \Phi_P$.
%\end{itemize}
%
%This assumption has a consequence on the terms that we can neglet in the expression of $Q_{S}$:
%
%
%	\begin{eqnarray*}
%      Q_S&=&-\frac{\sqrt{2} \epsilon_S V_T}{L_D}\cdot \\
%         & &\displaystyle\left\{{e^{\frac{-\Psi(0)}{V_T}}}+ {\frac{\Psi(0)}{V_T}}{-1}+
%             \frac{n_i^2}{N_A^2}\left[\displaystyle {e^{ \frac{\Psi(0)}{V_T}}} -
%             {\frac{\Psi(0)}{V_T}}{-1}\right]\right\}^{\frac{1}{2}}\\[3ex]      
%    \end{eqnarray*}
%
%
%\begin{itemize}
%\item $e^{-\frac{\Phi(0)}{V_{TH}}} \ll 1$ when $\Phi(0) = 12\ \text{or}\ 24$ so it is negligible;
%\item -1 negligible;
%\item$-\frac{\Phi(0)}{V_{TH}} -1 \ll e^{\frac{\Phi(0)}{V_{TH}}}$ when $\Phi(0) = 12\ \text{or}\ 24$ so it is negligible.
%\end{itemize}
%
%The total charge density for different substrate doping levels can be related to the $q\Phi_P$ values:
%
%	\begin{figure}[H]
%	\centering
%	\includegraphics[width=0.75\textwidth]{figsubvt1}
%	\caption{The quantity of charge in function of $V_{S}$ for different substrate doping levels.}
%	\label{}
%	\end{figure}
%
%Remembering that:
%
%\begin{equation}
%\Phi_P = \frac{kT}{q}ln\left(\frac{N_A}{n_i}\right) \Rightarrow \frac{N_A}{n_i} = e^{\frac{q \Phi_P}{V_{TH}}} \Rightarrow \frac{n_{i}^2}{N_{A}^2} = e^{\frac{-2q \Phi_P}{V_{TH}}}
%\end{equation}
%
%Substituting the above expression of $\frac{n_{i}^2}{N_{A}^2}$ in the previous equation of $Q_{S}$ and considering only the terms not negligible, the approximated expression of $Q_{S}$ when the system works in weak inversion ($\Phi_P < \Phi(0) < 2\Phi_P$) is:
%
%   \begin{eqnarray*}
%      Q_S&=&-\displaystyle\frac{\sqrt{2}\epsilon_S V_T}{L_D}
%            \left\{\frac{\Psi(0)}{V_T} + 
%          e^{\displaystyle \frac{\Psi(0)-2\Phi_P}{V_T}}\right\}^{\frac{1}{2}}
%    \end{eqnarray*}
%
%Now we replace the expression of the Debye length in $Q_{S}$ obtaining:
%
%    \begin{eqnarray*}
%      Q_S&=&-V_T \sqrt{\frac {2q \epsilon_S N_A}{V_T} \cdot \frac {\Psi(0)}{V_T}}\cdot \left\{ {\frac{V_T}{\Psi(0)} e^{\displaystyle \frac{\Psi(0)-2\Phi_P}{V_T}}} +1 \right\}^{\frac{1}{2}}
%     \end{eqnarray*}
%
%Since we are in weak inversion (under the threshold $V_{T}$), $\Psi(0) - 2\Phi_P < 0$ (because $\Phi_P < \Psi(0) < 2\Phi_P$); so $\frac{V_T}{\Psi(0)}e^{\frac{\Psi(0) - 2\Phi_P}{V_T}} < 1$ and we can apply the approximation $\sqrt{x+1} \approx 1+ \frac{x}{2}$ obtaining:
%
% \begin{eqnarray*}
%      Q_S &=&-\sqrt{2q \epsilon_S N_A \Psi(0)}\cdot \left\{ 1 +\frac{1}{2}\left( \frac{V_T}{\Psi(0)} e^{\displaystyle\frac{\Psi(0)-2\Phi_P}{V_T}}\right) \right\}
%  \end{eqnarray*}
%
%Now, since $Q_S = Q_n + Q_d$, we can obtain $Q_{n}$ as $Q_n = Q_S - Q_d$, where:
%
%\begin{itemize}
%\item $Q_{S}$ is the \textbf{total charge in the semiconductor} (in this case we are considering a p-type substrate);
%\item $Q_{n}$ is the \textbf{inversion layer charge};
%\item $Q_{d}$ is the \textbf{depletion layer charge}.
%\end{itemize}
%
%
%	\begin{figure}[H]
%	\centering
%	\includegraphics[scale=0.45]{08_charge_regions}
%	\caption{}
%	\label{}
%	\end{figure}
%
%
%
%
%
%Observing the above figure we can say that:
%
%\begin{itemize}
%\item in the \textbf{accumulation region} the charge is only due to \textbf{holes};
%\item in the \textbf{depletion region} the charge is due to \textbf{ions};
%\item in the \textbf{weak inversion region} the charge is mainly due to \textbf{ions} plus a \textbf{small quantity of electrons};
%\item in the \textbf{strong inversion region} the total charge is due to \textbf{electrons} plus a negligible part of ions.
%\end{itemize}
%
%So, in weak inversion, to get the inversion charge we have to subtract the depletion charge from the total charge, as we wrote before ($Q_n = Q_S - Q_d$). We know the expression for $Q_{S}$ but we need to evaluate the expression of the  depletion charge under the gate. $Q_{d}$ can be obtained evaluating the thickness of the depleted region: in weak inversion the channel is uniform and it presents a voltage drop that is exactly $\Psi(0)$. Having an ideal voltmeter, the voltage drop can be measured taking as reference the voltage of the substrate and measuring the surface potential that is the value that can be measured at the interface between silicon and oxide:
%
%	\begin{figure}[H]
%	\centering
%	\includegraphics[scale=0.6]{08_channel_voltage_meas}
%	\caption{}
%	\label{}
%	\end{figure}
%
%The general expression for the potential across the depleted region is:
%
%\begin{equation}
%\Psi(0) = \frac{qN_A}{2 \epsilon_S} x_d^2
%\label{depl_pot}
%\end{equation}
%
%From (\ref{depl_pot}) we can derive $x_{d}$:
%
%\begin{equation}
%x_d = \sqrt{\frac{2 \epsilon_S \Psi(0)}{qN_A}}
%\label{xd}
%\end{equation}
%
%Considering a \textbf{uniform doped substrate}, we can write the expression of the depleted charge as:
%
%\begin{equation}
%%Q_d = -q N_A x_d
%\label{Qd}
%\end{equation}
%
%Substituting (\ref{xd}) in (\ref{Qd}) we get:
%
%\begin{equation}
%%Q_d = -\sqrt{q N_A 2 \epsilon_S \Psi(0)}
%\label{}
%\end{equation}
%
%Now, we can derive $Q_{n}$ as $Q_n = Q_S - Q_d$ (the expression of $Q_{S}$ is the one that we have derived after applying the expansion $\sqrt{x+1} \approx 1+ \frac{x}{2}$):
%
%  \begin{eqnarray*}
%      Q_n &=&-\sqrt{2q \epsilon_S N_A \Psi(0)} \cdot \left\{\frac{1}{2}\left( \frac{V_T}{\Psi(0)} e^{\displaystyle 			\frac{\Psi(0)-2\Phi_P}{V_T}}\right) \right\}
%  \end{eqnarray*}
%
%The chart below compares three curves associated to:
%
%\begin{enumerate}
%\item $Q_{S}$ not approximated;
%\item $Q_{S}$ approximated (Qs\_a);
%\item $Q_{S}$ approximated and expanded (Qs\_as);
%\end{enumerate}
%
%	\begin{figure}[H]
%	\centering
%	\includegraphics[width=0.70\textwidth]{figsubvt2}
%	\caption{}
%	\label{}
%	\end{figure}
%
%It can be seen that the approximation that we did on $Q_{S}$ is really good; infact, for $\Phi_P \approx 0.54 V$, the \% errors of the approximate and approximate-expanded functions with respect to the original one are around 1-2\%:
%
%	\begin{figure}[H]
%	\centering
%	\includegraphics[width=0.70\textwidth]{figsubvt3}
%	\caption{}
%	\label{}
%	\end{figure}
%
%The expressions that we have derived since now are valid only if the \textbf{channel is uniform ($V_{DS} = 0$)}.
%
%Now we have to consider that the \textbf{channel is not uniform (drain biased $\Rightarrow$ channel not uniform)}.\\
%When we measure $I_{off}$ we are in this situation:
%
%	\begin{figure}[H]
%	\centering
%	\includegraphics[scale=0.36]{08_Ioff_meas}
%	\caption{The MOS is switched off while the drain is biased.}
%	\label{}
%	\end{figure}
%
%We consider a bi-dimensional system (x, y): the origin is the interface between Si and oxide, y is the direction along the channel. We define a bi-dimensional potential $\Psi(x,y)$; when $x = 0 \rightarrow \Psi(0, y) = \Psi_S(y)$, where $\Psi_S(y)$ is the surface potential  function of the position (y) in the channel.\\
%Using the quasi-Fermi level for the concentration of the electrons along the channel we can write:
%
%  \begin{eqnarray*}
%      Q_n(y)&=&-\frac{\sqrt{2q \epsilon_S N_A \Psi_S(y)}}{2} \frac{V_T}{\Psi_S(y)}
%             \left(e^{\displaystyle \frac{\Psi_S(y)-2\Phi_P}{V_T}}
%              \right)\\[2ex]
%      Q_n(y)&=&-\sqrt{\frac{q \epsilon_S N_A}{2\Psi_S(y)}}~~ V_T
%             \left(e^{\displaystyle \frac{\Psi_S(y)-2\Phi_P}{V_T}}
%              \right)
%     \end{eqnarray*}
%
%We can see that the inversion charge along the channel decreases moving from source to drain. Moreover, the inversion charge, in weak inversion, depends on both $\Psi_S(y)$ in the squareroot and in the exponential, but the \emph{main dependency is the exponential one}. The term $1/\sqrt{\Psi_S(y)}$ can be approximated with $1/\sqrt{\Psi_S(0)}$ where, since we are interested to the \emph{system near the strong inversion (just below the threshold)}, we can approximate $\Psi_S(0) \approx 2\Phi_P$.
%This is done in order to obtain a simpler expression of $Q_{n}(y)$.
%
%	\begin{eqnarray*}
%	      Q_n(y)&\approx &-\sqrt{\frac{q \epsilon_S N_A}{4\Phi_P}}~~V_T
%	             \left(e^{\displaystyle \frac{\Psi_S(y)-2\Phi_P}{V_T}}
%	              \right)\\[2ex]
%	       C_{dep}&=& {\sqrt{\frac{q \epsilon_S N_A}{ 4\Phi_P}}}
%	\end{eqnarray*}
%
%Summirizing, the steps to obtain the approximate expression of $Q_{n}(y)$ are:
%\begin{enumerate}
%\item approximating $\Psi_S(y)$ with the value at the surface $\Psi_S(0)$;
%\item approximating $\Psi_S(0)$ with the value near the threshold $2\Phi_P$.
%\end{enumerate}
%
%$C_{dep}$ is the \textbf{depletion capacitance} (actually, it is a density). Since it is a capacitance, it can be written as:
%
%\begin{equation}
%%C_{dep} = \frac{\epsilon_S}{x_d}
%\label{C_dep}
%\end{equation}
%
%Infact, in the depleted region of Si the capacitance can be evaluated by considering the depleted silicon as a dielectric material:
%
%	\begin{figure}[H]
%	\centering
%	\includegraphics[scale=0.4]{08_C_dep}
%	\caption{}
%	\label{}
%	\end{figure}
%
%We recall that in weak inversion:
%
%\begin{equation}
%x_d = \sqrt{\frac{2 \epsilon_S \Psi(0)}{qN_A}},\ \Psi(0) \approx 2\Phi_P \Rightarrow x_d = \sqrt{\frac{2 \epsilon_S 2\Phi_P}{qN_A}}
%\label{xd_2}
%\end{equation}
%
%Substituting (\ref{xd_2}) in (\ref{C_dep}) we get:
%
%\begin{equation}
%%C_{dep} = \frac{\epsilon_S}{\sqrt{\frac{4 \epsilon_S \Phi_P}{qN_A}}}
%\label{}
%\end{equation}
%
%This is the expression of the capacitance of the depleted region below the channel when the system is near the threshold; this expression is the same in the strong inversion case.
%
%After defining $C_{dep}$ we can write the expression of $Q_{n}(y)$ as:
%
%	\begin{eqnarray*}
%%	Q_n(y)&\approx& -C_{dep} V_T \left(e^{\displaystyle \frac{\Psi_S(y)-2\Phi_P}{V_T}}\right)
%	\end{eqnarray*}
%
%Assuming a linear variation of the $\Psi_S(y)$ between Source and Drain, we can be compare the curve associated with the original expression of $Q_{n}(y)$ with the curve associated to the approximated one:
%
%	\begin{figure}[H]
%	\centering
%	\includegraphics[width=0.65\textwidth]{figsubvt4}
%	\caption{}
%	\label{}
%	\end{figure}
%
%It can be seen that the approximation gives very good results as the approximated curve (Qn\_a) has a trend which is really similar to the trend of the original curve (Qn). Moreover we can observe that the inversion charge, when the transistor is in the subthreshold region, decays exponentially along the channel. It is worth noticing that the graph above shows the trend of the inversion charge density and it assumes really small values (few electrons). So, the inversion charge in subthreshold condition is too small to produce significant drift current and the total current is mainly due to the diffusion current (that represent a leakage) which, instead, depends on the gradient.
%
%We define:
%
%	\begin{eqnarray*}
%%	Q_n(0)&\approx&-C_{dep} V_T
%	             \left(e^{\displaystyle \frac{\Psi_S(0)-2\Phi_P}{V_T}}
%	               \right)\\[2ex]
%%	Q_n(L)&\approx&-C_{dep} V_T
%	             \left(e^{\displaystyle \frac{\Psi_S(0)- {V_{DS}}-2\Phi_P}{V_T}}
%	              \right)
%	\end{eqnarray*}
%
%where $Q_{n}(0)$ is the quantity of the inversion charge at the source and $Q_{n}(L)$ is the concentration of the inversion charge at the drain, both in WI.
%
%We can approximate the diffusion current in this way:
%
%	 \begin{eqnarray*}
%	  I_{ndiff}(y) &=& D_nW \frac{\partial  {qt_{ch} n(y)}}{\partial y} D_nW \frac{\partial Q_n(y)}{\partial y}
%	 \end{eqnarray*}
%
%The gradient can be approximated as:
%
%	 \begin{eqnarray*}
%	  \frac{\partial Q_n(y)}{\partial y} &\approx& - \frac{Q_n(L)-Q_n(0)}{L}
%	 \end{eqnarray*}
%
%Now we can write that:
%
%	 \begin{eqnarray*}
%	  I_{ndiff} &=& -I_{DS}\\[2ex]\\
%	  I_{DS}    &=& \frac{W D_n C_{dep} V_T}{L}  
%	            e^{\displaystyle \frac{ {\Psi_S(0)-2\Phi_P}}{V_T}}
%	           \left( 1-e^{\displaystyle\frac{-V_{DS}}{V_T}}\right)\\[2ex]
%	  I_{DS}    &=& \frac{W \mu_n C_{dep} {V_T}^2}{L}  
%	            e^{\displaystyle \frac{ {\Psi_S(0)-2\Phi_P}}{V_T}}
%	           \left( 1-e^{\displaystyle\frac{-V_{DS}}{V_T}}\right)\\[2ex]
%	  \end{eqnarray*}
%
%Regarding the last expression ($I_{DS}$), we can note that:
%
%\begin{itemize}
%\item when $V_{DS} = 0 \Rightarrow I_{DS} = 0$ (no voltage drop between source and drain means no current flux);
%\item when $V_{DS} = V_T \Rightarrow I_{DS} = \text{const.}$
%\end{itemize}
%
%In the second case what happens is that, increasing $V_{DS}$, even if we are in the subthreshold region, leads to the channel pinch-off at the drain (the transistor is in saturation).
%
%Moreover, it is important to notice the dependency of $I_{DS}$ on the channel length L (\textbf{channel length modulation}): reducing the dimensions of the transistor (in particular L) the current increases because the gradient of the charge along the channel is increased. This explains why scaling has the effect of increasing the subthreshold current.
%
%Now, looking at the $I_{DS}$ equation that we have derived before, we can observe that there is an exponential dependency on $\Psi_S(0)$. However, the value of $\Psi_S(0)$ is unknown, it depends on the gate voltage. Thus, we have to \emph{put in relation $\Psi_S(0)$ with the external gate voltage}.
%
%The band diagram of an MOS structure is the following:
%
%	\begin{figure}[H]
%	\centering
%	\includegraphics[scale=0.5]{08_band_diag}
%	\caption{The band diagram of an MOS structure.}
%	\label{}
%	\end{figure}
%
%Observing it we can write:
%
%\begin{equation}
%%V_G - V_{FB} = \Psi_S(0) + V_{OX} \Rightarrow V_G = V_{FB} + \Psi_S(0) + V_{OX}
%\label{Vg}
%\end{equation}
%
%In weak inversion the charge distribution is:
%
%	\begin{figure}[H]
%	\centering
%	\includegraphics[scale=0.4]{08_charge_distrib}
%	\caption{The charge distribution in weak inversion.}
%	\label{}
%	\end{figure}
%
%\begin{equation}
%\rho(x) = -qN_A
%\label{}
%\end{equation}
%
%The electric field is:
%
%	\begin{figure}[H]
%	\centering
%	\includegraphics[scale=0.3]{08_electric_field}
%	\caption{The electric field in weak inversion.}
%	\label{}
%	\end{figure}
%
%\begin{equation}
%%E_S(0) = \frac{qN_Ax_d}{\epsilon_S}
%\label{electric_field}
%\end{equation}
%
%$E_{S}(0)$ is the electric field at the interface between silicon and oxide.
%
%The potential is:
%
%	\begin{figure}[H]
%	\centering
%	\includegraphics[scale=0.3]{08_potential}
%	\caption{The potential in weak inversion.}
%	\label{}
%	\end{figure}
%
%\begin{equation}
%\Phi_S(0) = \frac{qN_Ax_d^2}{\epsilon_S}
%\label{}
%\end{equation}
%
%$\Phi_S(0)$ is the potential at the interface between silicon and oxide.
%
%Substituting the expression of $x_{d}$ in \ref{electric_field} we get:
%
%\begin{equation}
%%E_S(0) = \frac{qN_Ax_d}{\epsilon_S} = \frac{qN_A}{\epsilon_S} \sqrt{\frac{2\epsilon_S\Psi_S(0)}{qN_A}} = \sqrt{\frac{2 qN_A \Psi_S(0)}{\epsilon_S}}
%\label{}
%\end{equation}
%
%Now we can evaluate the voltage drop across the oxide:
%
%\begin{equation}
%5E_S(0) \epsilon_S = E_{OX} \epsilon_{OX} \Rightarrow E_{OX} = \frac{\epsilon_S}{\epsilon_{OX}}E_S(0) = \frac{1}{\epsilon_{OX}} \sqrt{2q \epsilon_S N_A \Psi_S(0)}
%\label{}
%\end{equation}
%
%\begin{equation}
%%V_{OX} = t_{OX} E_{OX} = \frac{1}{C_{OX}} \sqrt{2q \epsilon_S N_A \Psi_S(0)}
%\label{VOX}
%\end{equation}
%
%Now we have an expression of $V_{OX}$ in function of $\Psi_S(0)$.
%
%Substituting (\ref{VOX}) in (\ref{Vg}) we get:
%
%\begin{equation}
%%V_G = V_{FB} + \Psi_S(0) + \frac{1}{C_{OX}} \sqrt{2q \epsilon_S N_A \Psi_S(0)}
%\label{}
%\end{equation}
%
%This expression is too complex and we want to approximate it performing an expansion of $V_{G}$ around $2\Phi_P$:
%
%\begin{equation}
%f(\Psi_S(0)) \approx f(2\Phi_P) + f'(2\Phi_P)(\Psi_S(0) - 2\Phi_P)
%\label{}
%\end{equation}
%
%The expanded version of $V_{G}$ is:
%
%	\begin{eqnarray*}
%	   V_G &=&  \underbrace{V_{FG}+2\Phi_P+\frac{1}{C_{ox}}\sqrt{2q\epsilon_S N_A 2\Phi_P}}_{f(2\Phi_P)} +  \underbrace{\left( 1+ \frac{1}{C_{ox}}\sqrt{\frac{q\epsilon_SN_A}{4\Phi_P}}\right)}_{f'(2\Phi_P)} \left(\Psi_S(0)-2\Phi_p\right)
%	\end{eqnarray*}
%
%The first part of the equation (f(2$\Phi_P$)) is indeed the \textbf{nominal threshold voltage} of the MOSFET:
%
%	\begin{eqnarray*}
%	   V_{Th} &=& V_{FB}+2\Phi_P+\frac{1}{C_{ox}}\sqrt{2q\epsilon_S N_A 2\Phi_P}
%	\end{eqnarray*}
%
%So we can arrange the expression of $V_{G}$ as:
%
%\begin{equation}
%%V_G = V_{th} + (1+ \dots) (\Psi_S(0) - 2\Phi_P)
%\label{Vg_arr}
%\end{equation}
%
%
%
%From (\ref{Vg_arr}) we can derive:
%
%          \begin{eqnarray*}
%             \Psi_S(0)-2\Phi_P=\frac{V_G-V_{Th}}{1 +\frac{C_{dep}}{C_{ox}}}
%          \end{eqnarray*}
%
%This expression can be substituted in the expression of $I_{DS}$ obtaining:
%
%    \begin{eqnarray*}
%    I_{DS}    &=& \frac{W \mu_n C_{dep} {V_T}^2}{L}
%                  \left( 1-e^{\displaystyle\frac{-V_{DS}}{V_T}}\right) 
%            e^{\displaystyle \frac{ {V_G-V_{Th}}}{V_T\left(1 +\frac{C_{dep}}{C_{ox}}\right)}} 
%     \end{eqnarray*}
%
%When we are interested in evaluating the threshold voltage, we have that $V_{DS} \approx V_{DD}$. For this reason, setting $V_{DS} = V_{DD}$, the term $e^{\frac{-V_{DS}}{V_T}}$ can be negleted since it is $\ll 1$.
%
%Defining \textbf{m}, we finally have the complete \textbf{expression for the subthreshold current}:
%
%    \begin{eqnarray*}
%       {m}&=& \left(1 +\frac{C_{dep}}{C_{ox}}\right)
%     \end{eqnarray*}
%
%   \begin{eqnarray*}
%    I_{DS}    &\approx & \frac{\mu_n W \left(m-1 \right) C_{ox} {V_T}^2}{L}\;
%            e^{\displaystyle \frac{ {V_G-V_{Th}}}{ mV_T}} 
%     \end{eqnarray*}
%
%This last equation can also be written as:
%
%\begin{equation*}[box=\fbox]{align}
%%I_{DS} \approx \frac{\mu_n W C_{dep} V_T^2}{L} e^{\frac{V_G - V_T}{m V_T}}
%\label{final_Ids}
%\end{equation*}
%
%The factor m is the \emph{most used parameter to optimize technological processes}.
%
%We can comment the equation (\ref{final_Ids}) saying that when the transistor is in subthreshold region ($V_G < V_T$), the term $e^{\frac{V_G - V_T}{m V_T}}$ is negative; this means that the subthreshold current decays exponentially when the transistor is switched off.
%
%The following chart shows the trend of $I_{DS}$.
%
%	\begin{figure}[H]
%	\centering
%	\includegraphics[scale=0.45]{08_Ids_Vgs_graph}
%	\caption{}
%	\label{}
%	\end{figure}
%
%\begin{itemize}
%\item In WI, as said before, $I_{DS}$ decays exponentially; since we are using a semilogarithmic chart, the current has a linear trend;
%\item in SI,  $I_{DS}$ increases in a logarithmic way assuming, then, a constant value.
%\end{itemize}
%
%
%We recall that the final approximated expression of the subthreshold current is the following (for $V_{DS} \approx V_{DD}$):
%
%   \begin{eqnarray*}
%    I_{DS}    &\approx & \frac{\mu_n W \left(m-1 \right) C_{ox} {V_T}^2}{L}\;
%            e^{\displaystyle \frac{ {V_G-V_{Th}}}{ mV_T}} 
%     \end{eqnarray*}
%
%Where `m' is:
%
%    \begin{eqnarray*}
%       {m}&=& \left(1 +\frac{C_{dep}}{C_{ox}}\right)
%     \end{eqnarray*}
%
%We remember that $C_{dep}$ and $C_{OX}$ are called depletion and oxide capacitance, respectively, but they are \emph{densities} measured in fF/cm$^{2}$.
%
%The behavior of the sub-threshold current for HP100 HP65 HP32 technologies is:
%
%	\begin{figure}[H]
%	\centering
%	\includegraphics[width=0.70\textwidth]{figsubvt5}
%	\caption{}
%	\label{}
%	\end{figure}
%
%We can notice that:
%
%\begin{itemize}
%\item the subthreshold current decades exponentially when we move in the subthreshold region;
%\item long channel (HP100) devices have a lower subthreshold current for the same value of $V_{GS} - V_{TH}$ with respect to small channel (HP32) devices.
%\end{itemize}
%
%If we compare the subthreshold current evaluated using the approximated expression and the one measured on a real device, we can assert that the theoretical model exploited to get the approximated equation is quite accurate. We can refer to this model as \textbf{Taur model}.
%
%Generally, it is difficult to say if a process is good or not just analizing the $I_{off}$ current since this parameter is not enough and it can be misleading. Now we define another parameter that can be used to classify a process. It is called \textbf{SS (Subthreshold Swing/Slope)}:
%
%	\begin{figure}[H]
%	\centering
%	\includegraphics[scale=0.5]{09_SS}
%	\caption{}
%	\label{}
%	\end{figure}
%
%For example, considering the $T_{1}$ curve, $SS_{1}$ can be defined as the variation of the gate voltage required to get a reduction of one order of magnitude of the leakage current (from 10I' to I').
%
%Supposing that $T_{1}$ and $T_{2}$ are two different technological processes (with $I_{off,\ T2} < I_{off,\ T1}$), which one is better? If we considered only the subthreshold current we could say that $T_{2}$ is better than $T_{1}$. But, we can observe that $SS2 > SS1$, which means that, in the $T_{2}$ case, we have to apply a bigger $\Delta V_G$ to get a reduction of one order of magnitude of the leakage current.
%
%So, the SS tells us how easily a transistor can be switched off. How we can evaluate SS? We can apply the definition of slope which is the inverse of the derivative of the $log_{10}I_{DS}$ curve:
%
%	\begin{equation}
%%	SS = \left( \frac{d(log_{10}I_{subV_{th})}{dV_G} \right)^{-1}
%	\label{}
%	\end{equation}
%	
%	\begin{equation}
%%	SS = \left \{ \frac{d}{dV_G} \left[ log_{10} \left( \frac{\mu_n W C_{dep} V_T^2}{L} e^{\frac{V_G - V_T}{m V_T}} \right) \right] \right \}^{-1}
%	\label{}
%	\end{equation}
%	
%	\begin{equation}
%%	SS = \left \{ \frac{d}{dV_G} \left[ log_{10} \left( \frac{\mu_n W C_{dep} V_T^2}{L} \right) + \frac{V_G - V_T}{m V_T} \cdot \frac{1}{ln10} \right] \right \}^{-1}
%	\label{}
%	\end{equation}
%	
%	\begin{equation}
%%	SS = \left( \frac{1}{m V_T} \cdot \frac{1}{ln10} \right)^{-1} = m V_T ln10
%	\label{}
%	\end{equation}
%	
%	\begin{equation}
%%	SS = 2,3 \cdot V_T \cdot m
%	\label{}
%	\end{equation}
%	
%	\begin{equation}
%%	SS|_{300K}  = 2,3 \cdot 26mV \cdot m \approx 59.8 mV \cdot m
%	\label{}
%	\end{equation}
%
%%SS is measured in \textbf{mV/dec}.
%
%The lower is the value of SS the better is the technological process. \textbf{The minimum reachable value for SS is 59.8 mV/dec}, when $m = 1$. So, from the technological point of view, all the efforts are concentrated in the direction that allows to have $m \approx 1$. Typical SS values for the bulk CMOS technology are in the range 80 - 100 mV. More advanced technologies have a lower SS and in some cases it is possible to reach values near the theoretical limit.\\
%Since $m = 1+ \frac{C_{dep}}{C_{OX}}$, its value can be reduced lowering the ratio $C_{dep}/C_{OX}$ and this can be done in two ways:
%
%\begin{enumerate}
%\item $C_{dep} \rightarrow 0$;
%\item $C_{OX} \rightarrow \infty$.
%\end{enumerate}
%
%The best choice is the first one from the technological point of view.
%
%In the graph below are shown different curves related to SS:
%
%
%
%	\begin{figure}[H]
%	\centering
%	\includegraphics[width=0.65\textwidth]{figsubvt6}
%	\caption{}
%	\label{}
%	\end{figure}
%
%We can observe that long channel devices (HP100) have a lower subthreshold current but a higher SS while small channel devices (HP32) have a higher subthreshold current but a smaller SS. So, in this case scaling is working in the right direction since it allows to reduce SS.

\subsection{Drive current}

The simplest expression of the drive current is the following:

\begin{equation}
I_{DS,\ sat} = \frac{\mu_n C_{OX} W}{2L} (V_{GS} - V_{TH})^2
\label{}
\end{equation}

However, this model \emph{overestimates} the current. Why? Because we consider just a linear dependency of the inversion charge on the gate voltage neglecting other square-root terms present in the charge expression. Clearly, this leads to an inexact approximation.

In general, we can use three models to estimate the drive current:

\begin{enumerate}
\item the expression obtained from the linear approximation of $Q_{n}(y)$

	\begin{eqnarray*}
	 I_{DS} &=&\frac{W \mu_n C_{ox}}{L}\left[\left(V_G-V_{Tn}\right)V_{DS}-\frac{V_{DS}^2}{2}\right]
	\end{eqnarray*}

\item the expression obtained using the `m' factor

	\begin{eqnarray*}
	 I_{DS} &=&\frac{W \mu_n C_{ox}}{L}\left[\left(V_G-V_{Tn}\right)V_{DS}-\frac{ {m}}{2}V_{DS}^2\right]
	\end{eqnarray*}

\item the expression derived using the complete $Q_{n}(y)$ (Spice model)

	\begin{eqnarray*}
	I_{DS} &=&\frac{W \mu_n C_{ox}}{L}\left[\left(V_G-V_{Tn}\right)V_{DS}-\frac{V_{DS}^2}{2}\right]
	        -\gamma_B \frac{W}{L}\mu_n C_{ox}\cdot\\
	        & &\cdot\left\{\frac{2}{3}\left[\sqrt{\left(2\Phi_P +V_{DS}\right)^3}
	            -\sqrt{\left(2\Phi_P\right)^3}\right]-\sqrt{2\Phi_P}\cdot V_{DS}\right\}
	\end{eqnarray*}

\end{enumerate}

We can observe that the second expression is equal to the first one except for \textbf{`m'} that acts as a \textbf{correction factor}.

Comparing the three equations it can be noted that the `m' factor expression represents a very good approximation and it has one big advantage: it is a very simple expression that can be easily used for calculations.

	\begin{figure}[H]
	\centering
	\includegraphics[width=0.55\textwidth]{figidson1}
	\caption{Comparison of the three different expression of $I_{DS}(V_{DS})$.}
	\label{}
	\end{figure}

Regarding the `m' factor equation for the drive current we can say that, the smaller m is, the higher $I_{DS}$ is. Again, the ideal value of m is one. So, by improving it ($m \approx 1$), we can get a lower subthreshold current and, at the same time, a higher drive current.

The \textbf{maximum drive current} (saturation current) can be obtained imposing maximum drain-source voltage:

	\begin{eqnarray*}
	     V_{DSmax} &=& \frac{V_{GS} - V_{TH}}{m}\\
	   I_{DSmax} &=& \frac{\mu_n W C_{ox}}{2L \cdot  {m}}\left(V_{GS}-V_{Tn}\right)^2
	\end{eqnarray*}

Again, this equation is equal equal to the one that can be obtained from $I_{DS}$ approximated except for `m' that acts as a correction factor.

The maximum speed of carriers in the channel is limited by \textbf{velocity saturation}\footnote{It occurs when the carriers lose energy on account of the increased number of scattering events with the lattice atoms.} and this has a strong effect on the maximum current that a transistor can drive. The effect of velocity saturation due to the longitudinal field can be modelled using the following expression ($I_{DS}$ is expressed using `m'):

	\begin{eqnarray*}
	     \mu_n&=&\frac{\mu_{n0}}{1+\frac{\mu_{n0}V_{DS}}{Lv_{sat}}}\\[2ex]
	   I_{DS} &=&\frac{W \mu_{n0} C_{ox}Wv_{sat}}{Lv_{sat}+\mu_{n0}V_{DS}}
	             \left[\left(V_G-V_{Tn}\right)V_{DS}-\frac{ {m}}{2}V_{DS}^2\right]
	\end{eqnarray*}

(\emph{\textbf{Note}: it is not requested to know the expression of $I_{DS}$ but just to remember what velocity saturation is.})


\newpage
\section{Scaling policies}

So far we have considered the main features and the key parameters that are used to study, analyze and compare the different technological processes and families, and how they are affected by scaling as we get closer and closer to the limits of the Bulk process. Although the Bulk process is getting to its natural end as stated at the beginning of the chapter, it is still of great interest to study its behavior and the limits and problems related to the scaling policies applied to it. As a matter of fact, many ideas and solutions that belong to the Bulk legacy can be exported and exploited by the new Multi-Gate technological process. Independently from the policy adopted, in \textbf{planar technology} scaling means reducing dimensions.\\
As we will see in a while, both for devices and interconnects we have different policy of scaling. Generally, transistors scaling is more aggressive than interconnects scaling. Actually, it is important to remark that the scaling policy for interconnects varies according to the interconnect hierarchy, which is mainly due to the different delays for local and global interconnects. Please refer to the following section for more details.
%: $\text{\textbf{K}} > \text{\textbf{S}}$

%Similarly, we have three different scaling policies for interconnects: 
%
%\begin{enumerate}
%\item \textbf{Ideal scaling};
%\item \textbf{Quasi-ideal scaling};
%\item \textbf{Fixed-height scaling}.
%\end{enumerate}
%
%The scaling factor for interconnects is \textbf{S}. Furthermore, another discrimination can be done among local, intermediate and global wires: local wires scale almost like devices, global wires mainly do not scale, whereas intermediate wires are somewhere in between. This approach is called \textbf{Hierarchical scaling policy} for interconnects.\\


%%QUESTO E' INCOMPLETO E FORSE STAREBBE MEGLIO DA UN'ALTRA PARTE
%The transistors dimensions are reduced by K:
%\begin{align*}
%&W \xrightarrow[]{scaling} W' & &L \xrightarrow[]{scaling} L' \\
%&W' = W \cdot \frac{1}{K} & &L' = L \cdot \frac{1}{K}\\
%&\frac{W}{W'} = K & &\frac{L}{L'} = K
%\end{align*}
%For interconnects we have (W is the width of the interconnect):
%\begin{align*}
%&W \xrightarrow[]{scaling} W'\\
%&W' = W \cdot \frac{1}{S}\\
%&\frac{W}{W'} = S
%\end{align*}

\subsection{Device scaling}

Even though new kinds of transistors are being developed and exploited in integrated circuit fabrication to solve some critical problems arisen in the planar Bulk process as technology keeps scaling, it is important to consider the scalable MOSFET device that has been taken as a reference point over the last years. Moreover, some general concepts and ideas that are eligible for the Bulk process may come in handy for future generations of new devices. That is the reason why we will still focus on the Bulk process in this part of the chapter as we discuss about scaling policies. 

\subsubsection{Basic scalable MOSFET}

The most common MOSFET structure is the LDD (Lightly-Doped Drain) one:

	\begin{figure}[h]
	\centering
	\includegraphics[width=0.6\textwidth]{figbasmos}
	\caption{A Lightly-Doped Drain transistor.}
	\label{}
	\end{figure}

\textbf{LDD} mos are used since the extension at the drain end of the channel can \textbf{significantly reduce the electric field} peak, thus \textbf{limiting the hot carrier injection} phenomenon that could severely damage the device. If there are too many electrons trapped in the gate oxide, the threshold voltage of the device can permanently change and irreversibly damage the transistor.
Fundamental parts that characterize this structure are:

\begin{itemize}
\item gate structure: $T_{gate}$, $T_{OX}$;
\item source/drain extensions: $X_{j}$ (junction depth);
\item source/drain contacts: $X_{jc}$ (contact junction depth);
\item spacers (regions beside the gate): $T_{SW}$.
\end{itemize}

In the basic CMOS process, the polysilicon\footnote{Nowadays, polysilicon has been replaced by metal as to avoid poly-depletion effects.} gate is used to mask source/drain implantation. Now we want to realize extensions and contacts exploiting the \textbf{LDD CMOS process}:

	\begin{figure}[H]
	\centering
	\includegraphics[width=0.65\textwidth]{figldd}
	\caption{Steps to realize LDD structure.}
	\label{LDD_process}
	\end{figure}

In the standard process source and drain are implanted using a high dose, $N_D = 10^{15}$, and a high energy, $E = 80/100\ keV$. In the LDD process, the dose is reduced by three orders of magnitude and the energy is around 20/30 keV in order to realize very shallow junctions ($X_j \approx 10 nm$). So, looking at figure \ref{LDD_process}, we can distinguish four steps:

\begin{enumerate}
\item implantation of shallow lightly-doped source/drain extensions ($N_D = 10^{12}$, $E = 20/30\ keV$);
\item CVD deposition of a thick ($\approx 1\mu m$) layer of silicon oxide;
\item anisotropic etching to remove $SiO_{2}$: part of the silicon oxide is not removed and it constitutes spacers;
\item high energy implantation of source/drain contacts ($N_D = 10^{15}$, $E = 80/100\ keV$).
\end{enumerate}

Spacers can be realized in $SiO_{2}$ or in silicon nitride ($Si_{3}N_{4}$) (more often in $SiO_{2}$). These structures are used to define source and drain contacts.

When we realize S/D extensions it is necessary to perform thermal annealing for a short time, in order to obtain shallow junctions. On the contrary, to realize S/D contacts, that have to be deep, it is required to perform thermal annealing for a longer time and at a higher temperature than in the previous case. The consequence of performing "heavy" thermal annealing consists of the lateral diffusion of the extensions below the gate, reducing the effective length of the channel, and increasing parasitics (because of the source-gate and drain-gate coupling). There exist techniques to reduce the lateral diffusion: the main idea is to implant S/D contacts before extensions.

The \emph{great advantage} of the LDD technology consists of the possibility to be \textbf{implemented without any additional mask and photolitography} because the oxide deposition and source/drain contacts implantation are obtained using the masking action of spacers. This means that all processes can be easily updated to LDD. Starting from the end of 80s, the reference device has become the LDD one. 

An example of the reference MOSFET is the following:

	\begin{figure}[H]
	\centering
	\includegraphics[width=0.6\textwidth]{figscalablemos}
	\caption{Microscopic picture of a basic scalable mosfet.}
	\label{}
	\end{figure}

We can see that the device has very large spacers and, as a consequence, very long extensions, which is negative since the resistance of the extension regions becomes very high because they are lightly-doped and long. As a consequence, we can notice the presence of silicide (the silicide is a compound of silicon and metal, in this case nickel) contacts useful to reduce the parasitic resistance at least for the source/drain contacts. We recall that source/drain extensions have been introduced in order to improve the reliability of the device limiting the high electric field near the drain. Extensions offer another advantage: \textbf{better scaling} because contact geometry does not depend on S/D extension dimensions. This means that we can scale independently $X_{j}$ and $X_{jc}$ so, we can have a thin junction depth (generally scaled by a factor K) and a thick contact region (less scaled than the junction depth) to reduce the resistance. Why does the junction depth $X_{j}$ have to be scaled by the same factor by which we scale the device dimensions? The answer is given by the \textbf{short channel effects}, which become more and more relevant as technology scales and they must be kept as small as possible. \\

Traditionally, for the Bulk process we can distinguish among three different scaling approaches for \textbf{devices} that will be analyzed in details:

\begin{enumerate}

\item \textbf{Constant voltage scaling};

Constant voltage scaling features imply a reduction by K in all geometrical dimensions, while $V_{DD}$ is kept constant (and so is $V_{th}$). For many years, during the golden age of the CMOS technology, $V_{DD}$ has \textbf{not changed} (5 V) even if the dimensions of the transistors were scaled. This choice has been made in order to \textbf{guarantee compatibility}: same logic signals. We know that a fundamental parameter useful to distinguish different technological processes is the \textbf{ratio} $\frac{V_{DD}}{V_{TH}}$; \textbf{in constant voltage scaling this ratio too has to be maintained constant}. This means that even $V_{TH}$ has to be kept constant. But what are the consequences, from the technological point of view, of these choices? Let us analyze the following table:

 \begin{table}[h]
 \begin{center}
     \begin{tabular}{||l||c|c||} \hline \label{cvs}
      V$_{DD}$ & -& $ {1}$\\
      \hline
      L,W,$x_J$       &- & $ {1/K}$\\
      \hline
      N & $x_d=\sqrt{\frac{2\epsilon_S(\Phi_i+V_{DD})}{qN}}$&$ {K^2}$ \\
      \hline
      V$_{DD}/V_{Th}$ &- & ${1}$\\
      \hline
      V$_{Th}$ & $V_{FB}+2\Phi_P+\frac{1}{C_{ox}}\sqrt{2qN_A\epsilon_S 2\Phi_p}$& $ {1}$\\
      \hline
       $C_{ox}$&  $\epsilon_{ox}/Tox$ &$ {K}$\\ 
      \hline
      $T_{ox}$       &- & $ {1/K}$\\
      \hline
       $C_{dep}$&  $\sqrt{\frac{q\epsilon_SN_A}{4\Phi_P}}$ & ${K}$\\
      \hline
       $m$&  $1 +\frac{C_{dep}}{C_{ox}}$ &$ {1}$\\ 
      \hline
      ${\cal E}_y$ & $V_{DD}/L$ & $ {K}$\\
      \hline
      ${\cal E}_x$ & $V_{DD}/Tox$ & $ {K}$\\
      \hline
      $I_{DSsat}$      &$\frac{\mu_n Cox W}{2L}(V_{GS}-V_T)^2$ &$ {K}$\\ 
      \hline  
       $I_{ON}$      &- &$ {K}^2$\\ 
      \hline
       $I_{subVt}$      &$\frac{\mu_n W \left(m-1 \right) C_{ox} {V_T}^2}{L}\;
            e^{\displaystyle \frac{ {V_G-V_{Th}}}{ mV_T}}$ &$ {K}$\\ 
      \hline  
       $I_{off}$      &- &$ {K}^2$\\ 
      \hline
      $\tau$     &$Cox\cdot WL\cdot V_{DD}/Ion$ &$ {1/K^2}$\\ 
      \hline 
      $P_{Dyn}^{tr}$      &$fC_LV_{DD}^2$  &$ {K}$\\ 
      \hline 
      $P_{Dyn}^{Den}$      &-&$ {K^3}$\\ 
      \hline
       $P_{Isub}^{tr}$      &$I_{subV_{th}}V_{DD}$  &$ {K}$\\ 
      \hline 
       $P_{Isub}^{Den}$      &-&$ {K^3}$\\ 
      \hline
     \end{tabular}
\caption{CVS effects on main parameters}
\label{}
    \end{center}
	\end{table}

We can see that L and W are scaled as 1/K. $x_{J}$ is the \textbf{juction depth} of the source-drain junction near the channel and it is scaled as well of the same factor K:

	\begin{figure}[h]
	\centering
	\includegraphics[scale=0.5]{09_junction_depth_scaling}
	\caption{Geometry dimensions scaling.}
	\label{}
	\end{figure}

Why $x_{J}$ needs to be scaled? At first glance we can say that if the thickness of the highly doped regions is smaller their resistance increase and this is a drawback. So, why, despite this disadvantage, is $x_{J}$ scaled? Because it helps to control lateral diffusions. After source and drain are implanted, it is necessary to perform thermal annealing. During this phase lateral diffusions are created and this is a problem because they increase capacitances and create other problems. In order to reduce lateral diffusions it is necessary to implant less deeply the dopant. So, when we reduce planar dimensions (W, L) we have also to reduce the vertical dimension ($x_{J}$) otherwise it becomes impossible to control lateral diffusions. Moreover, scaling $x_{J}$ by the same factor of the channel helps to \textbf{keep under control the short channel effect (SCE) avoiding roll-off} (the threshold voltage is function of the channel length).
So, it is important to technologically control the doped regions, in terms of re-diffusion, and the SCE (maintaining the dependence of $V_{TH}$ on L limited to avoid roll-off).
Now we want to define how and why the doping level (N) changes when we scale a device. In order to keep SCE limited, the depletion width have to be reduced.

	\begin{figure}[H]
	\centering
	\includegraphics[width=0.7\textwidth]{09_depl_width}
	\caption{Depletion layer in a general device.}
	\label{}
	\end{figure}

The depletion layer visible in the picture above is a good approximation when $V_{DS}$ is equal to 0 V. So, the depletion width at source and drain is the same. Moreover, there is the depletion layer of the channel and its width is known when the system is in SI or near that region:

	\begin{equation}
	x_{dSI} \approx \sqrt{\frac{2 \epsilon_S 2 \Phi_P}{q N_A}}
	\label{}
	\end{equation}

The depletion width at the source junction (n\super{+} - p) can be estimated as:

	\begin{equation}
	x_{dSJ} = \sqrt{\frac{2 \epsilon_S \Phi_i}{q N_A}} 
	\label{}
	\end{equation}

where $\Phi_i \rightarrow \text{built-in voltage}$.
For an $n+p$ junction $2 \Phi_P \approx 0.7 V - 0.8 V$ and $2 \Phi_i \approx 0.7 V - 0.8 V$. In other words these two width can be confused and, as a consequence, $x_{d}$ and $x_{dSJ}$ can be confused as well. This is true only when $V_{DS} = 0$.

When $V_{DS} > 0$ what happens is that the depletion layer around the drain has a higher width than the depletion layer around the source.

	\begin{equation}
	x_{dDJ} = \sqrt{\frac{2 \epsilon_S (\Phi_i + V_{DD})}{q N_A}}
	\label{x_D}
	\end{equation}

As we said, in order to control and maintain limited the SCE it is necessary to scale the depletion width. But which depletion width? The worst depletion region is the one near the drain: $x_{dDJ} \gg x_{dSJ}$. This large depletion layer is completely controlled not by the gate voltage but by the drain voltage. The worst case is when  $V_{DS} = V_{DD}$. So we need to reduce $x_{dDJ}$: observing the equation (\ref{x_D}) we can say that $V_{DD}$ cannot be scaled (we are analyzing the constant voltage scaling), $\Phi_i$ depends on the logarithm of the concentration and its influence is not so strong in scaling, thus, the only way to reduce $x_{dDJ}$ is to increase $N_{A}$ like $K^{2}$.

	\begin{equation}
	x_{dDJ}' = \frac{x_{dDJ}}{K} \Rightarrow N_A' = K^2 \cdot N
	\label{x_D}
	\end{equation}

In constant voltage scaling, the doping level of the substrate is, node by node, increased in a non-linear way (strong variation of the doping level). The increase of the doping level has consequences on the threshold voltage:

	\begin{equation}
	V_{Th} = V_{FB}+2\Phi_P+\frac{1}{C_{ox}}\sqrt{2qN_A\epsilon_S 2\Phi_p}
	\end{equation}

$V_{FB}$ and $\Phi_P$ depend both on $N_{A}$ but they change in an opposite direction because: 

\begin{itemize}
\item $\Phi_P = kTln\frac{N_A}{n_i}$;
\item the expression of $V_{FB}$ contains the working function of the silicon that depends on $-\Phi_P$.
\end{itemize}

So these two terms partially compensate. Moreover, the variation of $\Phi_P$ is not so strong since it depends on the logarithm of $N_{A}$ thus, under the square-root, what prevails is $N_{A}$ that is scaled by $K^{2}$. This means that, overall, $V_{TH}$ should scale like K. But, since we want to maintain the ratio $\frac{V_{DD}}{V_{TH}}$ constant, $V_{TH}$ cannot scale. As a result, $C_{OX}$ must increase of a factor K. In this way $V_{TH}$ remains constant.

	\begin{equation}
	V_{TH}' \approx V_{TH} \Rightarrow C_{OX}' = K \cdot C_{OX}
	\label{}
	\end{equation}

How can we increase the oxide capacitance? Reducing the thickness of the oxide ($C_{OX} = \epsilon_{OX}/T_{OX}$):

	\begin{equation}
	T_{OX}' = \frac{1}{K} \cdot T_{OX}
	\label{}
	\end{equation}

This is the set of rules to implement the constant voltage scaling. Now we have to analyze what are the consequences of applying such rules on the electrical parameters.
Let us start from the depletion capacitance:

	\begin{equation}
	C_{dep}' = \sqrt{\frac{q\epsilon_S N_A'}{4\Phi_P}},\ N_A' = K^2 \cdot N \Rightarrow C_{dep}' = K \cdot C_{dep}
	\label{}
	\end{equation}

The `m' factor doesn't change:

	\begin{equation}
	m' = 1+\frac{C_{dep}'}{C_{OX}'} = 1+ \frac{KC_{dep}}{KC_{OX}} \Rightarrow m' = m
	\label{}
	\end{equation}

The average electric field along the channel ($E_{y}$) and the vertical one ($E_{x}$) will increase:

	\begin{equation}
	E_y' = \frac{V_{DD}}{L'} = \frac{V_{DD}}{\frac{1}{K}L} \Rightarrow E_y' = K \cdot E_y
	\label{}
	\end{equation}

	\begin{equation}
	E_x' = \frac{V_{DD}}{T_{OX}'} = \frac{V_{DD}}{\frac{1}{K}T_{OX}} \Rightarrow E_x' = K \cdot E_x
	\label{}
	\end{equation}

Both these results are not good. Now we try to understand why.
First of all, the expression of the electric field $E_y = \frac{V_{DD}}{L}$ is an acceptable approximation only when $V_{DS} \approx 0$ because in this case the behavior of the channel is resistive. In fact from Figure (\ref{ids}) we notice that the behavior of $I_{DS}$ is linear, for small values of $V_{DS}$.

	\begin{figure}[H]
	\centering
	\includegraphics[scale=0.4]{09_Ids_char}
	\caption{Drain current in function of $V_{DS}$.}
	\label{ids}
	\end{figure}

This means that when the device is in the linear region the channel is resistive. This implies that the electric field along the channel is roughly uniform. So, the trend of the channel potential for small values of $V_{DS}$ is:

	\begin{figure}[H]
	\centering
	\includegraphics[scale=0.4]{09_channel_pot}
	\caption{Channel potential for small $V_{DS}$ value.}
	\label{}
	\end{figure}

Since $\Phi_{ch}$ is linear the electric field can be simply evaluated as the ratio between $V_{DD}$ and L. This is not true for high values of $V_{DS}$ when the transistor is in saturation:

	\begin{figure}[H]
	\centering
	\includegraphics[scale=0.4]{09_channel_pot_2}
	\caption{For high values of $V_{DS}$ the depletion region around the drain is wider.}
	\label{}
	\end{figure}

The inversion charge ($Q_{n}$) is pinched-off (channel length modulation). The channel potential is approximately linear when $V_{DS} < V_{DSsat}$, while, when $V_{DS} > V_{DSsat}$, its trend strongly changes. So, the electric field assumes a lower value in the resistive part of the channel and a very high value in the region near the drain since the excess voltage with respect to the saturation voltage drops on a very short part of the channel. This causes the presence of \textbf{hot electrons}:

	\begin{figure}[H]
	\centering
	\includegraphics[scale=0.45]{09_hot_electr}
	\caption{Hot electron phenomenon.}
	\label{}
	\end{figure}

Electrons are travelling by mean of the electric field (drift current): near the source it is not so high so no problems arise; electrons are confined between two high energy barriers, one towards the oxide (3 eV) and one electrostatic barrier due to the depletion region; inside the channel there are collisions with the interface and with the lattice; the energy that electrons receive from the electric field, in this region, isn't high enough to make electrons jump the barriers in which they are confined. Near the drain electrons receive an overshoot and their speed is further increased. In addition, they receive such a high energy that they can overcome the barrier with the oxide and towards the substrate. Electrons become hot electrons and they cause different problems:

\begin{itemize}
\item current loss since some electrons don't reach the drain (gate current);
\item trapped charges in the oxide (ions in the oxide) that cause a change in $V_{FB}$ and, as a consequence, in $V_{TH}$.
\end{itemize}

If the threshold voltage shifts becoming too high or too low (it depends on the type of ions created in the oxide), we are no more able to either switch on or switch off the transistor. How we can limit the hot electrons phenomenon?

\begin{itemize}
\item Reducing $V_{DD}$;
\item Reducing $T_{OX}$ (with a very thin oxide the charge capture is strongly limited); this helps improving robustness;
\item Changing the width of the depletion region near the drain.
\end{itemize}

Let us focus on the last point. Now we consider only the channel-drain junction. It is reverse biased since $V_{DD}$ is applied on $n+$. The electric field varies linearly in the p region and has an abrupt trend in the $n+$ region. The problem is represented by $E_{max}$.

	\begin{figure}[H]
	\centering
	\includegraphics[scale=0.4]{09_LDD_proc}
	\caption{The behavior of the electric field with a standard process MOS and with a LDD process MOS.}
	\label{}
	\end{figure}

Reducing the doping level in the drain, the electric field varies linearly in both the regions. The built-in voltage must be kept constant, which means that the area of the triangular regions must be constant. The depletion region extends not only in the channel but also in the drain region. The result is that the maximum electric field is reduced. This solution is called \textbf{LDD (Lightly Doped Drain) technology}.
Starting from the middle of 80's all processes for CMOS technology became LDD: this means that the channel is no more out directly in contact with highly doped source/drain regions but it is necessary a buffer region characterized by a lower doping level.

 	\begin{figure}[H]
	\centering
	\includegraphics[scale=0.3]{09_LDD_device}
	\caption{A LDD device.}
	\label{}
	\end{figure}

The buffer regions are the \textbf{extension regions} that represent the true source and drain. Although the source region is not affected by the hot electrons problem, it has its extension region because it is not possible to distinguish the source contact from the drain one so the device is realized in a symmetrical way.

Now, the vertical electric field ($E_{x}$) represents a problem as well.

\begin{equation}
E_{x,\ MAX} = \frac{V_{DD}}{T_{OX}}
\end{equation}

$E_{x, MAX}$ is the maximum electric field that can be applied at the oxide and it is called \textbf{dielectric rigidity}. The electric field cannot overcome this value otherwise the oxide would be permanently damaged. The rigidity of $SiO_{2}$ is equal to 8 MV/cm. This value seems to be high but, actually, it is not because when we apply 5 V on few nanometers of oxide thickness it is very easy to reach this value. This a problem because, on the basis of what we said before, the oxide thickness has to be scaled by a factor K and, since $V_{DD}$ isn't scaled, it is very easy to overcome the $SiO_{2}$ rigidity. \\
Now we analyze the effect of constant voltage scaling on electrical parameters such as $I_{ON}$, $I_{off}$ end power density.
We can estimate how the \textbf{saturation current} for a long channel transistor varies applying CV scaling rules:

\begin{equation}
I_{DS,\ sat}' = \frac{\mu_n K C_{OX} W/K}{2 L/K}(V_{GS} - V_{TH})^2 = K \cdot I_{DS,\ sat}
\end{equation}

This equation tells us that we have a higher saturation current using a smaller transistor, as depicted in the following example:

	\begin{figure}[H]
	\centering
	\includegraphics[scale=0.5]{10_CVscaling_example}
	\caption{Drain-Source current for a scaled transistor.}
	\label{}
	\end{figure}

Smaller transistor means \textbf{smaller parasitics} and, thus, \textbf{faster technology} in addition to a higher saturation current. For these reasons, for many years, CV scaling has been strongly exploited: it guarantees higher performances.
Now let us see how $I_{ON}$ changes, while remembering that this parameter is measured considering \emph{W fixed} ($W_{ref}$ that is $1 \mu m$ according to the Roadmap).

	\begin{equation}
	I_{ON}' = \frac{\mu_n K C_{OX} W_{ref}}{2 L/K}(V_{GS} - V_{TH})^2 = K^2 \cdot I_{ON}
	\end{equation}

After this we can find out how $I_{subV_{th}}$ varies:

	\begin{equation}
	I_{subV_{th}}' = \frac{\mu_n K C_{dep} W/K}{L/K} V_{T}^2 \left( e^{\frac{V_{GS} - V_{TH}}{mV_T}} \right) = K \cdot I_{subV_{th}}
	\end{equation}

This is not a good result: the subthreshold current is subjected to an increase.
The same can be done for $I_{off}$ (this current, as $I_{ON}$, is evaluated considering $W_{ref}$):

	\begin{equation}
	I_{off}' = K^2 \cdot I_{off}
	\end{equation}

Now we can evaluate the dynamic power $P_{DYN}$ for one transistor. We recall that:

	\begin{equation}
	P_{DYN \ Tr} =f C_L V_{DD}^2
	\end{equation}

We need an estimation of the frequency f starting from the \textbf{intrinsic time $\tau$}. According to the Roadmap, $\tau$ can be evaluated in this way:

	\begin{figure}[H]
	\centering
	\includegraphics[scale=0.5]{10_tau_meas}
	\caption{Definition of the intrinsic delay.}
	\label{}
	\end{figure}

$C_{L}$ has the same value of the input capacitance of a transistor with the same characteristics of the one in the figure:

	\begin{equation}
	C_{L} = C_{OX} W_{ref} L
	\label{CL}
	\end{equation}

Transistor T1 is discharging the capacitance $C_{L}$ with its maximum current $I_{ON}$. $\tau$ can be written as:

	\begin{equation}
	\tau = \frac{C_L V_{DD}}{I_{ON}}
	\label{tau}
	\end{equation}

After scaling, $\tau$ will be (substituting (\ref{CL}) in (\ref{tau})):

	\begin{equation}
	\tau ' = \frac{K C_{OX} W_{ref} L/K V_{DD}}{K^2 I_{ON}} = \frac{1}{K^2} \cdot \tau
	\end{equation}

This represents a very good result: devices are K\super{2} times faster thanks to scaling.
The frequency can be expressed as $f = 1/\tau$. So:

	\begin{equation}
	f' = K^2 f
	\end{equation}

f' is higher compared to unscaled technology.
Now we can use $\tau'$ and f' to understand how the dynamic power changes. It is important to underline that we want to evaluate the dynamic power dissipated by a \emph{real transistor} characterized by a width equal to W, not $W_{ref}$, so, the expression of $C_{L}$, in this case, becomes $C_L = C_{OX} W L$ and after scaling

	\begin{equation}
	C_L ' = K C_{OX} W/K L/K = C_L/K
	\end{equation}

The \textbf{dynamic power} after scaling is:

	\begin{equation}
	P_{DYN \ Tr}' =  K^2 f C_L/K V_{DD}^2 = K \cdot P_{DYN \ Tr}
	\end{equation}

This a very bad result. Let us consider a scaling factor $K = 2$:

	\begin{figure}[H]
	\centering
	\includegraphics[scale=0.35]{10_power_density}
	\caption{}
	\label{}
	\end{figure}

After scaling we have 4 transistors that occupy the same area occupied by the unscaled device: so, same area but K\super{2} scaled devices. Each of them dissipate a power equal to $K \cdot P_{DYN \ Tr}$. If we evaluate the \textbf{power density per unit of area} what we get is:

	\begin{equation}
	P^{'den}_{DYN} = K^2 \cdot KP_{DYN}^{den} = K^3 \cdot P^{den}_{DYN}
	\end{equation}

This is the worst result that we have obtained until now; it has a strong influence on the circuit reliability since it is difficult to dissipate tens of W/cm\super{2}.
Regarding the \textbf{static power}, it can be expressed as $P_{STAT \ Tr} = I_{subV_{th}} V_{DD}$.

	\begin{equation}
	P_{STAT \ Tr}' = K I_{subV_{th}} V_{DD} = K \cdot P_{STAT \ Tr}
	\end{equation}

The \textbf{power density per unit of area} is:

	\begin{equation}
	P^{'den}_{STAT} = K^3 \cdot P^{den}_{STAT}
	\end{equation}

Both the static power per transistor and the static power density per unit of area increase like K and K\super{3} respectively. This is not a true problem since the level of leakages is low enough to guarantee an acceptable power dissipation.
Summarizing:

\begin{itemize}
\item \textbf{CV scaling pro}:
	\begin{itemize}
	\item faster devices.
	\end{itemize}

\item \textbf{CV scaling cons}:
	\begin{itemize}
	\item low reliability;
	\item unacceptable power consumption.
	\end{itemize}
\end{itemize}

Before moving on, it's important to analyse constant voltage scaling curves in the Figure (\ref{curves}):

	\begin{figure}[H]
	\centering
	\includegraphics[width=1.0\textwidth]{figscaVDD}
	\caption{Power supply and electric filed scaling trend.}
	\label{curves}
	\end{figure}

On the left side of the picture, the yellow line is representative of the CMOS technology golden age: for a very long period (from the 80s to the middle of the 90s)$ V_{DD}$ was 5 V. Then, it was substituted by a 3.3 $V_{DD}$, 2.5 V and so on.\\
The graph on the right side shows the evolution of the electric field: from the 80s to the middle of the 90s transistors were scaled but the supply voltage was kept constant; as a result, the average electric field in the channel increased like 1/year following the yellow curve trend.
It is fundamental to observe that the time period in which a power supply value is kept constant has become increasingly tight. Moving towards the most recent years $V_{DD}$ has been scaled year by year and this is true for HP, LOP and LSTP processes. Dimensions have been scaled as well so the electric field step becomes smaller, as the picture on the right show very well. This means that we are moving towards a constant field scaling.

\newpage

\item \textbf{Constant field scaling};

This policy involves the reduction of the transistor dimensions in order to maintain the electric field constant . As a consequence, if dimensions are scaled by a factor K, also $V_{DD}$ and $V_{th}$ has to be reduced like K.

  \begin{table}[H]
  {
  \begin{center}
   {
     \begin{tabular}{||l||c|c||}\hline
      V$_{DD}$ & -& $ {1/K}$\\
      \hline
      L,W,$x_J$       &- & $ {1/K}$\\
      \hline
      N & $x_d=\sqrt{\frac{2\epsilon_S(\Phi_i+V_{DD})}{qN}}$&$ {K}$ \\
      \hline
      V$_{DD}/V_{Th}$ &- & ${1}$\\
      \hline
      V$_{Th}$ & $V_{FB}+2\Phi_P+\frac{1}{C_{ox}}\sqrt{2qN_A\epsilon_S 2\Phi_p}$  & $ {1/K}$\\
      \hline
       $C_{ox}$&  $\epsilon_{ox}/T_{ox}$ &$ {K}^{+3/2}\approx K $\\ 
      \hline
      $T_{ox}$       &- & $ {1/{K}^{3/2}}\approx 1/K$\\
      \hline
       $C_{dep}$&   $\sqrt{\frac{q\epsilon_SN_A}{4\Phi_P}}$ &$K^{1/2}$\\
      \hline
       $ m-1 $&  $ \frac{C_{dep}}{C_{ox}}$ &$ {1/K^{1/2}} \approx 1 $\\ 
      \hline
      ${\cal E}_y$ & $V_{DD}/L$ & $ {1}$\\
      \hline
      ${\cal E}_x$ & $V_{DD}/Tox$ & $ {\approx 1}$\\
      \hline
       $I_{DSsat}$      &$\frac{\mu_n Cox W}{2L}(V_{GS}-V_T)^2$ &$ {1/K}$\\ 
      \hline  
       $I_{ON}$      &- &$ {1}$\\ 
      \hline
       $I_{subVt}$      &$\frac{\mu_n W \left(m-1 \right) C_{ox} {V_T}^2}{L}\;
            e^{\displaystyle \frac{{V_G-V_{Th}}}{ mV_T}}$ &$ K^{1/2} e^{\frac{V_{Th}}{mV_T}
               \left( \frac{K-1}{K}\right)}$\\ 
      \hline  
       $I_{off}$      &- & $ K^{3/2} e^{\frac{V_{Th}}{mV_T}
               \left( \frac{K-1}{K}\right)} $\\ 
      \hline
       $\tau$     &$Cox\cdot WL\cdot V_{DD}/Ion$ &$ {1/K}$\\ 
      \hline 
       $P_{Dyn}^{tr}$      &$fC_LV_{DD}^2$  &$ {1/K^{2}}$\\ 
      \hline 
       $P_{Dyn}^{Den}$      &-&$ {1}$\\ 
      \hline
       $P_{Isub}^{tr}$     &$I_{subV_{th}}V_{DD}$  &$ K^{-1/2} e^{\frac{V_{Th}}{mV_T}
               \left( \frac{K-1}{K}\right)} $\\ 
      \hline 
       $P_{Isub}^{Den}$      &-&$ K^{3/2} e^{\frac{V_{Th}}{mV_T}
               \left( \frac{K-1}{K}\right)} $\\ 
      \hline
     \end{tabular}
  }
\caption{CFS effects on main parameters}
    \end{center}
}
    \end{table}

We can notice that $E_{y}$ and $E_{x}$ are kept constant solving the problem of the presence of high energy electrons.
The next step is try to understand the consequences of the constant field scaling. Again, we want to maintain the ratio $V_{DD}/V_{TH}$ unscaled because, if, for instance, the starting process is the LOP one, after scaling the process must be same and in order to do this we have to keep constant the ratio. As a result $V_{TH}$ should be scaled like 1/K and this has a main downside: higher leakage current.

\underline{Example:}

	\begin{itemize}
	\item Before scaling: let us consider a technology with $V_{TH} = 200 mV$; in ideal conditions when $V_{GS} = 0$ the transistor is off; $V_{GS} - V_{TH} = -200 mV$;
	\item after scaling ($K = 2$): $V'_{TH} = V_{TH}/K = 100 mV$; $V'_{GS} - V'_{TH} = -100 mV$.
	\end{itemize}

	\begin{figure}[H]
	\centering
	\includegraphics[scale=0.45]{10_higher_leak_current}
	\caption{Subthreshold current for two different devices.}
	\label{}
	\end{figure}

In the unscaled situation the subthreshold current is $I_{subV_{th}}$ while in the scaled case the leakage current is $I'_{subVT}$ which is quite higher than $I_{subV_{th}}$. In the last case it becomes difficult to switch off the transistor. This is the problem of the CF scaling.
Now we can analyse how the technological parameters change applying the CF scaling.

\textbf{Doping level}: we know that, in order to limit the SCE, the drain depletion width has to be reduced.

	\begin{equation}
	x_d=\sqrt{\frac{2\epsilon_S(\Phi_i+V_{DD})}{qN}}
	\end{equation}

Since $V_{DD}$ is scaled by a factor K, it is sufficient to increase N by the same factor (this is an approximation) to reduce $x_{d}$ of a quantity equal to K. So:

	\begin{equation}
	N' = K \cdot N
	\end{equation}

Differently from CV scaling, in this case is sufficient a smaller increase of the substrate doping level.

\textbf{Threshold voltage}: we want to maintain $V_{DD}/V_{TH}$ constant; since $V_{DD}$ is scaled like 1/K it is necessary to scale $V_{TH}$ like 1/K as well. For this purpose, considering that $N_{A}$ increases like K, $C_{OX}$ has to be increased like $K^{3/2}$.

	\begin{equation}
	V'_{TH} = V_{FB} + 2\Phi_P + \frac{1}{K^{3/2} C_{OX}} \sqrt{2q \epsilon_S K N_A 2\Phi_P}
	\end{equation}
	
	\begin{equation}
	\frac{V'_{TH}}{V_{TH}} \approx \frac{K^{1/2}} {K^{3/2}} = \frac{1}{K}
	\end{equation}

As we said the \textbf{gate oxide capacitance} becomes:

	\begin{equation}
	C'_{OX} = K^{3/2} \cdot C_{OX}  \approx K \cdot C_{OX}
	\label{cox}
	\end{equation}

As a consequence we get the \textbf{oxide thickness}:

	\begin{equation}
	C'_{OX} = \frac{\epsilon_{OX}}{T'_{OX}} \Rightarrow T'_{OX} = \frac{\epsilon_{OX}}{K^{3/2} \cdot C_{OX}} \approx \frac{1}{K} \cdot T_{OX}
	\label{tox}
	\end{equation}

In the last two equations ((\ref{cox}) and (\ref{tox})) $K^{3/2}$ has been approximated as K to reduce the complexity of the expressions, on one hand; on the other hand, the oxide thickness is already very thin: scaling it like 1/K is a hard task, scaling it like $1/K^{3/2}$ is even harder and, in many cases, it is impossible. So, it is preferable to reduce $T_{OX}$ of a factor K and this, clearly, has a consequence: $V_{TH}$ will not scale exactly as expected but this is not a problem.

\textbf{Depletion capacitance}: let's consider the following equation, with some parameters already scaled:

	\begin{equation}
	C'_{dep} = \sqrt{\frac{q \epsilon_S K N_A}{4 \Phi_P}} = K^{1/2} \cdot C_{dep} 
	\label{}
	\end{equation}

And the same happen when we consider the ratio $\frac{C'_{dep}}{C'_{OX}}$, that is equal to $(m - 1)$:

	\begin{equation}
	\frac{C'_{dep}}{C'_{OX}} = \frac{1}{K^{1/2}} \approx 1
	\label{}
	\end{equation}

Now we'll analyse the electrical parameters.
\textbf{Saturation current}: it's evaluated for $V_{GS} = V_{DD}$, the maximum voltage applicable to a transistor.

	\begin{equation}
	I_{DS,\ sat}' = \frac{\mu_n K C_{OX} W/K}{2 L/K}\frac{(V_{DD} - V_{TH})^2}{K^2} = \frac{1}{K} \cdot I_{DS,\ sat}
	\label{}
	\end{equation}

The saturation current is reduced so, the effect on the reliability of the device is improved. For what concern $I_{ON}$ current:

	\begin{equation}
	I_{ON}' = \frac{\mu_n K C_{OX} W_{ref}}{2 L/K}\frac{(V_{DD} - V_{TH})^2}{K^2} = 1 \cdot I_{ON}
	\end{equation}

\textbf{Instrinsic time}:

	\begin{equation}
	\tau ' = \frac{K C_{OX} W_{ref} L/K V_{DD}/K}{I'_{ON}} = \frac{1}{K} \cdot \tau
	\end{equation}

With respect to the CV scaling, now the performance are less good (devices are just K times faster, not K\super{2}).

\textbf{Dynamic power per transistor}:

	\begin{equation}
	P'_{DYN \ Tr} =f' C'_L V_{DD}^{'2}
	\label{din}
	\end{equation}

where:

	\begin{align}
	&f' = \frac{1}{\tau'} = K \cdot f \\
	&C_L ' = K C_{OX} W/K L/K = C_L/K
	\end{align}

So $P'_{DYN} \ Tr$ of equation (\ref{din}) becomes:

	\begin{equation}
	P_{DYN \ Tr}' =  K f C_L/K \frac{V_{DD}^2}{K^2} = \frac{P_{DYN \ Tr}}{K^2}
	\end{equation}

\textbf{Dynamic power density per unit area}:

	\begin{equation}
	P^{'den}_{DYN} = K^2 \cdot \frac{P^{den}_{DYN}}{K^2} = 1 \cdot P^{den}_{DYN}
	\end{equation}

These last two expressions represent an excellent result: the dynamic power consumption per transistor is reduced by $1/K^{2}$ (while in the CV scaling it increases like K) and the dynamic power density remains constant (while in the CV scaling it increases like $K^{3}$).
The subthreshold current is evaluated with $V_{GS} = 0$:

	\begin{eqnarray*}
	I_{subV_{th}} & = &\frac{\mu_n W C_{dep} {V_T}^2}{L}\;e^{\displaystyle \frac{ {V_G-V_{Th}}}{ mV_T}}
	\end{eqnarray*}

After scaling, and approximating m' with m, it becomes:

	\begin{eqnarray*}
	I_{subV_{th}}^{\prime} & = &\frac{K \mu_n W K^{1/2} C_{dep} {V_T}^2}{K L}\;e^{\displaystyle \frac{ {V_G-(V_{Th}/K)}}{ m^{\prime}V_T}}
	\end{eqnarray*}

The ratio between $I'_{subVT}$ and $I_{subV_{th}}$ is:

	\begin{eqnarray*}
	\frac{I_{subV_{th}}^{\prime}}{I_{subV_{th}}} & =& K^{1/2} e^{\frac{V_{Th}}{mV_T}\left( \frac{K-1}{K}\right)}
	\end{eqnarray*}

The ratio increases exponentially with K:

	\begin{figure}[H]
	\centering
	\includegraphics[scale=0.5]{11_I'sub_Isub_graph}
	\caption{Evolution of subthreshold current with scaling factor K.}
	\label{}
	\end{figure}

Summarizing:

\begin{itemize}
\item Constant field scaling pros:
	\begin{itemize}
	\item limited electric field;
	\item limited dynamic power.
	\end{itemize}
\item Constant field scaling cons:
	\begin{itemize}
	\item high subthreshold leakages (they increase exponentially with scaling).
	\end{itemize}
\end{itemize}

It is not possible to scale $V_{DD}$ of the same scaling factor that we use for dimensions at each technological node since, doing in such a way, the power supply now should be equal to few millivolts and this, clearly, is not possible. In other words, we cannot apply only the constant field scaling because we would need a very small power supply and, of course, a very low threshold voltage. This solution can't be operated for transistors, mainly for two reasons: high leakages and noise (few mV of noise signal could erroneously switch on/off a transistor). On the other hand we cannot apply only constant voltage field because of the low reliability. We have to mix these two policies creating the \textbf{Generalized scaling}.

\newpage

\item \textbf{Generalized scaling}.

In generalized scaling are used two different parameters:

	\begin{enumerate}
	\item K for dimensions (W, L, $x_{j}$);
	\item $\alpha$ for the others, that is: $1< \alpha < K$.
	\end{enumerate}

How do we scale $V_{DD}$? As we said, scaling $V_{DD}$ by a factor K is too much, so, in generalized scaling, we correct it using the factor $\alpha$:

	\begin{equation}
	V'_{DD} = V_{DD} \cdot \frac{\alpha}{K}
	\end{equation}

$\alpha$ can be used as a parameter to measure the type of scaling: varying it, we modify the kind of scaling.

	\begin{figure}[H]
	\centering
	\includegraphics[scale=0.4]{11_alpha}
	\caption{Range of variation for $\alpha$ parameter.}
	\label{}
	\end{figure}

In the table below is summarized how technological and electrical parameters change applying the generalized scaling:

\begin{table}[H]
{
   \begin{center}
  {
     \begin{tabular}{||l||c|c||}\hline
      V$_{DD}$ & -& $ {\alpha/K}$\\
      \hline
      L,W,$x_J$       &- & $ {1/K}$\\
      \hline
      N & $x_d=\sqrt{\frac{2\epsilon_S(\Phi_i-V_{DD})}{qN}}$&$ {\alpha K}$ \\
      \hline
      V$_{DD}/V_{Th}$ &- & ${1}$\\
      \hline
      V$_{Th}$ & $V_{FB}+2\Phi_P+\frac{1}{C_{ox}}\sqrt{2qN_A\epsilon_S 2\Phi_p}$  & $ {\alpha/K}$\\
      \hline
       $C_{ox}$&  $\epsilon_{ox}/T_{ox}$ &$ {K}^{+3/2}/\alpha^{1/2}\approx K $\\ 
      \hline
      $T_{ox}$       &- & $ {\alpha^{1/2}/{K}^{3/2}}\approx 1/K$\\
      \hline
       $C_{dep}$&  $\sqrt{\frac{q\epsilon_SN_A}{4\Phi_P}}$ &$ {(\alpha K)}^{1/2}$\\ 
      \hline
      ${\cal E}_y$ & $V_{DD}/L$ & $ {\alpha}$\\
      \hline
      ${\cal E}_x$ & $V_{DD}/Tox$ & $ {\approx \alpha}$\\
      \hline
     \end{tabular}
}
\caption{Generalized scaling effects on main parameters}
    \end{center}
}
\end{table}

Now it should be clear how to deduce all these rules.

\end{enumerate}

\newpage

\subsection{Interconnect scaling}
Although we have mostly talked about device scaling and its effects so far, as technology continues to scale the \textbf{most critical part} of an IC in terms of power and delay is represented by the \textbf{interconnects}. As matter of fact, the \textbf{limitations on interconnect scaling are due to performance} rather than technological processes. On an IC there are three main types of interconnects: 

\begin{itemize}
\item Interconnects for power supply (i.e. $V_{dd}$ and ground).
\item Interconnects for timing signals (i.e. clock tree).
\item Interconnects for logic signals, which can be classified as local, intermediate and global.
\end{itemize}

Power supply interconnects suffer the most from electro-migration phenomena among all types of interconnects, since the current always flows into the same direction and it can causes \textit{hillocks} or \textit{depletions} according to ion flux. Another problem that affects power supply is \textit{IR-drop}, which leads to an effective value of $V_{dd}$ or ground different from the nominal one.\\ Clock tree is another critical part of the circuit because of the unbalanced distribution over the IC area of the sequential elements, i.e. registers, flip-flops, and so on. That leads to an uneven load on the different branches of the tree and it can cause a worse clock skew. \\ Finally, logic signal interconnects are affected by different delays depending on their position in the interconnect \textbf{hierarchy}. Our attention will be focused on this last kind of interconnects, for which there will be explanation about the different approaches to their scaling. \\However, we will at first describe which are the electrical quantities that play a key role in interconnect scaling.

\subsection{Electrical and geometrical parameters for interconnects}
As we can observe from the Figure (\ref{figint}) below, there are different geometrical quantities involved:

	\begin{figure}[H]
	\centering
	\includegraphics[width=0.6\textwidth]{interconnection}
	\caption{Geometrical quantities for interconnections.}
	\label{figint}
	\end{figure} 

\begin{itemize}
\item $W_{int}$: cross-section width;
\item $H_{int}$: cross-section height;
\item $L_{int}$: longitudinal lenght of the interconnection;
\item $W_{sp}$: distance between two wires belonging to the same matel layer;
\item $T_{ILD}$: distance between two consecutive metal layers;
\item $A/R$: it is the aspct ratio; it relates the vertical and the horizontal dimension of a wire and it is definied as $A/R=H_{int}/W_{int}$.
\end{itemize}

Other important parameters that determine the good behavior of interconnects are the dielectric constants $\epsilon_{ILD}$ (for Inter-Layer Dielectric) and $\epsilon_{IMD}$ (for Inter-Metal Dielectric). The former must have good thermal properties to dissipate heat in an efficient way from layer to layer, the latter must be very low in order to reduce \textit{cross-talk capacitance}.\\ Finally, the three main electrical quantities involved are indicated in Figure (\ref{figint2}):

\begin{itemize}
\item $R_{int}$=$\rho(\frac{L_{int}}{W_{int}H_{int}})$, where $\rho$ is the material resistivity;
\item $C_{IMD}$=$\epsilon_{IMD}(\frac{L_{int}H_{int}}{W_{sp}})$;
\item $C_{ILD}$=$\epsilon_{ILD}(\frac{L_{int}W_{int}}{T_{int}})$.
\end{itemize}

	\begin{figure}[H]
	\centering
	\includegraphics[width=0.5\textwidth]{interconnection2}
	\caption{Interconnects parameters.}
	\label{figint2}
	\end{figure}

It would be much simpler to define and compare the parameters above independently from the length of each wire. Therefore, we can define:
\begin{itemize}
\item $\mathcal{R}_{int}$=$\rho(\frac{1}{W_{int}H_{int}})$, where $\rho$ is the material resistivity;
\item $\mathcal{C_{IMD}}$=$\epsilon_{IMD}(\frac{H_{int}}{W_{sp}})$;
\item $\mathcal{C_{ILD}}$=$\epsilon_{ILD}(\frac{W_{int}}{T_{int}})$.
\end{itemize}

\subsection{Scaling policies for interconnects and their effects}
As it has been said in the main introduction about scaling, there are three different policies that can be followed:
\begin{itemize}
\item ideal scaling: all dimensions, both vertical and horizontal ones, are scaled using a factor S (constant aspect ratio);
\item quasi-ideal scaling: horizontal quantities are scaled using S, while horizontal ones are scaled by $\sqrt(S)$ (aspect ratio slightly increased); 
\item fixed-height scaling: only horizontal quantities are scaled by S, whereas the vertical dimension is \textbf{not scaled} at all (aspect ratio largely increased). 
\end{itemize}

	\begin{figure}[H]
	\centering
	\includegraphics[width=0.85\textwidth]{interconnection3}
	\caption{Interconnects Scaling strategies.}
	\label{figint2}
	\end{figure}

\begin{table}[H] {
\begin{center} {
\begin{tabular}{||c|c|c|c||}
\hline
&Ideal&Quasi-ideal&Fixed-Height\\
\hline
$W_{int}$, $W_{sp}$ & $S^{-1}$ & $S^{-1}$ & $S^{-1}$\\
\hline
$H_{int}$, $T_{ILD}$ & $S^{-1}$ & $S^{-1/2}$ & 1\\
\hline
Aspect Ratio & 1 & $S^{-1/2}$ & S\\
\hline
$\mathcal{R}_{int}$ & $S^{2}$ & $S^{3/2}$ & S\\
\hline
$\mathcal{C_{IMD}}$ & 1 & $S^{1/2}$ & S\\
\hline
$\mathcal{C_{ILD}}$ & 1 & $S^{-1/2}$ & $S^{-1}$\\
\hline
J & $S^{2}$ & $S^{3/2}$ & S\\
\hline
$\mathcal{R}_{int}*\mathcal{C_{IMD}}$ & $S^{2}$ & $S^{2}$ & $S^{2}$\\
\hline
$\mathcal{R}_{int}*\mathcal{C_{ILD}}$ & $S^{2}$ & S & 1\\
\hline
\end{tabular} }
\caption{Scaling policies for interconnects}
\end{center} }
\end{table}

Now that we have seen the main geometrical and electrical parameters and how they vary when scaled, we can analyze the different effects of scaling on local, intermediate and global wires.\\ Firstly, a distinction must be made: the scaling factor \textbf{S will be kept for local wires}, whereas another factor \textbf{$S_{G}$ will be introduced for global interconnects}. Secondly, it must be taken into account that as scaling proceeds the dimensions of the die will increase according to Moore's Law. Therefore, the length of the global wires will increase as they will need to reach parts of the IC that are further apart. As a result, the delay of local and global interconnects will behave differently.\\ Let us look at the following table:

\begin{table}[H] {
\begin{center} {
\begin{tabular}{||c|c|c|c||}
\hline
&Ideal&Quasi-ideal&Fixed-Height\\
\hline
Local length \textbf{$L_{L}$}&$S^{-1}$&$S^{-1}$&$S^{-1}$\\
\hline
Global lenght \textbf{$L_{G}$}&$S_{G}$&$S_{G}$&$S_{G}$\\
\hline
Local Delay (IMD) \textbf{$\mathcal{R}_{int}*\mathcal{C_{IMD}}*L_L^2$}&1&1&1\\
\hline
Local Delay (ILD) \textbf{$\mathcal{R}_{int}*\mathcal{C_{ILD}}*L_L^2$}&1&$S^{-1}$&$S^{-2}$\\
\hline
Global Delay (IMD) \textbf{$\mathcal{R}_{int}*\mathcal{C_{IMD}}*L_G^2$}&$S^{2}S_G^2$&$S^{2}S_G^2$&$S^{2}S_G^2$\\
\hline
Global Delay (ILD) \textbf{$\mathcal{R}_{int}*\mathcal{C_{ILD}}*L_G^2$}&$S^{2}S_G^2$&$S S_G^2$&$S_G^2$\\
\hline
\end{tabular} }
\caption{Effects of the different scaling policies on Delay}
\end{center} }
\end{table}

The results in the last two tables show us that the \textbf{fixed-height} scaling for \textbf{local} wires offers the outcome in terms of \textit{current density} and \textit{delay} even though the higher aspect ratio increases the complexity of the process. 
At the same time, the delay for global wires grows excessively and it is \textbf{not affordable} at system level. Consequently, a different approach has become necessary and the \textbf{hierarchical scaling} policing for interconnects has been adopted. We define three factors: $S_{L}$, $S_{I}$ and $S_{G}$ for local, intermediate and global wires respectively. The magnitude of the three different parameters will be $S_{L}>S_{I}>S_{G}$:

\begin{table}[H] {
\begin{center} {
\begin{tabular}{||c|c|c|c||}
\hline
&Local&Intermediate&Global\\
\hline
$W_{int}$, $W_{sp}$&1/$S_{L}$&1/$S_{I}$&$1/S_{G} \simeq 1$\\
\hline
$H_{int}$, $T_{ILD}$&$\simeq 1$&$\simeq 1$&$\simeq 1$\\
\hline
\end{tabular} }
\caption{Hierarchical scaling factors}
\end{center} }
\end{table}

The Fixed-Height policy has been applied to all hierarchical levels of interconnects, whereas horizontal quantities scale according to their own scaling factor. For global wires $S_{G} \simeq 1$, which means that basically they do not scale. Local wires scale like $S_{L} \simeq K$, thus they almost behave the same way as devices do. Finally, intermediate wires gradually range from local to global or vice-versa.

\newpage

\section{Short Channel Effects}

\textbf{Short Channel Effects (SCE)} is a general term that includes all the consequences and the effects that are visible in deeply-scaled devices. For scaling continues with years, SCE are getting more and more evident and difficult to deal with. As we are entering the age of nanoscale FET devices, where the channel length will be a few nanometers only, SCE are leading to the end of the Bulk planar process and into a new era of different solutions ranging from finFETs to Gate-All-Around (GAA) nanowire transistors. 

\subsection{Roll-off} \label{subsec:roll_off}

For a long channel device the threshold voltage can be written as:

	\begin{equation}
	V_{TH,\ LC} = V_{FB} + 2\Phi_P + \frac{1}{C_{OX}} \sqrt{2q \epsilon_s N_A 2 \Phi_P} = V_{FB} + 2\Phi_P - \frac{Q_d}{C_{OX}}
	\end{equation}
	$[Q_{d}] = C/cm^2$ ($Q_{d}$ is a charge density)\\
	$[C_{OX}] = F/cm^2$

We can rewrite the ratio between the depletion charge and the oxide capacitance multiplying and dividing it by $W \cdot L$; in this way we are considering the effective charge and capacitance, not their densities:

	\begin{equation}
	V_{TH,\ LC} = V_{FB} + 2\Phi_P - \frac{Q_d \cdot W \cdot L}{C_{OX} \cdot W \cdot L}
	\end{equation}

In a long channel device the quantity $W \cdot L$ is the same at the numerator and at the denominator, so it can be simplified; in addition, considering that $Q_d = -q N_A x_d$, we can simply write:

	\begin{equation}
	V_{TH,\ LC} = V_{FB} + 2\Phi_P + \frac{q N_A x_d}{C_{OX}}
	\end{equation}

But, in a short channel device these semplifications are no more valid:

	\begin{figure}[H]
	\centering
	\includegraphics[width=0.75\textwidth]{figsce}
	\caption{A short channel LDD MOS.}
	\label{figsce}
	\end{figure}

The oxide capacitance does not change: $C_{OX} \cdot W \cdot L$. What varies is the depletion capacitance because, now, the geometry of the depleted region under the channel is changed.

The total depletion charge becomes:

\begin{equation}
Q_d = -q N_A x_d W \left( \frac{L + L - 2x_1}{2} \right)
\end{equation}

We want to evaluate $x_{1}$ making two assumptions:

	\begin{itemize}
	\item S/D junctions have a circular profile;
	\item the depletion region beside S/D junctions has the same width of the depletion region under the channel. Is this assumption reasonable?
	
	The voltage drop accross the depleted region is:
	
	\begin{equation}
	V = \frac{q N_A}{2 \epsilon_s} x_d^2
	\end{equation}
	
	In strong inversion:
		\begin{itemize}
		\item for the channel $V = 2\Phi_P \approx 0.8 V \Rightarrow x_{d,\ CH} = \sqrt{\frac{2 \Phi_P 2 \epsilon_s}{q N_A}}$;
		\item for source and drain $V = \Phi_i \approx 0.8 V \Rightarrow x_{d,\ S/D} = \sqrt{\frac{\Phi_i 2 \epsilon_s}{q N_A}}$.
		\end{itemize}
	
	Since $2\Phi_P$ and $\Phi_i$ are approximately equal we can say that $x_{d,\ CH} \approx x_{d,\ S/D}$. Thus, the assumption is correct.
	\end{itemize}

Now, considering figure \ref{figsce}, we can write:

	\begin{eqnarray*}
	 \left(x_1+ x_J\right)^2 + {x_d}^2 &=& \left(x_J+ x_d \right)^2
	\end{eqnarray*}

Developing the square terms we obtain:

	\begin{eqnarray*}
	   x_1&=& x_J \left(\sqrt{1+\frac{2x_d}{x_J}} -1 \right)
	\end{eqnarray*}

Substituting this expression in the threshold voltage expression for a short channel device we get:

	\begin{eqnarray*}
	   V_{ThLC}&=& V_{FB}+2\Phi_P-\frac{Q_d\;WL}{C_{ox}WL}\\
	           &=& V_{FB}+2\Phi_P+\frac{qN_A\;WL x_d}{C_{ox}WL}\\
	           &=& V_{FB}+2\Phi_P+\frac{qN_A x_d}{C_{ox}}
	\end{eqnarray*}

The quantity $- \frac{q N_A x_d x_1}{C_{OX}L}$ is a \textbf{correction factor}.

We said that the expression of the threshold voltage for a short channel device is:

	\begin{eqnarray*}
	   V_{ThSC}&=& V_{FB}+2\Phi_P+\frac{qN_AW\left(L+L-2x_1\right)x_d}{2 C_{ox}WL}\\
	           &=& V_{ThLC} - \frac{qN_A x_d}{C_{ox}}\frac{x_1}{L}
	\end{eqnarray*}

The second term on the right-hand side of the equation represents the contribution given by the SCE and it causes a threshold voltage lowering. When we scale a device, the ratio $\frac{x_1}{L}$ should be kept constant. If it doesn't remain constant, which means that L is scaled whereas $x_{1}$ (which is half the portion of the depletion charge \textbf{not} controlled by the gate) is not, $\frac{x_1}{L}$ increases causing a further reduction of the threshold voltage. Thus, in order to keep limited the $V_{th}$ reduction we have to scale $x_{1}$. Unfortunately, $x_{1}$ is not a technological parameter.

	\begin{eqnarray*}
	   x_1&=& x_J \left(\sqrt{1+\frac{2x_d}{x_J}} -1 \right)
	\end{eqnarray*}

We can reduce $x_{1}$ scaling $x_{j}$ by the same factor of the channel length (that is the reason why it is essential to scale S/D junctions like the channel). However, by looking at the expression of $x_{1}$, we notice that if we want to reduce it by a factor K, we have to \textbf{scale both the depletion width and the junction depth}. Scaling the depletion width can be realized by increasing the doping level by a factor $K^2$ (because of the square-root behavior in the formula for $x_{d}$); unfortunately, there is a drawback about increasing the doping level, since it worsens the electric field peak at the drain terminal and it turns into degraded $m$ and $\mu$ values. Instead, the junction depth can be scaled during the junction-implementation steps.

	\begin{equation}
	x'_d = \frac{1}{K} x_d\ \text{and}\ x'_j = \frac{1}{K} x_j
	\end{equation}

So:

	\begin{equation}
	x'_1 = \frac{1}{K} x_j \left( \sqrt{1+ 2\frac{1/K \cdot x_d}{1/K \cdot x_j}} - 1 \right) = \frac{1}{K} \cdot x_1
	\end{equation}

Scaling $x_{d}$ and $x_{j}$ we are sure that the effect of the SCE on $V_{th}$ is limited. If we do not respect this rule the consequence is \textbf{roll-off}. Roll-off means that the threshold voltage is dominated by the SCE and it becomes a function of the channel length, whereas designers would like to have a $V_{th}$ invariant with transistor dimensions and biasing conditions. This effect is exacerbated by $V_{DS}$, that is, roll-off gets worse as $V_{DS}$ increases to $V_{dd}$. Roll-off becomes significant for channels that are shorter than 100 nm. Since there is no way to make sure that all transistors would have a channel length equal to the nominal value, even a small variation in the process (due to poor photolitography resolution) can determine large changes in the threshold voltage when the curve is steep. It is important to choose the optimal length in such a way that, if its value changes, the roll-off region is still avoided and the threshold voltage remains around the nominal value.

	\begin{figure}[H]
	\centering
	\includegraphics[width=0.5\textwidth]{figrolloff}
	\caption{Threshold voltage dependency from roll-off effect.}
	\label{}
	\end{figure}

\subsection{Drain-Induced Barrier Lowering (DIBL) effect}

As technology scales and the channel gets shorter, source and drain regions get closer to each other. When they become sufficiently close, an electrostatic coupling occurs between source and drain. When $V_{DS}=V_{dd}$ the energy barrier that prevents the flow of carriers through the channel becomes very low. At this point, even a small value of the gate-source voltage, with $0 < V_{GS} < V_{th}$, can completely flatten the energy barrier and allow a free flow of carriers in the channel. DIBL worsens as the doping level of the extensions increases, whereas it gets less dramatic by using LDD structures. As a matter of fact, low doping level at the extensions allows to absorb part of the drain potential, so that the energy barrier in the channel is not lowered as much as it would be with high doping.

\begin{equation}
\Delta V_{th,DIBL} = V_{th,V_{DS} \simeq 0} - V_{th,V_{DS} \simeq V_{DD}}
\end{equation}

DIBL is expressed in terms of $mV/V$ as we want it independent from power supply voltage values when comparing different technological nodes. For deeply scaled devices DIBL can account up to 50\% of the threshold voltage. It is obvious that DIBL can be a critical aspect in modern devices.

	\begin{figure}[H]
	\centering
	\includegraphics[width=0.8\textwidth]{dibl_band_diagram}
	\caption{We can appreciate how short-channel devices are sensitive to even small values of $V_{GS}$ for triggering a free flow of electrons.}
	\label{}
	\end{figure}

Finally, another phomenon that must be avoided is \textbf{punchthrough}, which is showed in Fig. (\ref{fig:punch}). Punchthrough consists of the union of the source and drain depletion regions on account of the enlargement towards the source terminal of the depletion region at drain. If the channel between source and drain is analyzed as the base of a parasitic BJT structure\footnote{In fact, a channel length smaller than $1 \mu m$ is a typical base width for a BJT transistor.}, a full depletion will lead to an \textbf{uncontrolled current}. This is another problem that can be solved by increasing the level of doping at drain (since it reduces the depletion width), even though it causes some drawbacks as it is explained in the \ref{subsec:roll_off}.\\
	
From the formula

\begin{equation}
x_{d, n^{+}-p} = \sqrt{\frac{2 \epsilon_s (\Phi_i + V_D)}{q N_A}}
\end{equation}

it is straight-forward to see that the depletion width $x_{d, n^{+}-p}$ decreases as the doping level $N_A$ increases.

	\begin{figure}[H]
	\centering
	\includegraphics[width=0.6\textwidth]{punchthrough}
	\caption{Punchthrough. As $V_D$ increases, source and drain depletion regions tend to merge together.}
	\label{fig:punch}
	\end{figure}


\chapter{Scaling on MultiGate process} \label{multigate}

\section{Analytical approach to MultiGate structure}
Let us now broach a new and more interesting topic about \textbf{scaling of MultiGate devices}. As stated in the previous sections, non-planar processes are the future of technological evolution.\\
The kind of device that lead us into the MG era is the \emph{Silicon-On-Insulator (SOI)} transistor. Although it has already been replaced and taken over by its successors, i.e. FinFETs, vertical and lateral Gate-All-Around (GAA) transistors etc\dots, it has allowed us to think of the \textbf{buried oxide (BOX)} as a secondary gate that is useful to tune the device or even behave as a main gate, for example in FinFET devices. 

	\begin{figure}[H]
	\centering
	\includegraphics[width=0.6\textwidth]{banddiagram}
	\caption{Band diagram for a double gate structure.}
	\label{band}
	\end{figure}

By integrating the charge distribution we obtain the surface potential $\Phi(x)$, that can be expressed at the interface gate-oxide with the name $\Phi_{S1}$ and at the interface BOX-substrate with the name $\Phi_{S2}$:

\begin{eqnarray}
\Phi(x) = \Phi_{S1} - \int_{0}^{x}\frac{q N_A}{\epsilon_s}(x_1 - x)\,dx \\
		= \Phi_{S1} - \frac{q N_A}{\epsilon_s}(x_1 x) + \frac{q N_A}{2\epsilon_s}x^2 \\
\Phi(T_{Si}) = \Phi_{S2} = \Phi_{S1} - \frac{q N_A}{\epsilon_s}(x_1 T_{Si}) + \frac{q N_A}{2\epsilon_s}T_{Si}^2
\end{eqnarray}

Then, \emph{$x_{1}$} can be obtained as a function of surface potentials $\Phi_{S1}$ and $\Phi_{S2}$:

\begin{eqnarray}
\Phi_{S2} - \Phi_{S1} = \frac{q N_A}{\epsilon_s}(x_1 T_{Si}) + \frac{q N_A}{2\epsilon_s}T_{Si}^2 \\
\emph{$x_{1}$} = (\Phi_{S1} - \Phi_{S2}) \frac{C_{Si}}{q N_A} + \frac{T_{Si}}{2}
\end{eqnarray}

It is clear how the position of \emph{$x_{1}$} depends on the surface potential at both gates, as shown in Figure (\ref{band}) and in the equation above. In particular if $\Phi_{S1}$ is equal to $\Phi_{S2}$ then \emph{$x_{1}$} will be centered in $\frac{T_{Si}}{2}$; when $\Phi_{S1}$ is greater than $\Phi_{S2}$ then \emph{$x_{1}$} will move to the right. Finally, when $\Phi_{S1}$ is smaller than $\Phi_{S2}$ then \emph{$x_{1}$}will move to the left. It means that the greatest surface potential is the one that controls the most the behavior inside the silicon overlay.
Knowing the two surface potential, in order to derive the threshold voltage $V_{Th}$, we have to take into account all possible condition for both front-gate and back-gate:

\begin{enumerate}
\item Front-gate inverted $Q_{n1} = 0$, $\Phi_{S1} = 2\Phi_P$ \footnote{This condition tell us that the system is in strong inversion.};
\item Back-gate accumulated $\Phi_{S2} = 0$;
\item Front-gate inverted $Q_{n1} = 0$, $\Phi_{S1} = 2\Phi_P$;
\item Back-gate inverted $\Phi_{S2} = 2\Phi_P$ (i.e. in strong inversion);
\end{enumerate}

Let's assume the first condition. $V_{Th,1}$ depends on the surface potential $\Phi_{S2}$, in particular if:
\begin{itemize}
\item the back-gate is accumulated, so the expression will be:
	\begin{equation}
	V_{Th1,acc2} = -\Phi_{MS1} + (1 + \frac{C_{Si}}{C_{ox1}})2\Phi_P + \frac{q N_A T_{Si}}{2 C_{ox1}}
	\label{eq1}
	\end{equation}
\item the back-gate is in strong inversion, so:
	\begin{equation}
	V_{Th1,dep2} = -\Phi_{MS1} + 2\Phi_P + \frac{q N_A T_{Si}}{2 C_{ox1}}
	\end{equation}
\end{itemize}

As we can observe the value of $V_{Th,1}$ is larger when the back-gate is accumulated (eq.(\ref{eq1})) due to a significant value of silicon-overlay capacitance, whereas the threshold voltage is lower when the back-gate is in strong inversion, therefore, the device has a better on/off behavior.
In the same manner, considering the back-gate and using $\Phi_{S2}$ as silicon potential, we are able to derive the two expressions of the threshold voltage $V_{Th,2}$ that will depend on the surface potential $\Phi_{S1}$ according to the possible state it can assume.\\
To complete the analysis of the multi-gate device, it is mandatory to show how the source-drain current can be derived. Starting from the expression of the surface potentials, and choosing a particular condition that leads to an expression for $V_{Th,1}$, $I_{DS}$ is derived by integrating the total quantity of charge $Q_{n1}$:

	\begin{equation}
	I_{DS} = - \frac{\mu_n W}{L}\int_{\Phi_{S1}(0)}^{\Phi_{S1}(L)}Q_{n1}(y)\,d\Phi_{S1}
	\end{equation}

 As it has been done before, we have to distinguish two cases while solving the integral, to derive the final result:

\begin{itemize}
\item when the back-gate is accumulated:
	\begin{equation}
	I_{DS} = \frac{\mu_n W C_{ox1}}{L}\Bigl[(V_{G1} - V_{Th1,acc2})V_{DS} - \frac{m_{acc2}}{2}V_{DS}^2\Bigr]
	\end{equation}

	\begin{equation}
	m = m_{acc2} = \Bigl(1 + \frac{C_{Si}}{C_{ox1}}\Bigr)
	\end{equation}
Notice that since both source and drain regions of the back-gate are accumulated, we refer to correction term $m$ as $m_{acc2}$. The same subscript it is used for the threshold voltage expression.

\item When the back-gate is inverted:
	\begin{equation}
	I_{DS} = \frac{\mu_n W C_{ox1}}{L}\Bigl[(V_{G1} - V_{Th1,dep2})V_{DS} - \frac{m_{dep2}}{2}V_{DS}^2\Bigr]
	\end{equation}

	\begin{equation}
	m = m_{dep2} = \Bigl(1 + \frac{1}{C_{ox1}}\frac{C_{Si} C_{ox2}}{C_{Si} + C_{ox2}}\Bigr)
	\end{equation}
Notice that since both source and drain regions of the back-gate are depleted, we refer to correction term $m$ as $m_{dep2}$. The same subscript it is used for the threshold voltage expression.

\end{itemize}

% Here the silicon-overlay capacitance $C_{Si}=\frac{\epsilon_s}{T_{Si}}$

%	\begin{eqnarray*}
%	I_{DS} = \frac{\mu_n W C_{ox1}}{L}\Bigl[(V_{G1} - V_{Th,1})V_{DS} - \frac{m}{2}V_{DS}^2\Bigr]\\
%	m = \Bigl(1 + \frac{C_{Si}}{C_{ox1}}\Bigr) 
%	\end{eqnarray*}

%	\begin{equation}
%	I_{DS} = \frac{\mu_n W C_{ox1}}{L}\Bigl[(V_{G1} - V_{Th,1})V_{DS} - \frac{m}{2}V_{DS}^2\Bigr]
%	\end{equation}
%%	\begin{equation}
%	m = \Bigl(1 + \frac{C_{Si}}{C_{ox1}}\Bigr)
%	\end{equation}

%%In the last equation notice that both source and drain region are accumulated.

\newpage

\section{Main parameters for MG scaling}
The tables from the Roadmap for multi-gate technologies are listed below in this section. As it can be observed, there are some parameters that are the same value as for Bulk process (section \ref{sec:bulk}). Once again, the color legend indicates for a certain parameter whether there exist manufacturing solutions or not. Please refer to Table \ref{tab:color}.\\
There are several changes to highlight when comparing the values of the parameters of LP to HP. Firstly, the $I_{off}$ current differs from HP to LP by several orders of magnitude, and this behavior is consistent both for Bulk and MG technology. Secondly, the $I_{on}$ current is significantly higher in HP than it is in LP (both for MG and Bulk) even though it starts to gradually drop in 2015. In the future, lower drive current values will be determined by a higher threshold voltage, which would be required to keep SCE under control. Another important parameter $\tau$ will constantly decrease over the years, showing an even better behavior for the HP family in MG process\footnote{Consequently, the inverse of $\tau$ (i.e. speed parameter) will increase.}. Finally, a last important difference to be underlined is the level of channel doping $Doping_{ch}$: in the past decades there was a trend that would forecast a continuous increase in the doping level of the channel. That was done with the aim to reduce the SCE in the device\footnote{Doping level almost reached the limit value of $10^{20} cm^{-3}$.}. Nowadays, MG technology does not require such a high level of doping, therefore it does not follow the traditional rules of scaling about this parameter.

\begin{table}[h]
\centering
\resizebox{\textwidth}{!}{\begin{tabular}{||l||c|c|c|c|c|c|c|c|c|c|c|c||}
\hline
Year&2013&2014&2015&2016&2017&2018&2019&2020&2021&2022&2023&2024\\
\hline
$L_{g}$ [nm]&20&18&16.7&15.2&13.9&12.7&11.6&10.6&\ye 9.7&\ye 8.8&\ye 8&\ye 7.3\\
\hline
$L_{ch}$ [nm]&16&14.4&13.4&12.2&11.1&10.2&9.3&8.5&\ye 7.8&\ye 7&\ye 6.4&\ye 5.8\\
\hline
$V_{dd}$ [V]&0.86&0.85&0.83&0.81&0.8&0.78&0.77&0.75&0.74&0.72&0.71&0.69\\
\hline
EOT [nm]&0.8&0.77&0.73&\ye 0.7&\ye 0.67&\ye 0.64&\ye 0.61&\ye 0.59&\ye 0.56&\ye 0.54&\ye 0.51&\re 0.49\\
\hline
$\epsilon_{HK}$&12.5&13&13.5&\ye 14&\ye 14.5&\ye 15&\ye 15.5&\ye 16&\ye 16.5&\ye 17&\ye 17.5&\re 18\\
\hline
$T_{HK}$ [nm]&2.56&2.57&2.53&\ye 2.51&\ye 2.49&\ye 2.46&\ye 2.42&\ye 2.42&\ye 2.37&\ye 2.35&\ye 2.29&\re 2.26\\
\hline
$Doping_{ch}$ [$10^{18}$/$cm^{3}$]&0.1&0.1&0.1&0.1&0.1&0.1&0.1&0.1&0.1&0.1&0.1&0.1\\
\hline
$T_{body}$ [nm]&6.4&5.8&\ye 5.3&\ye 4.9&\ye 4.4&\re 4.1&\re 3.7&\re 3.4&\re 3.1&\re 2.8&\re 2.6&\re 2.3\\
\hline
CET [nm]&1.1&1.07&1.03&\ye 1&\ye 0.97&\ye 0.94&\ye 0.91&\ye 0.89&\ye 0.86&\ye 0.84&\ye 0.81&\re 0.79\\
\hline
$C_{ch}$ [fF/$\mu$m]&0.502&0.465&0.448&\ye 0.42&\ye 0.396&\ye 0.373&\ye 0.352&\ye 0.329&\ye 0.311&\ye 0.289&\ye 0.273&\re 0.255\\
\hline
$\mu$ [$cm^{2}$/V*s]&250&250&250&250&250&200&200&200&200&200&200&150\\
\hline
$I_{off}$ [pA/$\mu$m]&100&100&100&100&100&100&100&100&100&100&100&100\\
\hline
$I_{on}$ [$\mu$A/$\mu$m]&1670&1680&\ye 1700&\ye 1660&\ye 1660&\re 1610&\re 1600&\re 1480&\re 1450&\re 1350&\re 1330&\re 1170\\
\hline
$V_{t,lin}$ [V]&0.219&0.225&0.231&0.239&0.264&0.266&0.265&0.276&0.295&0.303&0.306&0.319\\
\hline
$V_{t,sat}$ [V]&0.174&0.179&0.179&0.186&0.191&0.196&0.199&0.205&0.214&0.219&0.222&0.233\\
\hline
S/D $R_{s}$ [$\omega$*$\mu$m]&128&146&\ye 130&\ye 126&\ye 124&\re 117&\re 120&\re 116&\re 112&\re 111&\re 113&\re 123\\
\hline
$C_{fringing}/C_{intrinsic}$ &1.2&1.3&1.4&1.5&1.6&1.7&1.8&1.9&2&2&2&2\\
\hline
$C_{fringing}$ [fF/$\mu$m]&0.6&0.6&0.63&0.63&0.63&0.63&0.63&0.62&0.62&0.58&0.55&0.51\\
\hline
$C_{g,total}$ [fF/$\mu$m]&1.1&1.07&1.07&1.05&1.03&1.01&0.99&0.95&0.93&0.87&0.82&0.77\\
\hline
$CV^{2}$ [fJ/$\mu$m]&0.82&0.77&0.74&0.69&0.66&0.61&0.58&0.54&0.51&0.45&0.41&0.36\\
\hline
$\tau$ [ps]&0.569&0.541&\ye 0.525&\ye 0.512&\ye 0.496&\re 0.488&\re 0.474&\re 0.483&\re 0.477&\re 0.463&\re 0.437&\re 0.451\\
\hline
1/$\tau$ [1/ps]&1.76&1.85&\ye 1.91&\ye 1.95&\ye 2.02&\re 2.05&\re 2.11&\re 2.07&\re 2.1&\re 2.16&\re 2.29&\re 2.22\\
\hline
$I_{d,satn}/I_{d,satp}$&1.25&1.24&1.22&1.21&1.2&1.19&1.18&1.16&1.15&1.14&1.13&1.12\\
\hline
\end{tabular}}
\caption{\textbf{MG process for HP (ITRS 2013)}}
\end{table}


\begin{table}[H]
\centering
\resizebox{\textwidth}{!}{\begin{tabular}{||l||c|c|c|c|c|c|c|c|c|c|c|c||}
\hline
Year&2013&2014&2015&2016&2017&2018&2019&2020&2021&2022&2023&2024\\
\hline
$L_{g}$ [nm]&23&21&19&18&16&14.6&13.3&12.2&11.1&10.1&\ye 9.3&\ye 8.5\\
\hline
$L_{ch}$ [nm]&18.4&16.8&15.2&14.4&12.8&11.7&10.6&9.8&8.9&8.1&\ye 7.4&\ye 6.8\\
\hline
$V_{dd}$ [V]&0.86&0.85&0.83&0.81&0.8&0.78&0.77&0.75&0.74&0.72&0.71&0.69\\
\hline
EOT [nm]&0.8&0.77&0.73&\ye 0.7&\ye 0.67&\ye 0.64&\ye 0.61&\ye 0.59&\ye 0.56&\ye 0.54&\ye 0.51&\re 0.49\\
\hline
$\epsilon_{HK}$&12.5&13&13.5&\ye 14&\ye 14.5&\ye 15&\ye 15.5&\ye 16&\ye 16.5&\ye 17&\ye 17.5&\re 18\\
\hline
$T_{HK}$ [nm]&2.56&2.57&2.53&\ye 2.51&\ye 2.49&\ye 2.46&\ye 2.42&\ye 2.42&\ye 2.37&\ye 2.35&\ye 2.29&\re 2.26\\
\hline
$Doping_{ch}$ [$10^{18}$/$cm^{3}$]&0.1&0.1&0.1&0.1&0.1&0.1&0.1&0.1&0.1&0.1&0.1&0.1\\
\hline
$T_{body}$ [nm]&7.4&6.7&6.1&5.8&\ye 5.1&\ye 4.7&\ye 4.3&\re 3.9\re &\re 3.6&\re 3.2&\re 3&\re 2.7\\
\hline
CET [nm]&1.1&1.07&1.03&\ye 1&\ye 0.97&\ye 0.94&\ye 0.91&\ye 0.89&\ye 0.86&\ye 0.84&\ye 0.81&\re 0.79\\
\hline
$C_{ch}$ [fF/$\mu$m]&0.577&0.542&0.509&\ye 0.497&\ye 0.455&\ye 0.429&\ye 0.404&\ye 0.379&\ye 0.356&\ye 0.332&\ye 0.317&\re 0.297\\
\hline
$\mu$ [$cm^{2}$/V*s]&375&375&375&375&375&375&375&300&300&300&300&300\\
\hline
$I_{off}$ [pA/$\mu$m]&10&10&10&10&10&10&10&20&20&20&20&20\\
\hline
$I_{on}$ [$\mu$A/$\mu$m]&643&610&\ye 618&\ye 589&\ye 574&\re 556&\re 550&\re 533&\re 537&\re 461&\re 458&\re 395\\
\hline
$V_{t,lin}$ [V]&0.483&0.492&0.492&0.496&0.507&0.507&0.51&0.501&0.507&0.521&0.511&0.52\\
\hline
$V_{t,sat}$ [V]&0.446&0.453&0.453&0.454&0.461&0.459&0.461&0.447&0.446&0.454&0.453&0.46\\
\hline
S/D $R_{s}$ [$\omega$*$\mu$m]&128&146&\ye 130&\ye 126&\ye 124&\re 117&\re 120&\re 116&\re 112&\re 111&\re 113&\re 123\\
\hline
$C_{fringing}/C_{intrinsic}$ &1.1&1.2&1.3&1.4&1.5&1.6&1.7&1.8&1.9&2&2&2\\
\hline
$C_{fringing}$ [fF/$\mu$m]&0.64&0.65&0.66&0.7&0.68&0.69&0.69&0.68&0.68&0.66&0.63&0.59\\
\hline
$C_{g,total}$ [fF/$\mu$m]&1.21&1.19&1.17&\ye 1.19&\ye 1.14&\ye 1.12&\ye 1.09&\ye 1.06&\ye 1.03&\ye 1&\ye 0.95&\re 0.89\\
\hline
$CV^{2}$ [fJ/$\mu$m]&0.9&0.86&0.81&\ye 0.78&\ye 0.73&\ye 0.68&\ye 0.65&\ye 0.6&\ye 0.57&\ye 0.52&\ye 0.48&\re 0.42\\
\hline
$\tau$ [ps]&1.622&1.661&\ye 1.573&\ye 1.64&\ye 1.587&\re 1.564&\re 1.525&\re 1.491&\re 1.424&\re 1.556&\re 1.474&\re 1.557\\
\hline
1/$\tau$ [1/ps]&0.62&0.6&\ye 0.64&\ye 0.61&\ye 0.63&\re 0.64&\re 0.66&\re 0.67&\re 0.7&\re 0.64&\re 0.68&\re 0.64\\
\hline
$I_{d,satn}/I_{d,satp}$&1.27&1.26&1.25&1.24&1.22&1.21&1.2&1.19&1.18&1.16&1.15&1.14\\
\hline
\end{tabular}}
\caption{\textbf{MG process for LP (ITRS 2013)}}
\end{table}

A common trend in both HP and LP families is given by the threshold voltage: it is predicted that the $V_{th}$ will slowly increase with years as to \textbf{limit subthreshold currents} that otherwise would become too large, causing more than one problem in terms of power consumption and heat dissipation. \\ Similarly to what has been done in the Bulk section, the graphs of the different parameters are reported in Figure (\ref{MGdiagrams}). Each graph shows for every parameter the evolution over the years both for HP and LP. Compared to the Bulk diagrams previously showed, the diagrams on MG technology are spread on a longer time range up to year 2024, as the MG process is supposed to be the technology of the next future.

\begin{figure}[htp] {
\centering 
%
\subfloat[Dynamic power $CV^{2}$ analysis.]{
	\includegraphics[width=0.4\textwidth]{figCV2mg} \label{CV^2}} 
\qquad
%
\subfloat[Ion current as a function of time.]{
	\includegraphics[width=0.4\textwidth]{figIonmg} \label{}} 
\qquad
%
\subfloat[Ioff current as a function of time.]{
	\includegraphics[width=0.4\textwidth]{figIoffmg} \label{}} 
\qquad
%
\subfloat[Ion/Ioff ratio.]{
	\includegraphics[width=0.43\textwidth]{figOnOffmg} \label{Ion-Ioff}} 
\qquad
%
\subfloat[Intrinsic delay as a function of time.]{
	\includegraphics[width=0.4\textwidth]{figtaumg} \label{tau}}  
\qquad
%
\subfloat[$V_{DD}/V_{T}$ in function of time.]{
	\includegraphics[width=0.4\textwidth]{figVddVtmg} \label{}} 
\caption{Parameters for MG technology.}
\label{MGdiagrams} }
\end{figure}

\newpage
\section{New devices and solutions}
In order to maintain the historical trend described by Moore's Law, the requirement for new architectures of devices is to keep up with aggressive, continuous scaling. As EUV lithography is far from its cost-effective, mass-production utilization, new integration techniques (e.g., self-aligned contacts, new materials, etc\dots) are needed for technology nodes beyond a gate length of 10 nm\cite{7153333}. Among all new devices and structures that have been proposed in the past few years, so far the most important and successful transistor has been the \textbf{finFET}. The finFET is the very first 3D device, where the gate is wrapped around three sides of the channel. As a matter of fact, finFETs can be either \emph{dual-gate} or \emph{tri-gate} transistor. The main difference consists of a larger dielectric thickness on top of the fin for the dual-gate structure, so the gate voltage does not control the channel from the top. In tri-gate devices the thickness of the dielectric around the fin is uniform and the gate controls the channel conductivity from three different sides. The process for the tri-gate structure is simpler when reducing dimensions since the gate length scales as follows:

\begin{itemize}
\item $L_{g} < D_{fin}$ for tri-gate transistors;
\item $L_{g} < D_{fin}/2$ for dual-gate transistors: this requirement is \textbf{stricter} than the one above; as a matter of fact, tri-gate devices are easier to manufacture than dual-gate ones.
\end{itemize}

\begin{figure}[h]
\centering
\includegraphics[width=0.35\textwidth]{finFET_schema}
\caption{Simple 3D schematic of a finFET device.}
\label{fig:finFET}
\end{figure} 
%% SITO http://spie.org/newsroom/technical-articles/4743-directed-self-assembly-for-ever-smaller-printed-circuits

$D_{fin}$ is the fin thickness as shown in Figure (\ref{fig:finFET}). The last geometrical quantity is the fin height $H_{fin}$, which allows us to define the aspect ratio for the fin. One of the most important differences between a standard planar process like Bulk or FD-SOI and a MG technology like finFET is that \textbf{width is no more an independent parameter} left to the designer's choice, but it is determined by the technology:

\begin{itemize}
\item $W_{fin}=2*H_{fin}+D_{fin}$ for tri-gate finFETs;
\item $W_{fin}=2*H_{fin}$ for dual-gate finFETs.
\end{itemize}

\textbf{The electrostatic control of the channel is largely dependent on $W_{fin}$}: thus, there must exist an \emph{optimum value} of $W_{fin}$ that minimizes variability: the \textbf{optimum $L_{g}/W_{fin}$ ratio is about 0.33} for technology nodes ranging from 6nm to 14nm\cite{7175588} as it is also showed in Fig. (\ref{fig:Wfin_Lg}):

\begin{figure}[H]
\centering
\includegraphics[width=0.42\textwidth]{optimum_Wfin_Lg}
\caption{The graph shows both optimum $W_{fin}$ and $L_{g}/W_{fin}$ vs technology node.}
\label{fig:Wfin_Lg}
\end{figure} 

According to the value of the aspect ratio, there are three different cross-sections that can be considered:

\begin{enumerate}
\item \textbf{tall} finFET that carries the advantage of having a large $W_{fin}$, thus permitting a greater drive current;
\item \textbf{short} finFET that has the advantage of less challenging lithography and etching steps;
\item \textbf{nanowire} finFET that gives the gate even more control on the channel by surrounding it.
\end{enumerate}

As it can be observed from Figure (\ref{fig:finFET}), the \textbf{channel region between source and drain is wrapped around by the gate: that is the reason why we can obtain a much better channel control even at very short gate length}.\\
Since the fin structure tends to be characterized by a narrow cross-section and long S/D distance, the resistance of the channel is a critical aspect to consider. There are two solutions that are adopted in order to increase $I_{on}$: the \emph{Stress Memorization Technique (STM)} helps by increasing the carrier mobility, while the exploitation of larger number of fins per gate allows to increases the total current that flows between the source and drain regions (\emph{multi-fin structures}).
Since scaling is supposed to continue over and over, the \emph{contacted gate pitch (CGP)} will be forced to scale endlessly. Continuous scaling cannot be achieved without a significant tradeoff between different quantities, namely: $L_{g}$, spacer thickness (spacers isolate S/D regions from the gate) and S/D contact dimesion. In particular, $L_{g}$ can be continuously scaled only by reducing $D_{fin}$, leading to a very large aspect ratio. High aspect ratio fins are difficult to manufacture as tall fins with straight profiles are necessary to reach a good control of the channel. It is straight-forward that procces costs are likely to increase as these structures become smaller and smaller. New solutions that could be implemented as a low-cost alternative are at study, e.g. \emph{Ultra-Shallow Junction Extensions (USEJ)} and \emph{recessed-channel (U-channel)} devices that have proved to be effective in controlling DIBL and lowering SS in scaling length to the 18-20 nm range\cite{7339689}.
%% PAPER 0707886 sito IEEE
Reliabilty requirements and/or capacitance between gate and S/D regions tell us what is the minimum value for the spacer thickness. Along with finFETs, the other two main structures that are being investigated and developed nowadays:

\begin{itemize}
\item \emph{lateral Gate-All-Around FETs (GAAFETs)} consisting of horizontally-stacked nanowires that allow more relaxed constraints on the channel length while still providing similar control on SCE and it is possible to avoid a large variability in the threshold voltage;
\item \emph{vertical GAAFETs} have even more relaxed on the gate length since they are vertically oriented and show a better scalability. 
\end{itemize}

For the sake of clearness, a schematic of the different structures is showed below.
\begin{figure}[h]
\centering
\includegraphics[width=0.8\textwidth]{finFET}
\caption{3D schematics of finFET, lateral GAAFET and the vertical one as reported from\cite{7078860}.}
\label{fig:finFET}
\end{figure} 

What has been stated above and in the whole chapter is an overview about scaling and how it has changed affected technological processes as new solutions have become necessary when old technologies could not improve anymore. While the Bulk process is reaching its natural death in 2017, new transistors like finFETs and GAAFETs bring us new challenges and problems. It will be our duty to come up with smart solutions that could give us better and more reliable technologies. 

\bibliographystyle{plain}
\bibliography{scaling_bib}
\end{document}
